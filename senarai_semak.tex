\documentclass{article}
\title{Senarai Semak}
\begin{document}

\textbf{:pengurusan:} Sistem \textbf{pengurusan} kualiti - Keperluan

\textbf{:tanggungjawab:} MS yang disahkan: Malaysian Standard yang telah dikaji semula oleh jawatankuasa yang
ber\textbf{tanggungjawab} dan mengesahkan bahawa kandungannya adalah terkini.

\textbf{:pengurusan:} Jawatankuasa Standard Perindustrian mengenai \textbf{pengurusan} Kualiti dan Penentuan Kualiti (ISC Y) yang di bawah
kuasanya Malaysian Standard ini diterima guna dianggotai oleh wakil daripada organisasi yang berikut:

\textbf{:pengurusan:} Jawatankuasa Teknikal mengenai \textbf{pengurusan} Kualiti dan Penentuan Kualiti (TC2) mengenai Sistem Kualiti yang
mengesyorkan penerimagunaan Standard ISO dianggotai oleh wakil daripada organisasi yang berikut:

\textbf{:pengurusan:} Dewan Perdagangan dan Industri Antarabangsa Malaysia
Gabungan Pembekal-pembekal Perkhidmatan Malaysia
Institut Penyelidikan Sains dan Teknologi Pertahanan
Institute of Quality Malaysia
Jabatan Standard Malaysia
Lembaga Pembangunan Industri Pembinaan Malaysia
SIRIM Berhad (Sekretariat)
SIRIM QAS International Sdn Bhd
Unit Pemodenan Tadbiran dan Perancangan \textbf{pengurusan} Malaysia
Universiti Malaya
Universiti Utara Malaysia

\textbf{:pengurusan:} Dewan Bahasa dan Pustaka Malaysia
Institut Terjemahan & Buku Malaysia
Institute of Quality Malaysia
Jabatan Standard Malaysia
Perbadanan Produktiviti Malaysia
SIRIM Berhad (Sekretariat)
SIRIM QAS International Sdn Bhd
Unit Pemodenan Tadbiran dan Perancangan \textbf{pengurusan} Malaysia
Universiti Utara Malaysia

\textbf{:pengurusan:} Penerimagunaan Standard ISO sebagai Malaysian Standard telah disyorkan oleh
Jawatankuasa Teknikal mengenai \textbf{pengurusan} Kualiti dan Penentuan Kualiti (TC2) mengenai
Sistem Kualiti di bawah kuasa Jawatankuasa Standard Perindustrian mengenai \textbf{pengurusan}
Kualiti dan Penentuan Kualiti.

\textbf{:tanggungjawab:} Perhatian perlu diberikan terhadap kemungkinan terdapat beberapa unsur dalam dokumen ini
boleh menjadi hal perkara hak paten. ISO tidak boleh diper\textbf{tanggungjawab}kan untuk
mengenal pasti apa-apa atau semua hak paten itu. Perincian mana-mana hak paten yang
dikenal pasti semasa pembangunan dokumen akan dimasukkan ke dalam Pengenalan
dan/atau senarai perisytiharan paten ISO yang diterima (lihat [http://www.iso.org/patents).](http://www.iso.org/patents).)

\textbf{:senarai:} Perhatian perlu diberikan terhadap kemungkinan terdapat beberapa unsur dalam dokumen ini
boleh menjadi hal perkara hak paten. ISO tidak boleh diper\textbf{tanggungjawab}kan untuk
mengenal pasti apa-apa atau semua hak paten itu. Perincian mana-mana hak paten yang
dikenal pasti semasa pembangunan dokumen akan dimasukkan ke dalam Pengenalan
dan/atau \textbf{senarai} perisytiharan paten ISO yang diterima (lihat [http://www.iso.org/patents).](http://www.iso.org/patents).)

\textbf{:tanggungjawab:} Jawatankuasa yang ber\textbf{tanggungjawab} tentang dokumen ini ialah Technical Committee
ISO/TC 176, Quality management and quality assurance, Subcommittee SC 2, Quality
systems.

\textbf{:pengurusan:} Edisi kelima ini membatalkan dan menggantikan edisi keempat (ISO 9001:2008), yang telah
disemak semula dari segi teknikal, melalui penerimagunaan kajian semula urutan klausa dan
penyesuaian kajian semula prinsip \textbf{pengurusan} kualiti dan konsep baharu. Edisi ini juga
membatalkan dan menggantikan Technical Corrigendum ISO 9001:2008/Cor.1:2009.

\textbf{:pengurusan:} Penerimagunaan sistem \textbf{pengurusan} kualiti ialah suatu keputusan strategik untuk sesebuah
organisasi yang boleh membantu untuk meningkatkan prestasi secara menyeluruh dan
menyediakan asas yang kukuh bagi inisiatif pembangunan mampan.

\textbf{:prestasi:} Penerimagunaan sistem \textbf{pengurusan} kualiti ialah suatu keputusan strategik untuk sesebuah
organisasi yang boleh membantu untuk meningkatkan \textbf{prestasi} secara menyeluruh dan
menyediakan asas yang kukuh bagi inisiatif pembangunan mampan.

\textbf{:dasar:} Faedah yang mungkin kepada organisasi dalam melaksanakan sistem pengurusan kualiti
ber\textbf{dasar}kan Standard Antarabangsa ini adalah:

\textbf{:pengurusan:} Faedah yang mungkin kepada organisasi dalam melaksanakan sistem \textbf{pengurusan} kualiti
ber\textbf{dasar}kan Standard Antarabangsa ini adalah:

\textbf{:pelanggan:} a) keupayaan untuk menyediakan secara tekal produk dan perkhidmatan yang memenuhi
keperluan \textbf{pelanggan}, serta keperluan berkanun dan peraturan yang diguna pakai;

\textbf{:pelanggan:} b) memudahkan peluang untuk meningkatkan kepuasan \textbf{pelanggan};

\textbf{:pengurusan:} d) keupayaan untuk menunjukkan keakuran terhadap keperluan sistem \textbf{pengurusan} kualiti
yang ditetapkan.

\textbf{:pengurusan:} - keseragaman dalam struktur sistem \textbf{pengurusan} kualiti yang berbeza;
- keselarasan pendokumenan dengan struktur klausa Standard Antarabangsa ini;
- penggunaan istilah khusus Standard Antarabangsa ini dalam organisasi.

\textbf{:pengurusan:} Keperluan sistem \textbf{pengurusan} kualiti yang dinyatakan dalam Standard Antarabangsa ini
merupakan pelengkap kepada keperluan untuk produk dan perkhidmatan.

\textbf{:penambahbaikan:} Kitaran PDCA membolehkan sesuatu organisasi memastikan yang prosesnya disumberkan
dan diuruskan secukupnya, dan peluang untuk \textbf{penambahbaikan} ditentukan dan diambil
tindakan.

\textbf{:kawalan:} Pemikiran berasaskan risiko membolehkan sesuatu organisasi menentukan faktor yang boleh
menyebabkan proses dan sistem pengurusan kualitinya menyimpang daripada hasil yang
dirancang, menyediakan \textbf{kawalan} pencegahan untuk meminimumkan kesan negatif dan
untuk menggunakan secara maksimum peluang yang wujud (lihat Klausa A.4).

\textbf{:pengurusan:} Pemikiran berasaskan risiko membolehkan sesuatu organisasi menentukan faktor yang boleh
menyebabkan proses dan sistem \textbf{pengurusan} kualitinya menyimpang daripada hasil yang
dirancang, menyediakan \textbf{kawalan} pencegahan untuk meminimumkan kesan negatif dan
untuk menggunakan secara maksimum peluang yang wujud (lihat Klausa A.4).

\textbf{:penambahbaikan:} Memenuhi kehendak serta menangani keperluan dan jangkaan masa depan secara tekal
menjadi cabaran kepada organisasi dalam persekitaran yang semakin dinamik dan kompleks.
Bagi mencapai matlamat ini, organisasi mungkin perlu mengguna terima pelbagai bentuk
\textbf{penambahbaikan} sebagai tambahan kepada pembetulan dan \textbf{penambahbaikan} berterusan,
seperti perubahan besar, inovasi dan penyusunan semula organisasi.

\textbf{:pengurusan:} \#\#\# 0.2 Prinsip \textbf{pengurusan} kualiti

\textbf{:dasar:} Standard Antarabangsa ini ber\textbf{dasar}kan prinsip pengurusan kualiti yang diperihalkan dalam
ISO 9000. Pemerihalan termasuk pernyataan tentang setiap prinsip, rasional tentang sebab
prinsip itu penting bagi organisasi, beberapa contoh manfaat yang berkaitan dengan prinsip
itu dan contoh tindakan biasa untuk menambah baik prestasi organisasi apabila
menggunakan prinsip itu.

\textbf{:pengurusan:} Standard Antarabangsa ini ber\textbf{dasar}kan prinsip \textbf{pengurusan} kualiti yang diperihalkan dalam
ISO 9000. Pemerihalan termasuk pernyataan tentang setiap prinsip, rasional tentang sebab
prinsip itu penting bagi organisasi, beberapa contoh manfaat yang berkaitan dengan prinsip
itu dan contoh tindakan biasa untuk menambah baik prestasi organisasi apabila
menggunakan prinsip itu.

\textbf{:prestasi:} Standard Antarabangsa ini ber\textbf{dasar}kan prinsip \textbf{pengurusan} kualiti yang diperihalkan dalam
ISO 9000. Pemerihalan termasuk pernyataan tentang setiap prinsip, rasional tentang sebab
prinsip itu penting bagi organisasi, beberapa contoh manfaat yang berkaitan dengan prinsip
itu dan contoh tindakan biasa untuk menambah baik \textbf{prestasi} organisasi apabila
menggunakan prinsip itu.

\textbf{:pengurusan:} Prinsip \textbf{pengurusan} kualiti adalah seperti yang berikut:

\textbf{:pelanggan:} - fokus \textbf{pelanggan};
- kepimpinan;

\textbf{:penambahbaikan:} − \textbf{penambahbaikan};

\textbf{:pengurusan:} − \textbf{pengurusan} perhubungan.

\textbf{:pengurusan:} Standard Antarabangsa ini menggalakkan penerimagunaan pendekatan proses apabila
membangunkan, melaksanakan dan menambah baik keberkesanan sistem \textbf{pengurusan}
kualiti, bagi meningkatkan kepuasan pelanggan dengan cara memenuhi keperluan
pelanggan. Keperluan khusus yang dianggap penting dalam penerimagunaan pendekatan
proses dimasukkan dalam 4 .4.

\textbf{:pelanggan:} Standard Antarabangsa ini menggalakkan penerimagunaan pendekatan proses apabila
membangunkan, melaksanakan dan menambah baik keberkesanan sistem \textbf{pengurusan}
kualiti, bagi meningkatkan kepuasan \textbf{pelanggan} dengan cara memenuhi keperluan
\textbf{pelanggan}. Keperluan khusus yang dianggap penting dalam penerimagunaan pendekatan
proses dimasukkan dalam 4 .4.

\textbf{:prestasi:} Memahami dan mengurus proses saling berkait sebagai suatu sistem adalah menyumbang
kepada keberkesanan dan kecekapan organisasi dalam mencapai hasil yang dimaksudkan.
Pendekatan ini membolehkan organisasi mengawal saling hubungan dan saling
kebergantungan antara proses sistem itu, supaya \textbf{prestasi} keseluruhan organisasi boleh
dipertingkatkan.

\textbf{:dasar:} Pendekatan proses melibatkan pentakrifan dan pengurusan proses yang sistematik, dan
saling tindaknya, untuk mencapai hasil yang dimaksudkan selaras dengan \textbf{dasar} kualiti dan
hala tuju strategik organisasi. Pengurusan proses dan sistem secara keseluruhan boleh
dicapai dengan menggunakan kitaran PDCA (lihat 0.3.2) dengan tumpuan keseluruhan
terhadap pemikiran berasaskan risiko (lihat 0.3.3) yang bertujuan untuk memanfaatkan
peluang dan mencegah keputusan yang tidak diingini.

\textbf{:pengurusan:} Pendekatan proses melibatkan pentakrifan dan \textbf{pengurusan} proses yang sistematik, dan
saling tindaknya, untuk mencapai hasil yang dimaksudkan selaras dengan \textbf{dasar} kualiti dan
hala tuju strategik organisasi. \textbf{pengurusan} proses dan sistem secara keseluruhan boleh
dicapai dengan menggunakan kitaran PDCA (lihat 0.3.2) dengan tumpuan keseluruhan
terhadap pemikiran berasaskan risiko (lihat 0.3.3) yang bertujuan untuk memanfaatkan
peluang dan mencegah keputusan yang tidak diingini.

\textbf{:pengurusan:} Pemakaian pendekatan proses dalam sistem \textbf{pengurusan} kualiti membolehkan:

\textbf{:prestasi:} c) pencapaian \textbf{prestasi} proses yang berkesan;

\textbf{:dasar:} d) penambahbaikan proses ber\textbf{dasar}kan penilaian data dan maklumat.

\textbf{:penambahbaikan:} d) \textbf{penambahbaikan} proses ber\textbf{dasar}kan penilaian data dan maklumat.

\textbf{:kawalan:} Rajah 1 memberi gambaran skematik proses dan menunjukkan saling tindak unsurnya.
Pemantauan dan pengukuran titik semak, yang diperlukan untuk \textbf{kawalan}, adalah khusus
untuk setiap proses dan berubah, bergantung pada risiko yang berkaitan.

\textbf{:ukur:} Rajah 1 memberi gambaran skematik proses dan menunjukkan saling tindak unsurnya.
Pemantauan dan peng\textbf{ukur}an titik semak, yang diperlukan untuk \textbf{kawalan}, adalah khusus
untuk setiap proses dan berubah, bergantung pada risiko yang berkaitan.

\textbf{:pengurusan:} Pemikiran berasaskan risiko (lihat Klausa A.4) adalah penting bagi mencapai sistem
\textbf{pengurusan} kualiti yang berkesan. Konsep pemikiran berasaskan risiko yang tersirat dalam
edisi terdahulu Standard Antarabangsa ini termasuk, sebagai contoh, menjalankan tindakan
pencegahan untuk menghapuskan ketakakuran yang mungkin berlaku, menganalisis apa-apa
ketakakuran yang berlaku, dan mengambil tindakan untuk mencegah berulangnya
ketakakuran, sesuai dengan kesan daripada ketakakuran itu.

\textbf{:ketakakuran:} Pemikiran berasaskan risiko (lihat Klausa A.4) adalah penting bagi mencapai sistem
\textbf{pengurusan} kualiti yang berkesan. Konsep pemikiran berasaskan risiko yang tersirat dalam
edisi terdahulu Standard Antarabangsa ini termasuk, sebagai contoh, menjalankan tindakan
pencegahan untuk menghapuskan \textbf{ketakakuran} yang mungkin berlaku, menganalisis apa-apa
\textbf{ketakakuran} yang berlaku, dan mengambil tindakan untuk mencegah berulangnya
\textbf{ketakakuran}, sesuai dengan kesan daripada \textbf{ketakakuran} itu.

\textbf{:pengurusan:} Sesuatu organisasi perlu merancang dan melaksanakan tindakan untuk menangani risiko dan
peluang supaya akur dengan keperluan Standard Antarabangsa ini. Penanganan kedua-dua
risiko dan peluang dapat menetapkan asas bagi meningkatkan keberkesanan sistem
\textbf{pengurusan} kualiti, mencapai hasil yang lebih baik dan mencegah kesan negatif.

\textbf{:pelanggan:} Peluang boleh muncul sebagai hasil daripada keadaan yang sesuai untuk mencapai hasil
yang dimaksudkan, sebagai contoh, suatu keadaan yang membolehkan organisasi menarik
\textbf{pelanggan}, membangunkan produk dan perkhidmatan baharu, mengurangkan pembaziran
atau menambah baik produktiviti. Tindakan untuk menangani peluang boleh juga termasuk
pertimbangan terhadap risiko yang berkaitan. Risiko ialah kesan ketakpastian, dan apa-apa
ketakpastian itu boleh mempunyai kesan positif atau negatif. Sisihan positif yang timbul
daripada risiko boleh memberi peluang, tetapi bukan semua kesan positif risiko menghasilkan
peluang.

\textbf{:pengurusan:} \#\#\# 0.4 Hubungan dengan standard sistem \textbf{pengurusan} yang lain

\textbf{:pengurusan:} Standard Antarabangsa ini menggunakan rangka kerja yang dibangunkan oleh ISO untuk
menambah baik keselarasan antara Standard Antarabangsa bagi sistem \textbf{pengurusan}nya
(lihat Klausa A.1).

\textbf{:pengurusan:} Standard Antarabangsa ini membolehkan sesuatu organisasi menggunakan pendekatan
proses, ditambah pula dengan kitaran PDCA dan pemikiran berasaskan risiko, untuk
menyelaraskan atau mengintegrasikan sistem \textbf{pengurusan} kualitinya dengan keperluan
standard sistem \textbf{pengurusan} yang lain.

\textbf{:pengurusan:} Lampiran B menyediakan perincian tentang Standard Antarabangsa \textbf{pengurusan} kualiti dan
sistem \textbf{pengurusan} kualiti lain yang telah dibangunkan oleh ISO/TC 176.

\textbf{:pengurusan:} Standard Antarabangsa ini tidak termasuk keperluan khusus untuk sistem \textbf{pengurusan} lain,
seperti \textbf{pengurusan} alam sekitar, \textbf{pengurusan} keselamatan dan kesihatan pekerjaan, atau
\textbf{pengurusan} kewangan.

\textbf{:dasar:} Standard sistem pengurusan kualiti sektor khusus yang ber\textbf{dasar}kan keperluan Standard
Antarabangsa ini telah dibangunkan untuk beberapa sektor. Sesetengah Standard ini
menetapkan keperluan sistem pengurusan kualiti tambahan, manakala yang lainnya terhad
kepada menyediakan panduan terhadap pemakaian Standard Antarabangsa ini dalam sektor
tertentu.

\textbf{:pengurusan:} Standard sistem \textbf{pengurusan} kualiti sektor khusus yang ber\textbf{dasar}kan keperluan Standard
Antarabangsa ini telah dibangunkan untuk beberapa sektor. Sesetengah Standard ini
menetapkan keperluan sistem \textbf{pengurusan} kualiti tambahan, manakala yang lainnya terhad
kepada menyediakan panduan terhadap pemakaian Standard Antarabangsa ini dalam sektor
tertentu.

\textbf{:pengurusan:} 
\#\#\# Sistem \textbf{pengurusan} kualiti - Keperluan

\textbf{:pengurusan:} Standard Antarabangsa ini menetapkan keperluan bagi satu sistem \textbf{pengurusan} kualiti
apabila sesuatu organisasi:

\textbf{:pelanggan:} a) perlu menunjukkan keupayaannya menyediakan secara tekal produk dan perkhidmatan
yang memenuhi keperluan \textbf{pelanggan} serta keperluan berkanun dan peraturan yang
diguna pakai, dan

\textbf{:pelanggan:} b) mempunyai tujuan untuk meningkatkan kepuasan \textbf{pelanggan} melalui pemakaian sistem
yang berkesan, termasuk proses penambahbaikan untuk sistem itu dan jaminan
keakuran terhadap keperluan \textbf{pelanggan} serta keperluan berkanun dan peraturan yang
diguna pakai.

\textbf{:penambahbaikan:} b) mempunyai tujuan untuk meningkatkan kepuasan \textbf{pelanggan} melalui pemakaian sistem
yang berkesan, termasuk proses \textbf{penambahbaikan} untuk sistem itu dan jaminan
keakuran terhadap keperluan \textbf{pelanggan} serta keperluan berkanun dan peraturan yang
diguna pakai.

\textbf{:pelanggan:} NOTA 1. Dalam Standard Antarabangsa ini, istilah "produk" atau "perkhidmatan" hanya diguna pakai
kepada produk dan perkhidmatan yang dimaksudkan untuk, atau diperlukan oleh, \textbf{pelanggan}.

\textbf{:pengurusan:} MS ISO 9000 (BM), Sistem \textbf{pengurusan} kualiti - Asas dan kosa kata

\textbf{:pengurusan:} Organisasi hendaklah menentukan isu luaran dan dalaman yang relevan dengan tujuan dan
haluan strategiknya serta yang memberi kesan kepada kebolehannya untuk mencapai hasil
yang dimaksudkan daripada sistem \textbf{pengurusan} kualitinya.

\textbf{:prestasi:} NOTA 3. Memahami konteks dalaman boleh dipermudahkan dengan mempertimbangkan isu yang
berkaitan dengan nilai, budaya, pengetahuan dan \textbf{prestasi} organisasi.

\textbf{:pelanggan:} Disebabkan oleh kesan atau kesan yang mungkin wujud terhadap keupayaan organisasi
untuk menyediakan secara tekal produk dan perkhidmatan yang memenuhi keperluan
\textbf{pelanggan}, serta keperluan berkanun dan peraturan yang diguna pakai, organisasi hendaklah
menentukan:

\textbf{:pengurusan:} a) pihak yang berkepentingan yang relevan dengan sistem \textbf{pengurusan} kualiti;

\textbf{:pengurusan:} b) keperluan pihak yang berkepentingan ini yang relevan dengan sistem \textbf{pengurusan} kualiti.

\textbf{:pengurusan:} \#\#\#\# 4.3 Menentukan skop sistem \textbf{pengurusan} kualiti

\textbf{:pengurusan:} Organisasi hendaklah menentukan sempadan dan kesesuaian sistem \textbf{pengurusan} kualiti
untuk mewujudkan skopnya.

\textbf{:pengurusan:} Organisasi hendaklah mengguna pakai semua keperluan Standard Antarabangsa ini jika ia
terpakai dalam skop yang ditetapkan dalam sistem \textbf{pengurusan} kualitinya.

\textbf{:pengurusan:} Skop sistem \textbf{pengurusan} kualiti organisasi hendaklah tersedia dan diselenggarakan sebagai
maklumat didokumentasikan. Skop ini hendaklah menyatakan jenis produk dan perkhidmatan
yang diliputinya, dan memberikan justifikasi bagi apa-apa keperluan Standard Antarabangsa
ini yang ditentukan oleh organisasi sebagai tidak terpakai untuk skop sistem \textbf{pengurusan}
kualitinya.

\textbf{:tanggungjawab:} Keakuran terhadap Standard Antarabangsa ini hanya boleh diakui, jika keperluan ditentukan
sebagai tidak terpakai, tidak menjejaskan keupayaan atau \textbf{tanggungjawab} organisasi bagi
memastikan keakuran produk dan perkhidmatannya dan peningkatan kepuasan pelanggan.

\textbf{:pelanggan:} Keakuran terhadap Standard Antarabangsa ini hanya boleh diakui, jika keperluan ditentukan
sebagai tidak terpakai, tidak menjejaskan keupayaan atau \textbf{tanggungjawab} organisasi bagi
memastikan keakuran produk dan perkhidmatannya dan peningkatan kepuasan \textbf{pelanggan}.

\textbf{:pengurusan:} \#\#\#\# 4.4 Sistem \textbf{pengurusan} kualiti dan prosesnya

\textbf{:pengurusan:} \#\#\#\#\# 4.4.1 Organisasi hendaklah mewujudkan, melaksanakan, menyelenggarakan dan 
menambah baik secara berterusan sistem \textbf{pengurusan} kualiti, termasuk proses yang
diperlukan dan saling tindaknya, selaras dengan keperluan Standard Antarabangsa ini.

\textbf{:pengurusan:} Organisasi hendaklah menentukan proses yang diperlukan untuk sistem \textbf{pengurusan} kualiti
dan pemakaiannya dalam seluruh organisasi, dan hendaklah:

\textbf{:kawalan:} c) menentukan dan mengguna pakai kriteria dan kaedah (termasuk pemantauan,
pengukuran dan petunjuk prestasi yang berkaitan) yang diperlukan bagi memastikan
keberkesanan operasi dan \textbf{kawalan} proses ini;

\textbf{:prestasi:} c) menentukan dan mengguna pakai kriteria dan kaedah (termasuk pemantauan,
pengukuran dan petunjuk \textbf{prestasi} yang berkaitan) yang diperlukan bagi memastikan
keberkesanan operasi dan \textbf{kawalan} proses ini;

\textbf{:ukur:} c) menentukan dan mengguna pakai kriteria dan kaedah (termasuk pemantauan,
peng\textbf{ukur}an dan petunjuk \textbf{prestasi} yang berkaitan) yang diperlukan bagi memastikan
keberkesanan operasi dan \textbf{kawalan} proses ini;

\textbf{:tanggungjawab:} e) menetapkan \textbf{tanggungjawab} dan bidang kuasa untuk proses ini;

\textbf{:pengurusan:} h) menambah baik proses dan sistem \textbf{pengurusan} kualiti.

\textbf{:komitmen:} \#\#\#\# 5.1 Kepimpinan dan \textbf{komitmen}

\textbf{:pengurusan:} \textbf{pengurusan} atasan hendaklah menunjukkan kepimpinan dan komitmen tentang sistem
\textbf{pengurusan} kualiti dengan:

\textbf{:komitmen:} \textbf{pengurusan} atasan hendaklah menunjukkan kepimpinan dan \textbf{komitmen} tentang sistem
\textbf{pengurusan} kualiti dengan:

\textbf{:pengurusan:} a) mengambil kebertanggungjawaban terhadap keberkesanan sistem \textbf{pengurusan} kualiti;

\textbf{:tanggungjawab:} a) mengambil keber\textbf{tanggungjawab}an terhadap keberkesanan sistem \textbf{pengurusan} kualiti;

\textbf{:objektif:} b) memastikan dasar kualiti dan \textbf{objektif} kualiti diwujudkan untuk sistem pengurusan kualiti
dan adalah bersesuaian dengan konteks dan hala tuju strategik organisasi;

\textbf{:dasar:} b) memastikan \textbf{dasar} kualiti dan \textbf{objektif} kualiti diwujudkan untuk sistem pengurusan kualiti
dan adalah bersesuaian dengan konteks dan hala tuju strategik organisasi;

\textbf{:pengurusan:} b) memastikan \textbf{dasar} kualiti dan \textbf{objektif} kualiti diwujudkan untuk sistem \textbf{pengurusan} kualiti
dan adalah bersesuaian dengan konteks dan hala tuju strategik organisasi;

\textbf{:pengurusan:} c) memastikan integrasi keperluan sistem \textbf{pengurusan} kualiti ke dalam proses perniagaan
organisasi;

\textbf{:pengurusan:} e) memastikan sumber yang diperlukan untuk sistem \textbf{pengurusan} kualiti adalah tersedia;

\textbf{:pengurusan:} f) mengkomunikasikan pentingnya \textbf{pengurusan} kualiti yang berkesan dan keakuran
terhadap keperluan sistem \textbf{pengurusan} kualiti;

\textbf{:komunikasi:} f) meng\textbf{komunikasi}kan pentingnya \textbf{pengurusan} kualiti yang berkesan dan keakuran
terhadap keperluan sistem \textbf{pengurusan} kualiti;

\textbf{:pengurusan:} g) memastikan bahawa sistem \textbf{pengurusan} kualiti mencapai hasil yang dimaksudkan;

\textbf{:pengurusan:} h) melibatkan, mengarah dan menyokong pekerja supaya menyumbang kepada
keberkesanan sistem \textbf{pengurusan} kualiti;

\textbf{:penambahbaikan:} i) menggalakkan \textbf{penambahbaikan};

\textbf{:pengurusan:} j) menyokong peranan \textbf{pengurusan} lain yang relevan untuk menunjukkan kepimpinannya,
seperti yang terpakai dalam bidang tanggungjawabnya.

\textbf{:tanggungjawab:} j) menyokong peranan \textbf{pengurusan} lain yang relevan untuk menunjukkan kepimpinannya,
seperti yang terpakai dalam bidang \textbf{tanggungjawab}nya.

\textbf{:pelanggan:} \#\#\#\#\# 5.1.2 Fokus kepada \textbf{pelanggan}

\textbf{:pengurusan:} \textbf{pengurusan} atasan hendaklah menunjukkan kepimpinan dan komitmen berkaitan fokus
kepada pelanggan dengan memastikan bahawa:

\textbf{:komitmen:} \textbf{pengurusan} atasan hendaklah menunjukkan kepimpinan dan \textbf{komitmen} berkaitan fokus
kepada pelanggan dengan memastikan bahawa:

\textbf{:pelanggan:} \textbf{pengurusan} atasan hendaklah menunjukkan kepimpinan dan \textbf{komitmen} berkaitan fokus
kepada \textbf{pelanggan} dengan memastikan bahawa:

\textbf{:pelanggan:} a) keperluan \textbf{pelanggan}, serta keperluan berkanun dan peraturan yang terpakai ditentukan,
difahami serta dipenuhi secara tekal;

\textbf{:pelanggan:} b) risiko dan peluang yang boleh memberi kesan kepada keakuran produk dan
perkhidmatan dan keupayaan untuk meningkatkan kepuasan \textbf{pelanggan} ditentukan dan
dinyatakan;

\textbf{:pelanggan:} c) fokus untuk meningkatkan kepuasan \textbf{pelanggan} dikekalkan.

\textbf{:dasar:} \#\#\#\# 5.2 \textbf{dasar}

\textbf{:dasar:} \#\#\#\#\# 5.2.1 Membangunkan \textbf{dasar} kualiti

\textbf{:dasar:} Pengurusan atasan hendaklah mewujudkan, melaksanakan dan menyelenggarakan \textbf{dasar}
kualiti yang:

\textbf{:pengurusan:} \textbf{pengurusan} atasan hendaklah mewujudkan, melaksanakan dan menyelenggarakan \textbf{dasar}
kualiti yang:

\textbf{:objektif:} b) menyediakan rangka kerja untuk menetapkan \textbf{objektif} kualiti;

\textbf{:komitmen:} c) mengandungi \textbf{komitmen} untuk memenuhi keperluan berkenaan;

\textbf{:pengurusan:} d) mengandungi komitmen untuk menambah baik secara berterusan sistem \textbf{pengurusan}
kualiti.

\textbf{:komitmen:} d) mengandungi \textbf{komitmen} untuk menambah baik secara berterusan sistem \textbf{pengurusan}
kualiti.

\textbf{:dasar:} \#\#\#\#\# 5.2.2 Mengkomunikasikan \textbf{dasar} kualiti

\textbf{:komunikasi:} \#\#\#\#\# 5.2.2 Meng\textbf{komunikasi}kan \textbf{dasar} kualiti

\textbf{:dasar:} \textbf{dasar} kualiti hendaklah:

\textbf{:komunikasi:} b) di\textbf{komunikasi}kan, difahami dan diguna pakai dalam organisasi;

\textbf{:tanggungjawab:} \#\#\#\# 5.3 Peranan, \textbf{tanggungjawab} dan bidang kuasa organisasi

\textbf{:pengurusan:} \textbf{pengurusan} atasan hendaklah memastikan bahawa tanggungjawab dan bidang kuasa bagi
peranan yang relevan ditetapkan, dikomunikasikan dan difahami dalam organisasi itu.

\textbf{:komunikasi:} \textbf{pengurusan} atasan hendaklah memastikan bahawa tanggungjawab dan bidang kuasa bagi
peranan yang relevan ditetapkan, di\textbf{komunikasi}kan dan difahami dalam organisasi itu.

\textbf{:tanggungjawab:} \textbf{pengurusan} atasan hendaklah memastikan bahawa \textbf{tanggungjawab} dan bidang kuasa bagi
peranan yang relevan ditetapkan, di\textbf{komunikasi}kan dan difahami dalam organisasi itu.

\textbf{:pengurusan:} \textbf{pengurusan} atasan hendaklah menetapkan tanggungjawab dan bidang kuasa bagi:

\textbf{:tanggungjawab:} \textbf{pengurusan} atasan hendaklah menetapkan \textbf{tanggungjawab} dan bidang kuasa bagi:

\textbf{:pengurusan:} a) memastikan sistem \textbf{pengurusan} kualiti akur terhadap keperluan Standard Antarabangsa
ini;

\textbf{:pengurusan:} c) melaporkan prestasi sistem \textbf{pengurusan} kualiti dan peluang untuk penambahbaikan (lihat
10.1), khususnya kepada \textbf{pengurusan} atasan;

\textbf{:prestasi:} c) melaporkan \textbf{prestasi} sistem \textbf{pengurusan} kualiti dan peluang untuk penambahbaikan (lihat
10.1), khususnya kepada \textbf{pengurusan} atasan;

\textbf{:penambahbaikan:} c) melaporkan \textbf{prestasi} sistem \textbf{pengurusan} kualiti dan peluang untuk \textbf{penambahbaikan} (lihat
10.1), khususnya kepada \textbf{pengurusan} atasan;

\textbf{:pelanggan:} d) memastikan penggalakan bagi fokus terhadap \textbf{pelanggan} dalam seluruh organisasi;

\textbf{:pengurusan:} e) memastikan integriti sistem \textbf{pengurusan} kualiti dikekalkan apabila perubahan sistem
\textbf{pengurusan} kualiti dirancang dan dilaksanakan.

\textbf{:pengurusan:} \#\#\#\#\# 6.1.1 Apabila merancang untuk sistem \textbf{pengurusan} kualiti, organisasi hendaklah
mempertimbangkan isu yang disebutkan dalam 4.1 dan keperluan yang disebutkan dalam 4.
dan menentukan risiko dan peluang yang perlu dinyatakan untuk:

\textbf{:pengurusan:} a) memberi jaminan bahawa sistem \textbf{pengurusan} kualiti boleh mencapai hasil yang
dimaksudkan;

\textbf{:penambahbaikan:} d) mencapai \textbf{penambahbaikan}.

\textbf{:pengurusan:} 1) mengintegrasikan, dan melaksanakan tindakan itu ke dalam proses sistem
\textbf{pengurusan} kualiti (lihat 4.4);

\textbf{:dasar:} NOTA 1. Pilihan untuk menyatakan risiko boleh termasuk mengelakkan risiko, mengambil risiko untuk
mengejar peluang, menghapuskan punca risiko, mengubah kemungkinan atau akibat, berkongsi risiko,
atau mengekalkan risiko ber\textbf{dasar}kan keputusan bermaklumat.

\textbf{:pelanggan:} NOTA 2. Peluang boleh membawa kepada penerimagunaan amalan baharu, pelancaran produk
baharu, pembukaan pasaran baharu, menangani \textbf{pelanggan} baharu, membina perkongsian,
menggunakan teknologi baharu, dan kemungkinan lain yang diingini dan berdaya maju bagi menangani
keperluan organisasi atau \textbf{pelanggan}nya.

\textbf{:objektif:} \#\#\#\# 6.2 \textbf{objektif} kualiti dan perancangan untuk mencapainya

\textbf{:objektif:} \#\#\#\#\# 6.2.1 Organisasi hendaklah mewujudkan \textbf{objektif} kualiti pada fungsi, aras dan proses yang relevan yang diperlukan untuk sistem pengurusan kualiti

\textbf{:pengurusan:} \#\#\#\#\# 6.2.1 Organisasi hendaklah mewujudkan \textbf{objektif} kualiti pada fungsi, aras dan proses yang relevan yang diperlukan untuk sistem \textbf{pengurusan} kualiti

\textbf{:objektif:} \textbf{objektif} kualiti hendaklah:

\textbf{:dasar:} a) tekal dengan \textbf{dasar} kualiti;

\textbf{:ukur:} b) boleh di\textbf{ukur};

\textbf{:pelanggan:} d) relevan dengan keakuran produk dan perkhidmatan, dan peningkatan kepuasan
\textbf{pelanggan};

\textbf{:komunikasi:} f) di\textbf{komunikasi}kan;

\textbf{:objektif:} Organisasi hendaklah menyelenggara maklumat didokumentasikan tentang \textbf{objektif} kualiti.

\textbf{:objektif:} \#\#\#\#\# 6.2.2 Apabila merancang cara untuk mencapai \textbf{objektif} kualiti, organisasi hendaklah menentukan:

\textbf{:tanggungjawab:} c) siapa yang ber\textbf{tanggungjawab};

\textbf{:pengurusan:} Apabila organisasi menentukan keperluan untuk mengubah sistem \textbf{pengurusan} kualiti,
perubahan itu hendaklah dilaksanakan dengan cara yang terancang (lihat 4.4).

\textbf{:pengurusan:} b) integriti sistem \textbf{pengurusan} kualiti;

\textbf{:tanggungjawab:} d) pengagihan atau pengagihan semula \textbf{tanggungjawab} dan bidang kuasa.

\textbf{:pengurusan:} Organisasi hendaklah menentukan dan menyediakan sumber yang diperlukan bagi
mewujudkan, melaksanakan, menyelenggarakan, dan menambah baik secara berterusan
sistem \textbf{pengurusan} kualiti.

\textbf{:kawalan:} Organisasi hendaklah menentukan dan menyediakan modal insan yang diperlukan untuk
pelaksanaan sistem pengurusan kualiti yang berkesan dan untuk operasi dan \textbf{kawalan}
prosesnya.

\textbf{:pengurusan:} Organisasi hendaklah menentukan dan menyediakan modal insan yang diperlukan untuk
pelaksanaan sistem \textbf{pengurusan} kualiti yang berkesan dan untuk operasi dan \textbf{kawalan}
prosesnya.

\textbf{:komunikasi:} d) teknologi maklumat dan \textbf{komunikasi}.

\textbf{:ukur:} \#\#\#\#\# 7.1.5 Sumber pemantauan dan peng\textbf{ukur}an

\textbf{:ukur:} Organisasi hendaklah menentukan dan menyediakan sumber yang diperlukan untuk
memastikan keputusan yang sah dan boleh dipercayai, apabila pemantauan atau peng\textbf{ukur}an
digunakan untuk menentu sah keakuran produk dan perkhidmatan terhadap keperluan.

\textbf{:ukur:} a) adalah sesuai untuk jenis aktiviti pemantauan dan peng\textbf{ukur}an tertentu yang
dilaksanakan;

\textbf{:ukur:} Organisasi hendaklah menyimpan maklumat didokumentasikan yang sesuai sebagai bukti
kesesuaian untuk maksud tentang sumber pemantauan dan peng\textbf{ukur}an.

\textbf{:ukur:} \#\#\#\#\#\# 7.1.5.2 Kebolehkesanan peng\textbf{ukur}an

\textbf{:ukur:} Apabila kebolehkesanan peng\textbf{ukur}an ialah suatu keperluan, atau dianggap suatu perkara
yang penting oleh organisasi untuk memberikan keyakinan dalam kesahan hasil peng\textbf{ukur}an,
peralatan meng\textbf{ukur} hendaklah:

\textbf{:ukur:} a) ditent\textbf{ukur} atau ditentusahkan, atau kedua-duanya, pada sela waktu yang ditetapkan,
atau sebelum digunakan, dengan standard peng\textbf{ukur}an yang boleh dikesan daripada
standard peng\textbf{ukur}an antarabangsa atau kebangsaan; apabila standard sedemikian tidak
wujud, asas yang digunakan bagi tent\textbf{ukur}an atau penentusahan hendaklah disimpan
sebagai maklumat didokumentasikan;

\textbf{:ukur:} c) dilindungi daripada pelarasan, kerosakan atau kemerosotan yang akan mentaksahkan
status tent\textbf{ukur}an dan hasil peng\textbf{ukur}an seterusnya.

\textbf{:ukur:} Organisasi hendaklah menentukan sama ada kesahan hasil peng\textbf{ukur}an terdahulu terjejas
teruk apabila peralatan meng\textbf{ukur} didapati tidak sesuai untuk maksudnya yang ditetapkan,
dan hendaklah mengambil tindakan sewajarnya seperti yang diperlukan.

\textbf{:objektif:} NOTA 1. Pengetahuan organisasi ialah pengetahuan khusus bagi organisasi itu; ia diperoleh melalui
pengalaman. Ia merupakan maklumat yang digunakan dan dikongsi bagi mencapai \textbf{objektif} organisasi.

\textbf{:dasar:} NOTA 2. Pengetahuan organisasi boleh ber\textbf{dasar}kan:

\textbf{:penambahbaikan:} a) sumber dalaman (contoh, harta intelek; pengetahuan yang diperoleh melalui pengalaman;
pengajaran daripada kegagalan dan projek yang berjaya; perakaman dan perkongsian
pengetahuan dan pengalaman yang tidak didokumentasikan; hasil \textbf{penambahbaikan} dalam proses,
produk dan perkhidmatan);

\textbf{:pelanggan:} b) sumber luaran (contoh, standard; akademia; persidangan; pengumpulan pengetahuan daripada
\textbf{pelanggan} atau penyedia luar).

\textbf{:kompeten:} \#\#\#\# 7.2 Ke\textbf{kompeten}an

\textbf{:kompeten:} a) menentukan ke\textbf{kompeten}an yang diperlukan oleh orang yang melakukan kerja di bawah
kawalannya memberi kesan kepada prestasi dan keberkesanan sistem pengurusan
kualiti;

\textbf{:kawalan:} a) menentukan ke\textbf{kompeten}an yang diperlukan oleh orang yang melakukan kerja di bawah
\textbf{kawalan}nya memberi kesan kepada prestasi dan keberkesanan sistem pengurusan
kualiti;

\textbf{:pengurusan:} a) menentukan ke\textbf{kompeten}an yang diperlukan oleh orang yang melakukan kerja di bawah
\textbf{kawalan}nya memberi kesan kepada prestasi dan keberkesanan sistem \textbf{pengurusan}
kualiti;

\textbf{:prestasi:} a) menentukan ke\textbf{kompeten}an yang diperlukan oleh orang yang melakukan kerja di bawah
\textbf{kawalan}nya memberi kesan kepada \textbf{prestasi} dan keberkesanan sistem \textbf{pengurusan}
kualiti;

\textbf{:dasar:} b) memastikan bahawa orang tersebut kompeten ber\textbf{dasar}kan pendidikan, latihan, atau
pengalaman yang sesuai;

\textbf{:kompeten:} b) memastikan bahawa orang tersebut \textbf{kompeten} ber\textbf{dasar}kan pendidikan, latihan, atau
pengalaman yang sesuai;

\textbf{:kompeten:} c) mengambil tindakan, jika berkenaan, untuk memperoleh ke\textbf{kompeten}an yang diperlukan,
dan menilai keberkesanan tindakan yang diambil;

\textbf{:kompeten:} d) menyimpan maklumat didokumentasikan yang sesuai sebagai bukti ke\textbf{kompeten}an.

\textbf{:kompeten:} NOTA. Tindakan yang terpakai boleh termasuk, sebagai contoh, penyediaan latihan, pementoran, atau
penugasan semula pekerja sedia ada; atau pengambilan atau pengkontrakan orang yang \textbf{kompeten}.

\textbf{:kawalan:} Organisasi hendaklah memastikan bahawa orang yang melakukan kerja di bawah
\textbf{kawalan}nya mengetahui tentang:

\textbf{:dasar:} a) \textbf{dasar} kualiti;

\textbf{:objektif:} b) \textbf{objektif} kualiti yang relevan;

\textbf{:pengurusan:} c) sumbangan mereka kepada keberkesanan sistem \textbf{pengurusan} kualiti, termasuk manfaat
prestasi yang ditambah baik;

\textbf{:prestasi:} c) sumbangan mereka kepada keberkesanan sistem \textbf{pengurusan} kualiti, termasuk manfaat
\textbf{prestasi} yang ditambah baik;

\textbf{:pengurusan:} d) implikasi jika tidak mengakuri keperluan sistem \textbf{pengurusan} kualiti.

\textbf{:komunikasi:} \#\#\#\# 7.4 \textbf{komunikasi}

\textbf{:pengurusan:} Organisasi hendaklah menentukan komunikasi dalaman dan luaran yang relevan dengan
sistem \textbf{pengurusan} kualiti, termasuk:

\textbf{:komunikasi:} Organisasi hendaklah menentukan \textbf{komunikasi} dalaman dan luaran yang relevan dengan
sistem \textbf{pengurusan} kualiti, termasuk:

\textbf{:komunikasi:} a) perkara yang akan di\textbf{komunikasi}kan;

\textbf{:komunikasi:} b) bila perlu ber\textbf{komunikasi};

\textbf{:komunikasi:} c) dengan siapa untuk ber\textbf{komunikasi};

\textbf{:komunikasi:} d) cara untuk ber\textbf{komunikasi};

\textbf{:komunikasi:} e) siapa yang ber\textbf{komunikasi}.

\textbf{:pengurusan:} Sistem \textbf{pengurusan} kualiti organisasi hendaklah termasuk:

\textbf{:pengurusan:} b) maklumat didokumentasikan yang ditentukan perlu oleh organisasi bagi keberkesanan
sistem \textbf{pengurusan} kualiti.

\textbf{:pengurusan:} NOTA. Tahap maklumat didokumentasikan untuk sistem \textbf{pengurusan} kualiti boleh berbeza antara satu
organisasi dengan yang lain disebabkan oleh:

\textbf{:kompeten:} - saiz organisasi dan jenis aktiviti, proses, produk dan perkhidmatannya;
- kerumitan proses dan saling tindaknya;
- ke\textbf{kompeten}an orang.

\textbf{:kawalan:} \#\#\#\#\# 7.5.3 \textbf{kawalan} maklumat didokumentasikan

\textbf{:pengurusan:} \#\#\#\#\#\# 7.5.3.1 Maklumat didokumentasikan yang diperlukan oleh sistem \textbf{pengurusan} kualiti dan
Standard Antarabangsa ini hendaklah dikawal bagi memastikan:

\textbf{:kawalan:} c) \textbf{kawalan} perubahan (contoh, \textbf{kawalan} versi);

\textbf{:pengurusan:} Maklumat didokumentasikan yang berasal dari luar yang ditentukan penting oleh organisasi
bagi perancangan dan operasi sistem \textbf{pengurusan} kualiti hendaklah dikenal pasti sewajarnya,
dan dikawal.

\textbf{:kawalan:} \#\#\#\# 8.1 Perancangan dan \textbf{kawalan} operasi

\textbf{:kawalan:} d) melaksanakan \textbf{kawalan} proses selaras dengan kriteria;

\textbf{:komunikasi:} \#\#\#\#\# 8.2.1 \textbf{komunikasi} dengan pelanggan

\textbf{:pelanggan:} \#\#\#\#\# 8.2.1 \textbf{komunikasi} dengan \textbf{pelanggan}

\textbf{:komunikasi:} \textbf{komunikasi} dengan pelanggan hendaklah termasuk:

\textbf{:pelanggan:} \textbf{komunikasi} dengan \textbf{pelanggan} hendaklah termasuk:

\textbf{:pelanggan:} c) mendapatkan maklum balas \textbf{pelanggan} berkaitan dengan produk dan perkhidmatan,
termasuk aduan \textbf{pelanggan};

\textbf{:pelanggan:} d) mengendalikan atau mengawal harta \textbf{pelanggan};

\textbf{:pelanggan:} Apabila menentukan keperluan untuk produk dan perkhidmatan yang akan ditawarkan
kepada \textbf{pelanggan}, organisasi hendaklah memastikan bahawa:

\textbf{:komitmen:} \#\#\#\#\#\# 8.2.3.1 Organisasi hendaklah memastikan bahawa ia mempunyai keupayaan untuk
memenuhi keperluan produk dan perkhidmatan yang ditawarkan kepada pelanggan.
Organisasi hendaklah menjalankan kajian semula sebelum memberikan \textbf{komitmen} untuk
membekalkan produk dan perkhidmatan kepada pelanggan, termasuk:

\textbf{:pelanggan:} \#\#\#\#\#\# 8.2.3.1 Organisasi hendaklah memastikan bahawa ia mempunyai keupayaan untuk
memenuhi keperluan produk dan perkhidmatan yang ditawarkan kepada \textbf{pelanggan}.
Organisasi hendaklah menjalankan kajian semula sebelum memberikan \textbf{komitmen} untuk
membekalkan produk dan perkhidmatan kepada \textbf{pelanggan}, termasuk:

\textbf{:pelanggan:} a) keperluan yang ditetapkan oleh \textbf{pelanggan}, termasuk keperluan hantar serah dan aktiviti
selepas hantar serah;

\textbf{:pelanggan:} b) keperluan yang tidak dinyatakan oleh \textbf{pelanggan}, tetapi adalah perlu bagi kegunaan yang
ditetapkan atau yang dimaksudkan, jika hal itu diketahui;

\textbf{:pelanggan:} Keperluan \textbf{pelanggan} hendaklah disahkan oleh organisasi sebelum penerimaan, apabila
\textbf{pelanggan} tidak menyediakan pernyataan didokumentasikan tentang keperluan mereka.

\textbf{:kawalan:} Dalam menentukan tahap dan \textbf{kawalan} untuk reka bentuk dan pembangunan, organisasi
hendaklah mengambil kira:

\textbf{:tanggungjawab:} d) \textbf{tanggungjawab} dan kuasa yang terlibat dalam proses reka bentuk dan pembangunan;

\textbf{:pelanggan:} g) keperluan untuk pelibatan \textbf{pelanggan} dan pengguna dalam proses reka bentuk dan
pembangunan;

\textbf{:kawalan:} i) tahap \textbf{kawalan} yang dijangkakan untuk proses reka bentuk dan pembangunan oleh
pelanggan dan pihak berkepentingan lain yang relevan;

\textbf{:pelanggan:} i) tahap \textbf{kawalan} yang dijangkakan untuk proses reka bentuk dan pembangunan oleh
\textbf{pelanggan} dan pihak berkepentingan lain yang relevan;

\textbf{:prestasi:} a) keperluan fungsian dan \textbf{prestasi};

\textbf{:komitmen:} d) standard atau kod amalan yang organisasi telah memberikan \textbf{komitmen} untuk
dilaksanakan;

\textbf{:kawalan:} \#\#\#\#\# 8.3.4 \textbf{kawalan} reka bentuk dan pembangunan

\textbf{:kawalan:} Organisasi hendaklah mengenakan \textbf{kawalan} untuk proses reka bentuk dan pembangunan
bagi memastikan bahawa:

\textbf{:ukur:} c) merangkumi atau merujuk keperluan pemantauan dan peng\textbf{ukur}an, sebagaimana yang
sesuai, dan kriteria penerimaan;

\textbf{:kawalan:} \#\#\#\# 8.4 \textbf{kawalan} terhadap proses, produk dan perkhidmatan sediaan luar

\textbf{:kawalan:} Organisasi hendaklah menentukan \textbf{kawalan} yang akan diguna pakai bagi proses, produk dan
perkhidmatan sediaan luar apabila:

\textbf{:pelanggan:} b) produk dan perkhidmatan disediakan secara terus kepada \textbf{pelanggan} oleh penyedia luar
bagi pihak organisasi;

\textbf{:dasar:} c) proses, atau sebahagian daripada proses, disediakan oleh penyedia luar ber\textbf{dasar}kan
keputusan yang dibuat oleh organisasi.

\textbf{:dasar:} Organisasi hendaklah menentukan dan mengguna pakai kriteria penilaian, pemilihan,
pemantauan prestasi, dan penilaian semula penyedia luar, ber\textbf{dasar}kan keupayaan mereka
untuk menyediakan proses atau produk dan perkhidmatan selaras dengan keperluan.
Organisasi hendaklah mengekalkan maklumat didokumentasikan berkaitan aktiviti ini dan
apa-apa tindakan perlu yang berpunca daripada penilaian.

\textbf{:prestasi:} Organisasi hendaklah menentukan dan mengguna pakai kriteria penilaian, pemilihan,
pemantauan \textbf{prestasi}, dan penilaian semula penyedia luar, ber\textbf{dasar}kan keupayaan mereka
untuk menyediakan proses atau produk dan perkhidmatan selaras dengan keperluan.
Organisasi hendaklah mengekalkan maklumat didokumentasikan berkaitan aktiviti ini dan
apa-apa tindakan perlu yang berpunca daripada penilaian.

\textbf{:kawalan:} \#\#\#\#\# 8.4.2 Jenis dan takat \textbf{kawalan}

\textbf{:pelanggan:} Organisasi hendaklah memastikan bahawa proses, produk dan perkhidmatan sediaan luar
tidak memberi kesan bertentangan terhadap keupayaan organisasi untuk membekalkan
secara tekal produk dan perkhidmatan yang akur dengan keperluan kepada \textbf{pelanggan}nya.

\textbf{:kawalan:} a) memastikan bahawa proses sediaan luar kekal dalam \textbf{kawalan} sistem pengurusan
kualitinya;

\textbf{:pengurusan:} a) memastikan bahawa proses sediaan luar kekal dalam \textbf{kawalan} sistem \textbf{pengurusan}
kualitinya;

\textbf{:kawalan:} b) menetapkan \textbf{kawalan} yang dimaksudkan untuk diguna pakai terhadap kedua-dua
penyedia luar dan output yang terhasil;

\textbf{:pelanggan:} 1) impak yang mungkin terhasil daripada proses, produk dan perkhidmatan sediaan luar
terhadap keupayaan organisasi untuk memenuhi secara tekal keperluan \textbf{pelanggan},
serta keperluan berkanun dan peraturan yang terpakai;

\textbf{:kawalan:} 2) keberkesanan \textbf{kawalan} yang diguna pakai oleh penyedia luar;

\textbf{:komunikasi:} Organisasi hendaklah memastikan kecukupan keperluan sebelum di\textbf{komunikasi}kan kepada
penyedia luar.

\textbf{:komunikasi:} Organisasi hendaklah meng\textbf{komunikasi}kan kepada penyedia luar keperluannya tentang:

\textbf{:kompeten:} c) ke\textbf{kompeten}an, termasuk apa-apa kelayakan yang diperlukan oleh seseorang;

\textbf{:kawalan:} e) \textbf{kawalan} dan pemantauan prestasi penyedia luar yang akan diguna pakai oleh organisasi;

\textbf{:prestasi:} e) \textbf{kawalan} dan pemantauan \textbf{prestasi} penyedia luar yang akan diguna pakai oleh organisasi;

\textbf{:pelanggan:} f) aktiviti penentusahan atau pengesahan yang organisasi, atau \textbf{pelanggan}nya, bermaksud
untuk melaksanakan di premis penyedia luar.

\textbf{:kawalan:} \#\#\#\#\# 8.5.1 \textbf{kawalan} penyediaan pengeluaran dan perkhidmatan

\textbf{:ukur:} b) ketersediaan dan penggunaan sumber pemantauan dan peng\textbf{ukur}an yang sesuai;

\textbf{:kawalan:} c) pelaksanaan aktiviti pemantauan dan pengukuran pada tahap yang sesuai untuk
menentusahkan bahawa kriteria \textbf{kawalan} proses atau output, dan kriteria penerimaan
produk dan perkhidmatan telah dipenuhi;

\textbf{:ukur:} c) pelaksanaan aktiviti pemantauan dan peng\textbf{ukur}an pada tahap yang sesuai untuk
menentusahkan bahawa kriteria \textbf{kawalan} proses atau output, dan kriteria penerimaan
produk dan perkhidmatan telah dipenuhi;

\textbf{:kompeten:} e) pelantikan orang yang \textbf{kompeten}, termasuk apa-apa kelayakan yang diperlukan;

\textbf{:ukur:} f) pengesahan dan pengesahan semula secara berkala, terhadap keupayaan untuk
mencapai hasil yang dirancang daripada proses bagi penyediaan pengeluaran dan
perkhidmatan, jika output yang dihasilkan tidak boleh ditentusahkan melalui pemantauan
atau peng\textbf{ukur}an berikutnya;

\textbf{:ukur:} Organisasi hendaklah mengenal pasti status output berkenaan dengan keperluan
pemantauan dan peng\textbf{ukur}an sepanjang penyediaan pengeluaran dan perkhidmatan.

\textbf{:pelanggan:} \#\#\#\#\# 8.5.3 Harta kepunyaan \textbf{pelanggan} atau penyedia luar

\textbf{:kawalan:} Organisasi hendaklah memelihara harta kepunyaan pelanggan atau penyedia luar semasa
harta itu di bawah \textbf{kawalan} organisasi atau digunakan oleh organisasi itu.

\textbf{:pelanggan:} Organisasi hendaklah memelihara harta kepunyaan \textbf{pelanggan} atau penyedia luar semasa
harta itu di bawah \textbf{kawalan} organisasi atau digunakan oleh organisasi itu.

\textbf{:pelanggan:} Organisasi hendaklah mengenal pasti, menentusahkan, menjaga serta melindungi harta
\textbf{pelanggan} atau penyedia luar yang disediakan untuk kegunaan atau untuk dijadikan
sebahagian daripada produk dan perkhidmatan.

\textbf{:pelanggan:} Apabila harta \textbf{pelanggan} atau penyedia luar hilang, rosak atau selainnya didapati tidak sesuai
untuk digunakan, organisasi hendaklah melaporkan hal itu kepada \textbf{pelanggan} atau penyedia
luar, dan menyimpan maklumat didokumentasikan tentang hal yang telah berlaku.

\textbf{:pelanggan:} NOTA. Harta \textbf{pelanggan} atau penyedia luar boleh termasuk bahan, komponen, alat dan peralatan,
premis, harta intelek dan data peribadi.

\textbf{:kawalan:} NOTA. Pemeliharaan boleh termasuk pengenalpastian, pengendalian, \textbf{kawalan} pencemaran,
pembungkusan, penyimpanan, penghantaran atau pengangkutan, dan penjagaan.

\textbf{:pelanggan:} d) keperluan \textbf{pelanggan};

\textbf{:pelanggan:} e) maklum balas \textbf{pelanggan}.

\textbf{:kawalan:} \#\#\#\#\# 8.5.6 \textbf{kawalan} perubahan

\textbf{:pelanggan:} Pelepasan produk dan perkhidmatan kepada \textbf{pelanggan} tidak perlu diteruskan sehingga
perkiraan terancang disiapkan dengan memuaskan, melainkan jika diluluskan sebaliknya oleh
pihak berkuasa yang relevan dan, jika berkenaan, oleh \textbf{pelanggan}.

\textbf{:kawalan:} \#\#\#\# 8.7 \textbf{kawalan} output tak akur

\textbf{:dasar:} Organisasi hendaklah mengambil tindakan yang sesuai ber\textbf{dasar}kan keadaan ketakakuran
dan kesannya terhadap keakuran produk dan perkhidmatan. Perkara ini hendaklah juga
terpakai bagi produk dan perkhidmatan tak akur yang dikesan selepas hantar serah produk,
semasa atau selepas penyediaan perkhidmatan.

\textbf{:ketakakuran:} Organisasi hendaklah mengambil tindakan yang sesuai ber\textbf{dasar}kan keadaan \textbf{ketakakuran}
dan kesannya terhadap keakuran produk dan perkhidmatan. Perkara ini hendaklah juga
terpakai bagi produk dan perkhidmatan tak akur yang dikesan selepas hantar serah produk,
semasa atau selepas penyediaan perkhidmatan.

\textbf{:pelanggan:} c) memaklumkan \textbf{pelanggan};

\textbf{:ketakakuran:} a) memerihalkan \textbf{ketakakuran};

\textbf{:ketakakuran:} d) mengenal pasti kuasa yang memutuskan tindakan berkenaan dengan \textbf{ketakakuran}.

\textbf{:prestasi:} \#\#\# 9 Penilaian \textbf{prestasi}

\textbf{:ukur:} \#\#\#\# 9.1 Pemantauan, peng\textbf{ukur}an, analisis dan penilaian

\textbf{:ukur:} a) apa yang perlu dipantau dan di\textbf{ukur};

\textbf{:ukur:} b) kaedah pemantauan, peng\textbf{ukur}an, analisis dan penilaian yang diperlukan bagi
memastikan hasil yang sah;

\textbf{:ukur:} c) bila pemantauan dan peng\textbf{ukur}an hendak dilaksanakan;

\textbf{:ukur:} d) bila hasil pemantauan dan peng\textbf{ukur}an hendak dianalisis dan dinilai.

\textbf{:pengurusan:} Organisasi hendaklah menilai prestasi dan keberkesanan sistem \textbf{pengurusan} kualiti.

\textbf{:prestasi:} Organisasi hendaklah menilai \textbf{prestasi} dan keberkesanan sistem \textbf{pengurusan} kualiti.

\textbf{:pelanggan:} \#\#\#\#\# 9.1.2 Kepuasan \textbf{pelanggan}

\textbf{:pelanggan:} Organisasi hendaklah memantau tanggapan \textbf{pelanggan} tentang tahap keperluan dan
jangkaan mereka yang telah dipenuhi. Organisasi hendaklah menentukan kaedah untuk
memperoleh, memantau dan menyemak semula maklumat ini.

\textbf{:pelanggan:} NOTA. Contoh pemantauan tanggapan \textbf{pelanggan} boleh termasuk kaji selidik \textbf{pelanggan}, maklum
balas \textbf{pelanggan} mengenai produk dan perkhidmatan yang dihantar serah, mesyuarat dengan
\textbf{pelanggan}, analisis bahagian pasaran, pujian, tuntutan waranti dan laporan wakil penjual.

\textbf{:ukur:} Organisasi hendaklah menganalisis dan menilai data dan maklumat yang sesuai hasil
daripada pemantauan dan peng\textbf{ukur}an.

\textbf{:pelanggan:} b) tahap kepuasan \textbf{pelanggan};

\textbf{:pengurusan:} c) prestasi dan keberkesanan sistem \textbf{pengurusan} kualiti;

\textbf{:prestasi:} c) \textbf{prestasi} dan keberkesanan sistem \textbf{pengurusan} kualiti;

\textbf{:prestasi:} f) \textbf{prestasi} penyedia luar;

\textbf{:pengurusan:} g) keperluan untuk menambah baik sistem \textbf{pengurusan} kualiti.

\textbf{:pengurusan:} \#\#\#\#\# 9.2.1 Organisasi hendaklah menjalankan audit dalaman secara berkala bagi menyediakan
maklumat bahawa sistem \textbf{pengurusan} kualiti sama ada atau tidak:

\textbf{:pengurusan:} 1) keperluan sistem \textbf{pengurusan} kualiti organisasi itu sendiri;

\textbf{:tanggungjawab:} a) merancang, mewujudkan, melaksanakan dan menyelenggarakan program audit,
termasuk kekerapan, kaedah, \textbf{tanggungjawab}, keperluan perancangan dan pelaporan,
yang hendaklah mengambil kira kepentingan proses berkenaan, perubahan yang
memberi kesan kepada organisasi, dan keputusan audit terdahulu;

\textbf{:objektif:} c) memilih juruaudit dan menjalankan audit bagi memastikan ke\textbf{objektif}an dan
kesaksamaan proses audit;

\textbf{:pengurusan:} d) memastikan bahawa keputusan audit dilaporkan kepada pihak \textbf{pengurusan} yang relevan;

\textbf{:pengurusan:} \#\#\#\# 9.3 Kajian semula \textbf{pengurusan}

\textbf{:pengurusan:} \textbf{pengurusan} atasan hendaklah mengkaji semula sistem \textbf{pengurusan} kualiti organisasi, secara
berkala, bagi memastikan kesesuaian, kecukupan, keberkesanan dan keselarasan yang
berterusan dengan hala tuju strategik organisasi.

\textbf{:pengurusan:} \#\#\#\#\# 9.3.2 Input kajian semula \textbf{pengurusan}

\textbf{:pengurusan:} Kajian semula \textbf{pengurusan} hendaklah dirancang dan dijalankan dengan mengambil kira:

\textbf{:pengurusan:} a) status tindakan daripada kajian semula \textbf{pengurusan} yang terdahulu;

\textbf{:pengurusan:} b) perubahan dalam isu-isu luaran dan dalaman yang relevan dengan sistem \textbf{pengurusan}
kualiti;

\textbf{:pengurusan:} c) maklumat tentang prestasi dan keberkesanan sistem \textbf{pengurusan} kualiti, termasuk trend
dalam:

\textbf{:prestasi:} c) maklumat tentang \textbf{prestasi} dan keberkesanan sistem \textbf{pengurusan} kualiti, termasuk trend
dalam:

\textbf{:pelanggan:} 1) kepuasan \textbf{pelanggan} dan maklum balas daripada pihak berkepentingan yang
relevan;

\textbf{:objektif:} 2) takat pencapaian \textbf{objektif} kualiti;

\textbf{:prestasi:} 3) \textbf{prestasi} proses dan keakuran produk dan perkhidmatan;

\textbf{:ketakakuran:} 4) \textbf{ketakakuran} dan tindakan pembetulan;

\textbf{:ukur:} 5) hasil pemantauan dan peng\textbf{ukur}an;

\textbf{:prestasi:} 7) \textbf{prestasi} penyedia luar;

\textbf{:penambahbaikan:} f) peluang untuk \textbf{penambahbaikan}.

\textbf{:pengurusan:} \#\#\#\#\# 9.3.3 Output kajian semula \textbf{pengurusan}

\textbf{:pengurusan:} Output kajian semula \textbf{pengurusan} hendaklah termasuk keputusan dan tindakan yang
berkaitan dengan:

\textbf{:penambahbaikan:} a) peluang untuk \textbf{penambahbaikan};

\textbf{:pengurusan:} b) apa-apa keperluan untuk perubahan kepada sistem \textbf{pengurusan} kualiti;

\textbf{:pengurusan:} Organisasi hendaklah mengekalkan maklumat didokumentasikan sebagai bukti hasil kajian
semula \textbf{pengurusan}.

\textbf{:penambahbaikan:} \#\#\# 10 \textbf{penambahbaikan}

\textbf{:pelanggan:} Organisasi hendaklah menentukan dan memilih peluang untuk penambahbaikan, dan
melaksanakan apa-apa tindakan yang perlu, bagi memenuhi keperluan \textbf{pelanggan} dan
meningkatkan kepuasan \textbf{pelanggan}.

\textbf{:penambahbaikan:} Organisasi hendaklah menentukan dan memilih peluang untuk \textbf{penambahbaikan}, dan
melaksanakan apa-apa tindakan yang perlu, bagi memenuhi keperluan \textbf{pelanggan} dan
meningkatkan kepuasan \textbf{pelanggan}.

\textbf{:pengurusan:} c) menambah baik prestasi dan keberkesanan sistem \textbf{pengurusan} kualiti.

\textbf{:prestasi:} c) menambah baik \textbf{prestasi} dan keberkesanan sistem \textbf{pengurusan} kualiti.

\textbf{:penambahbaikan:} NOTA. Contoh \textbf{penambahbaikan} boleh termasuk pembetulan, tindakan pembetulan, \textbf{penambahbaikan}
berterusan, perubahan kejayaan besar, inovasi dan penyusunan semula.

\textbf{:ketakakuran:} \#\#\#\# 10.2 \textbf{ketakakuran} dan tindakan pembetulan

\textbf{:ketakakuran:} \#\#\#\#\# 10.2.1 Apabila \textbf{ketakakuran} berlaku, termasuk apa-apa yang timbul daripada aduan,
organisasi hendaklah:

\textbf{:ketakakuran:} a) bertindak balas terhadap \textbf{ketakakuran} itu dan, jika berkenaan:

\textbf{:ketakakuran:} b) menilai keperluan untuk mengambil tindakan menghapuskan penyebab \textbf{ketakakuran},
supaya tidak berulang atau berlaku di tempat lain, dengan cara:

\textbf{:ketakakuran:} 1) menyemak semula dan menganalisis \textbf{ketakakuran};

\textbf{:ketakakuran:} 2) menentukan penyebab \textbf{ketakakuran};

\textbf{:ketakakuran:} 3) menentukan jika \textbf{ketakakuran} serupa wujud, atau mungkin boleh berlaku;

\textbf{:pengurusan:} f) membuat perubahan dalam sistem \textbf{pengurusan} kualiti, jika perlu.

\textbf{:ketakakuran:} Tindakan pembetulan hendaklah bersesuaian dengan kesan \textbf{ketakakuran} yang dihadapi.

\textbf{:ketakakuran:} a) keadaan \textbf{ketakakuran} dan apa-apa tindakan susulan yang diambil;

\textbf{:penambahbaikan:} \#\#\#\# 10.3 \textbf{penambahbaikan} berterusan

\textbf{:pengurusan:} Organisasi hendaklah secara berterusan menambah baik kesesuaian, kecukupan dan
keberkesanan sistem \textbf{pengurusan} kualiti.

\textbf{:pengurusan:} Organisasi hendaklah mengambil kira keputusan analisis dan penilaian, dan output daripada
kajian semula \textbf{pengurusan}, bagi menentukan jika terdapat keperluan atau peluang yang
sepatutnya ditangani sebagai sebahagian daripada penambahbaikan berterusan.

\textbf{:penambahbaikan:} Organisasi hendaklah mengambil kira keputusan analisis dan penilaian, dan output daripada
kajian semula \textbf{pengurusan}, bagi menentukan jika terdapat keperluan atau peluang yang
sepatutnya ditangani sebagai sebahagian daripada \textbf{penambahbaikan} berterusan.

\textbf{:pengurusan:} Struktur klausa (iaitu urutan klausa) dan beberapa istilah edisi Standard Antarabangsa ini,
berbanding dengan edisi terdahulu (ISO 9001:2008), telah diubah untuk menambah baik
sejajar dengan standard sistem \textbf{pengurusan} yang lain.

\textbf{:pengurusan:} Dalam Standard Antarabangsa ini, tidak ada keperluan supaya struktur dan istilahnya diguna
pakai terhadap maklumat didokumentasikan dalam sistem \textbf{pengurusan} kualiti organisasi.

\textbf{:dasar:} Struktur klausa bermaksud menyediakan pembentangan keperluan yang koheren, bukannya
model untuk mendokumentasikan \textbf{dasar}, matlamat dan proses sesuatu organisasi. Struktur
dan kandungan maklumat didokumentasikan yang berkaitan dengan sistem pengurusan
kualiti sering boleh menjadi lebih relevan kepada penggunanya jika ia berkaitan dengan
proses yang dikendalikan oleh organisasi dan maklumat yang diselenggarakan untuk maksud
lain.

\textbf{:pengurusan:} Struktur klausa bermaksud menyediakan pembentangan keperluan yang koheren, bukannya
model untuk mendokumentasikan \textbf{dasar}, matlamat dan proses sesuatu organisasi. Struktur
dan kandungan maklumat didokumentasikan yang berkaitan dengan sistem \textbf{pengurusan}
kualiti sering boleh menjadi lebih relevan kepada penggunanya jika ia berkaitan dengan
proses yang dikendalikan oleh organisasi dan maklumat yang diselenggarakan untuk maksud
lain.

\textbf{:rekod:} Istilah yang digunakan oleh organisasi tidak perlu digantikan dengan istilah yang digunakan
dalam Standard Antarabangsa ini bagi menetapkan keperluan sistem pengurusan kualiti.
Organisasi boleh memilih untuk menggunakan istilah yang sesuai dengan operasinya
(contohnya, menggunakan "\textbf{rekod}", "pendokumenan" atau "protokol" dan bukannya
"maklumat didokumentasikan"; atau "pembekal", "rakan kongsi" atau "penjual" dan bukannya
"penyedia luar"). Jadual A.1 menunjukkan perbezaan utama dalam istilah antara edisi
Standard Antarabangsa ini dengan edisi terdahulu.

\textbf{:pengurusan:} Istilah yang digunakan oleh organisasi tidak perlu digantikan dengan istilah yang digunakan
dalam Standard Antarabangsa ini bagi menetapkan keperluan sistem \textbf{pengurusan} kualiti.
Organisasi boleh memilih untuk menggunakan istilah yang sesuai dengan operasinya
(contohnya, menggunakan "\textbf{rekod}", "pendokumenan" atau "protokol" dan bukannya
"maklumat didokumentasikan"; atau "pembekal", "rakan kongsi" atau "penjual" dan bukannya
"penyedia luar"). Jadual A.1 menunjukkan perbezaan utama dalam istilah antara edisi
Standard Antarabangsa ini dengan edisi terdahulu.

\textbf{:pelanggan:} Kemasukan khusus istilah "perkhidmatan" bermaksud untuk menunjukkan ada perbezaan
antara produk dengan perkhidmatan dalam pemakaian sesetengah keperluan. Ciri-ciri
perkhidmatan ialah sekurang-kurangnya sebahagian daripada output dihasilkan pada antara
muka dengan \textbf{pelanggan}. Ini bermakna, sebagai contoh, bahawa keakuran terhadap
keperluan tidak semestinya disahkan sebelum hantar serah perkhidmatan.

\textbf{:pelanggan:} Dalam kebanyakan hal, produk dan perkhidmatan digunakan bersama-sama. Kebanyakan
output yang disediakan organisasi kepada \textbf{pelanggan}, atau dibekalkan kepada mereka oleh
pembekal luar, termasuk kedua-dua produk dan perkhidmatan. Sebagai contoh, produk
ketara atau tak ketara boleh mempunyai beberapa perkhidmatan yang berkaitan atau sesuatu
perkhidmatan boleh mempunyai beberapa produk ketara atau tak ketara yang berkaitan.

\textbf{:pengurusan:} Subklausa 4.2 menetapkan keperluan untuk organisasi menentukan pihak berkepentingan
yang relevan dengan sistem \textbf{pengurusan} kualiti dan keperluan pihak berkepentingan tersebut.
Walau bagaimanapun, 4.2 tidak membayangkan tambahan keperluan sistem \textbf{pengurusan}
kualiti di luar skop Standard Antarabangsa ini. Seperti yang dinyatakan dalam skop, Standard
Antarabangsa ini diguna pakai jika organisasi perlu menunjukkan keupayaannya
menyediakan secara tekal produk dan perkhidmatan yang memenuhi keperluan pelanggan
serta keperluan berkanun dan peraturan yang diguna pakai, dan bertujuan untuk
meningkatkan kepuasan pelanggan.

\textbf{:pelanggan:} Subklausa 4.2 menetapkan keperluan untuk organisasi menentukan pihak berkepentingan
yang relevan dengan sistem \textbf{pengurusan} kualiti dan keperluan pihak berkepentingan tersebut.
Walau bagaimanapun, 4.2 tidak membayangkan tambahan keperluan sistem \textbf{pengurusan}
kualiti di luar skop Standard Antarabangsa ini. Seperti yang dinyatakan dalam skop, Standard
Antarabangsa ini diguna pakai jika organisasi perlu menunjukkan keupayaannya
menyediakan secara tekal produk dan perkhidmatan yang memenuhi keperluan \textbf{pelanggan}
serta keperluan berkanun dan peraturan yang diguna pakai, dan bertujuan untuk
meningkatkan kepuasan \textbf{pelanggan}.

\textbf{:pengurusan:} Dalam Standard Antarabangsa ini, tidak ada keperluan supaya organisasi mengambil kira
pihak berkepentingan jika telah diputuskan bahawa pihak berkepentingan tersebut tidak
relevan dengan sistem \textbf{pengurusan} kualitinya. Terpulang kepada organisasi untuk
memutuskan jika keperluan tertentu pihak berkepentingan yang relevan adalah berkaitan
dengan sistem \textbf{pengurusan} kualitinya.

\textbf{:pengurusan:} Konsep pemikiran berasaskan risiko tersirat dalam edisi terdahulu Standard Antarabangsa
ini, contohnya melalui keperluan untuk merancang, menyemak semula dan menambah baik.
Standard Antarabangsa ini menetapkan keperluan bagi organisasi memahami konteksnya
(lihat 4.1) dan menentukan risiko sebagai asas perancangan (lihat 6.1). Hal ini
menggambarkan penggunaan pemikiran berasaskan risiko untuk merancang dan
melaksanakan proses sistem \textbf{pengurusan} kualiti (lihat 4.4) dan membantu dalam menentukan
tahap maklumat didokumentasikan.

\textbf{:pengurusan:} Satu daripada matlamat utama sistem \textbf{pengurusan} kualiti adalah untuk berfungsi sebagai alat
pencegahan. Oleh itu, Standard Antarabangsa ini tidak mempunyai suatu klausa atau
subklausa berasingan tentang tindakan pencegahan. Konsep tindakan pencegahan
dinyatakan melalui penggunaan pemikiran berasaskan risiko dalam merangka keperluan
sistem \textbf{pengurusan} kualiti.

\textbf{:prestasi:} Pemikiran berasaskan risiko yang diguna pakai dalam Standard Antarabangsa ini
membolehkan sedikit pengurangan dalam keperluan preskriptif dan penggantiannya dengan
keperluan berasaskan \textbf{prestasi}. Terdapat keluwesan lebih besar berbanding yang terkandung
dalam ISO 9001:2008 dalam keperluan untuk proses, maklumat didokumentasikan dan
tanggungjawab organisasi.

\textbf{:tanggungjawab:} Pemikiran berasaskan risiko yang diguna pakai dalam Standard Antarabangsa ini
membolehkan sedikit pengurangan dalam keperluan preskriptif dan penggantiannya dengan
keperluan berasaskan \textbf{prestasi}. Terdapat keluwesan lebih besar berbanding yang terkandung
dalam ISO 9001:2008 dalam keperluan untuk proses, maklumat didokumentasikan dan
\textbf{tanggungjawab} organisasi.

\textbf{:pengurusan:} Walaupun 6.1 menetapkan bahawa organisasi hendaklah merancang tindakan bagi
menyatakan risiko, tidak ada keperluan terhadap kaedah formal untuk \textbf{pengurusan} risiko atau
proses \textbf{pengurusan} risiko yang didokumentasikan. Organisasi boleh memutuskan sama ada
atau tidak untuk membangunkan metodologi \textbf{pengurusan} risiko yang lebih luas daripada yang
diperlukan oleh Standard Antarabangsa ini, contohnya melalui penggunaan panduan atau
standard lain.

\textbf{:pengurusan:} Tidak semua proses sistem \textbf{pengurusan} kualiti mewakili tahap risiko yang sama dari segi
keupayaan organisasi mencapai matlamatnya, dan kesan ketaktentuan adalah tidak sama
bagi semua organisasi. Di bawah keperluan 6.1, organisasi bertanggungjawab atas
penggunaan pemikiran berasaskan risiko dan tindakan yang diambil oleh organisasi untuk
menyatakan risiko, termasuk sama ada atau tidak untuk mengekalkan maklumat
didokumentasikan sebagai bukti penentuan risikonya.

\textbf{:tanggungjawab:} Tidak semua proses sistem \textbf{pengurusan} kualiti mewakili tahap risiko yang sama dari segi
keupayaan organisasi mencapai matlamatnya, dan kesan ketaktentuan adalah tidak sama
bagi semua organisasi. Di bawah keperluan 6.1, organisasi ber\textbf{tanggungjawab} atas
penggunaan pemikiran berasaskan risiko dan tindakan yang diambil oleh organisasi untuk
menyatakan risiko, termasuk sama ada atau tidak untuk mengekalkan maklumat
didokumentasikan sebagai bukti penentuan risikonya.

\textbf{:pengurusan:} Standard Antarabangsa ini tidak menyebut "tidak dimasukkan" berhubung dengan
kebolehgunaan keperluannya kepada sistem \textbf{pengurusan} kualiti organisasi. Walau
bagaimanapun, sesuatu organisasi boleh menyemak semula kebolehgunaan keperluan
disebabkan oleh saiz atau kerumitan organisasi, model \textbf{pengurusan} yang diterima guna,
pelbagai aktiviti organisasi dan jenis risiko dan peluang yang dihadapi.

\textbf{:pengurusan:} Keperluan untuk kebolehgunaan dinyatakan dalam 4.3, yang menetapkan keadaan yang
sesuatu organisasi boleh memutuskan bahawa sesuatu keperluan tidak boleh diguna pakai
untuk mana-mana proses dalam skop sistem \textbf{pengurusan} kualitinya. Organisasi hanya boleh
memutuskan bahawa sesuatu keperluan itu tidak terpakai jika keputusannya tidak akan
mengakibatkan kegagalan untuk mencapai keakuran produk dan perkhidmatan.

\textbf{:pengurusan:} Sebagai sebahagian daripada kesejajaran dengan standard sistem \textbf{pengurusan} yang lain,
klausa umum tentang "maklumat didokumentasikan" telah diterima guna tanpa perubahan
atau tambahan ketara (lihat 7.5). Jika sesuai, teks lain dalam Standard Antarabangsa ini telah
disejajarkan dengan keperluannya. Oleh itu, "maklumat didokumentasikan" digunakan untuk
semua keperluan dokumen.

\textbf{:rekod:} Jika ISO 9001:2008 menggunakan istilah "\textbf{rekod}" bagi menandakan dokumen yang
diperlukan untuk menyediakan bukti keakuran terhadap keperluan, hal ini kini dinyatakan
sebagai keperluan untuk "mengekalkan maklumat didokumentasikan". Organisasi
bertanggungjawab menentukan maklumat didokumentasikan yang perlu dikekalkan, tempoh
pengekalannya dan media yang akan digunakan untuk mengekalkannya.

\textbf{:tanggungjawab:} Jika ISO 9001:2008 menggunakan istilah "\textbf{rekod}" bagi menandakan dokumen yang
diperlukan untuk menyediakan bukti keakuran terhadap keperluan, hal ini kini dinyatakan
sebagai keperluan untuk "mengekalkan maklumat didokumentasikan". Organisasi
ber\textbf{tanggungjawab} menentukan maklumat didokumentasikan yang perlu dikekalkan, tempoh
pengekalannya dan media yang akan digunakan untuk mengekalkannya.

\textbf{:kawalan:} \#\#\#\# A.8 \textbf{kawalan} proses, produk dan perkhidmatan sediaan luar

\textbf{:kawalan:} \textbf{kawalan} yang diperlukan untuk penyediaan luar boleh sangat berbeza bergantung kepada
keadaan proses, produk dan perkhidmatan. Organisasi boleh mengguna pakai pemikiran
berasaskan risiko bagi menentukan jenis dan takat \textbf{kawalan} yang sesuai untuk penyedia luar
dan proses, produk dan perkhidmatan sediaan luar yang tertentu.

\textbf{:pengurusan:} Standard Antarabangsa lain tentang \textbf{pengurusan} kualiti dan sistem

\textbf{:pengurusan:} \textbf{pengurusan} kualiti yang dibangunkan oleh ISO/TC 176

\textbf{:senarai:} Standard Antarabangsa yang diperihalkan dalam lampiran ini telah dibangunkan oleh
ISO/TC 176 untuk menyediakan maklumat sokongan kepada organisasi yang menggunakan
Standard Antarabangsa ini, dan untuk memberi panduan kepada organisasi yang memilih
untuk maju melampaui keperluannya. Panduan atau keperluan yang terkandung dalam
dokumen yang ter\textbf{senarai} dalam lampiran ini tidak menambah, atau mengubah suai,
keperluan Standard Antarabangsa ini.

\textbf{:pengurusan:} Lampiran ini tidak termasuk rujukan standard sistem \textbf{pengurusan} kualiti sektor khusus yang
dibangunkan oleh ISO/TC 176.

\textbf{:kawalan:} - ISO 9000 Quality management systems - Fundamentals and vocabulary menyediakan
 latar belakang yang penting untuk pemahaman dan pelaksanaan yang sesuai bagi
 Standard Antarabangsa ini. Prinsip pengurusan kualiti diperihalkan secara terperinci
 dalam ISO 9000 dan telah diambil kira semasa pembangunan Standard Antarabangsa
 ini. Prinsip ini bukanlah keperluan sebenar, tetapi merupakan asas bagi keperluan yang
 ditetapkan oleh Standard Antarabangsa ini. ISO 9000 juga menetapkan terma, takrifan
 dan konsep yang digunakan dalam Standard Antarabangsa ini.
- ISO 9001 (Standard Antarabangsa ini) menetapkan keperluan yang tujuan utamanya
 adalah untuk memberi keyakinan terhadap produk dan perkhidmatan yang disediakan
 oleh sesuatu organisasi dan dengan itu meningkatkan kepuasan pelanggan.
 Pelaksanaannya yang sesuai juga boleh dijangka akan membawa manfaat lain kepada
 organisasi, seperti komunikasi dalaman yang bertambah baik, pemahaman dan \textbf{kawalan}
 proses organisasi yang lebih baik.
- ISO 9004 Managing for the sustained success of an organization - A quality
 management approach menyediakan panduan kepada organisasi yang memilih untuk
 maju melampaui keperluan Standard Antarabangsa ini, bagi menangani topik yang lebih
 meluas yang boleh membawa kepada penambahbaikan prestasi keseluruhan organisasi.
 ISO 9004 termasuk panduan tentang metodologi swapenilaian yang membolehkan
 sesebuah organisasi menilai tahap kematangan sistem pengurusan kualitinya.

\textbf{:pengurusan:} - ISO 9000 Quality management systems - Fundamentals and vocabulary menyediakan
 latar belakang yang penting untuk pemahaman dan pelaksanaan yang sesuai bagi
 Standard Antarabangsa ini. Prinsip \textbf{pengurusan} kualiti diperihalkan secara terperinci
 dalam ISO 9000 dan telah diambil kira semasa pembangunan Standard Antarabangsa
 ini. Prinsip ini bukanlah keperluan sebenar, tetapi merupakan asas bagi keperluan yang
 ditetapkan oleh Standard Antarabangsa ini. ISO 9000 juga menetapkan terma, takrifan
 dan konsep yang digunakan dalam Standard Antarabangsa ini.
- ISO 9001 (Standard Antarabangsa ini) menetapkan keperluan yang tujuan utamanya
 adalah untuk memberi keyakinan terhadap produk dan perkhidmatan yang disediakan
 oleh sesuatu organisasi dan dengan itu meningkatkan kepuasan pelanggan.
 Pelaksanaannya yang sesuai juga boleh dijangka akan membawa manfaat lain kepada
 organisasi, seperti komunikasi dalaman yang bertambah baik, pemahaman dan \textbf{kawalan}
 proses organisasi yang lebih baik.
- ISO 9004 Managing for the sustained success of an organization - A quality
 management approach menyediakan panduan kepada organisasi yang memilih untuk
 maju melampaui keperluan Standard Antarabangsa ini, bagi menangani topik yang lebih
 meluas yang boleh membawa kepada penambahbaikan prestasi keseluruhan organisasi.
 ISO 9004 termasuk panduan tentang metodologi swapenilaian yang membolehkan
 sesebuah organisasi menilai tahap kematangan sistem \textbf{pengurusan} kualitinya.

\textbf{:komunikasi:} - ISO 9000 Quality management systems - Fundamentals and vocabulary menyediakan
 latar belakang yang penting untuk pemahaman dan pelaksanaan yang sesuai bagi
 Standard Antarabangsa ini. Prinsip \textbf{pengurusan} kualiti diperihalkan secara terperinci
 dalam ISO 9000 dan telah diambil kira semasa pembangunan Standard Antarabangsa
 ini. Prinsip ini bukanlah keperluan sebenar, tetapi merupakan asas bagi keperluan yang
 ditetapkan oleh Standard Antarabangsa ini. ISO 9000 juga menetapkan terma, takrifan
 dan konsep yang digunakan dalam Standard Antarabangsa ini.
- ISO 9001 (Standard Antarabangsa ini) menetapkan keperluan yang tujuan utamanya
 adalah untuk memberi keyakinan terhadap produk dan perkhidmatan yang disediakan
 oleh sesuatu organisasi dan dengan itu meningkatkan kepuasan pelanggan.
 Pelaksanaannya yang sesuai juga boleh dijangka akan membawa manfaat lain kepada
 organisasi, seperti \textbf{komunikasi} dalaman yang bertambah baik, pemahaman dan \textbf{kawalan}
 proses organisasi yang lebih baik.
- ISO 9004 Managing for the sustained success of an organization - A quality
 management approach menyediakan panduan kepada organisasi yang memilih untuk
 maju melampaui keperluan Standard Antarabangsa ini, bagi menangani topik yang lebih
 meluas yang boleh membawa kepada penambahbaikan prestasi keseluruhan organisasi.
 ISO 9004 termasuk panduan tentang metodologi swapenilaian yang membolehkan
 sesebuah organisasi menilai tahap kematangan sistem \textbf{pengurusan} kualitinya.

\textbf{:prestasi:} - ISO 9000 Quality management systems - Fundamentals and vocabulary menyediakan
 latar belakang yang penting untuk pemahaman dan pelaksanaan yang sesuai bagi
 Standard Antarabangsa ini. Prinsip \textbf{pengurusan} kualiti diperihalkan secara terperinci
 dalam ISO 9000 dan telah diambil kira semasa pembangunan Standard Antarabangsa
 ini. Prinsip ini bukanlah keperluan sebenar, tetapi merupakan asas bagi keperluan yang
 ditetapkan oleh Standard Antarabangsa ini. ISO 9000 juga menetapkan terma, takrifan
 dan konsep yang digunakan dalam Standard Antarabangsa ini.
- ISO 9001 (Standard Antarabangsa ini) menetapkan keperluan yang tujuan utamanya
 adalah untuk memberi keyakinan terhadap produk dan perkhidmatan yang disediakan
 oleh sesuatu organisasi dan dengan itu meningkatkan kepuasan pelanggan.
 Pelaksanaannya yang sesuai juga boleh dijangka akan membawa manfaat lain kepada
 organisasi, seperti \textbf{komunikasi} dalaman yang bertambah baik, pemahaman dan \textbf{kawalan}
 proses organisasi yang lebih baik.
- ISO 9004 Managing for the sustained success of an organization - A quality
 management approach menyediakan panduan kepada organisasi yang memilih untuk
 maju melampaui keperluan Standard Antarabangsa ini, bagi menangani topik yang lebih
 meluas yang boleh membawa kepada penambahbaikan \textbf{prestasi} keseluruhan organisasi.
 ISO 9004 termasuk panduan tentang metodologi swapenilaian yang membolehkan
 sesebuah organisasi menilai tahap kematangan sistem \textbf{pengurusan} kualitinya.

\textbf{:pelanggan:} - ISO 9000 Quality management systems - Fundamentals and vocabulary menyediakan
 latar belakang yang penting untuk pemahaman dan pelaksanaan yang sesuai bagi
 Standard Antarabangsa ini. Prinsip \textbf{pengurusan} kualiti diperihalkan secara terperinci
 dalam ISO 9000 dan telah diambil kira semasa pembangunan Standard Antarabangsa
 ini. Prinsip ini bukanlah keperluan sebenar, tetapi merupakan asas bagi keperluan yang
 ditetapkan oleh Standard Antarabangsa ini. ISO 9000 juga menetapkan terma, takrifan
 dan konsep yang digunakan dalam Standard Antarabangsa ini.
- ISO 9001 (Standard Antarabangsa ini) menetapkan keperluan yang tujuan utamanya
 adalah untuk memberi keyakinan terhadap produk dan perkhidmatan yang disediakan
 oleh sesuatu organisasi dan dengan itu meningkatkan kepuasan \textbf{pelanggan}.
 Pelaksanaannya yang sesuai juga boleh dijangka akan membawa manfaat lain kepada
 organisasi, seperti \textbf{komunikasi} dalaman yang bertambah baik, pemahaman dan \textbf{kawalan}
 proses organisasi yang lebih baik.
- ISO 9004 Managing for the sustained success of an organization - A quality
 management approach menyediakan panduan kepada organisasi yang memilih untuk
 maju melampaui keperluan Standard Antarabangsa ini, bagi menangani topik yang lebih
 meluas yang boleh membawa kepada penambahbaikan \textbf{prestasi} keseluruhan organisasi.
 ISO 9004 termasuk panduan tentang metodologi swapenilaian yang membolehkan
 sesebuah organisasi menilai tahap kematangan sistem \textbf{pengurusan} kualitinya.

\textbf{:penambahbaikan:} - ISO 9000 Quality management systems - Fundamentals and vocabulary menyediakan
 latar belakang yang penting untuk pemahaman dan pelaksanaan yang sesuai bagi
 Standard Antarabangsa ini. Prinsip \textbf{pengurusan} kualiti diperihalkan secara terperinci
 dalam ISO 9000 dan telah diambil kira semasa pembangunan Standard Antarabangsa
 ini. Prinsip ini bukanlah keperluan sebenar, tetapi merupakan asas bagi keperluan yang
 ditetapkan oleh Standard Antarabangsa ini. ISO 9000 juga menetapkan terma, takrifan
 dan konsep yang digunakan dalam Standard Antarabangsa ini.
- ISO 9001 (Standard Antarabangsa ini) menetapkan keperluan yang tujuan utamanya
 adalah untuk memberi keyakinan terhadap produk dan perkhidmatan yang disediakan
 oleh sesuatu organisasi dan dengan itu meningkatkan kepuasan \textbf{pelanggan}.
 Pelaksanaannya yang sesuai juga boleh dijangka akan membawa manfaat lain kepada
 organisasi, seperti \textbf{komunikasi} dalaman yang bertambah baik, pemahaman dan \textbf{kawalan}
 proses organisasi yang lebih baik.
- ISO 9004 Managing for the sustained success of an organization - A quality
 management approach menyediakan panduan kepada organisasi yang memilih untuk
 maju melampaui keperluan Standard Antarabangsa ini, bagi menangani topik yang lebih
 meluas yang boleh membawa kepada \textbf{penambahbaikan} \textbf{prestasi} keseluruhan organisasi.
 ISO 9004 termasuk panduan tentang metodologi swapenilaian yang membolehkan
 sesebuah organisasi menilai tahap kematangan sistem \textbf{pengurusan} kualitinya.

\textbf{:pengurusan:} Standard Antarabangsa yang digariskan di bawah boleh membantu organisasi apabila
organisasi mewujudkan atau ingin menambah baik sistem kualiti \textbf{pengurusan}, proses atau
aktivitinya.

\textbf{:kawalan:} - ISO 10001 Quality management - Customer satisfaction - Guidelines for codes of
 conduct for organizations menyediakan panduan untuk sesuatu organisasi dalam
 menentukan bahawa peruntukan kepuasan pelanggannya memenuhi keperluan dan
 jangkaan pelanggan. Penggunaannya dapat meningkatkan keyakinan pelanggan
 terhadap sesuatu organisasi dan menambah baik pemahaman pelanggan tentang apa
 yang diharapkan daripada sesuatu organisasi, dan dengan itu mengurangkan
 kemungkinan salah faham dan aduan.
- ISO 10002 Quality management - Customer satisfaction - Guidelines for complaints
 handling in organizations menyediakan panduan tentang proses pengendalian aduan
 dengan mengambil kira dan menangani keperluan dan jangkaan pengadu serta
 menyelesaikan apa-apa aduan yang diterima. ISO 10002 menyediakan proses aduan
 terbuka, berkesan dan mudah untuk digunakan, termasuk latihan modal insan. Ia juga
 menyediakan panduan untuk perniagaan kecil.
- ISO 10003 Quality management - Customer satisfaction - Guidelines for dispute
 resolution external to organizations menyediakan panduan bagi penyelesaian pertikaian
 luar yang cekap dan berkesan untuk aduan berkaitan produk. Penyelesaian pertikaian
 memberikan ruang bagi tebus rugi apabila organisasi tidak membetulkan aduan secara
 dalaman. Kebanyakan aduan boleh diselesaikan dengan jayanya dalam organisasi,
 tanpa prosedur bertentangan.
- ISO 10004 Quality management - Customer satisfaction - Guidelines for monitoring and
 measuring menyediakan garis panduan bagi tindakan untuk meningkatkan kepuasan
 pelanggan dan untuk menentukan peluang bagi penambahbaikan produk, proses dan
 sifat khusus yang dihargai oleh pelanggan. Tindakan sedemikian boleh mengukuhkan
 kesetiaan pelanggan dan membantu mengekalkan pelanggan.
- ISO 10005 Quality management systems - Guidelines for quality plans menyediakan
 panduan bagi mewujudkan dan menggunakan rancangan kualiti sebagai suatu cara
 untuk menghubungkan keperluan proses, produk, projek atau kontrak, dengan kaedah
 dan amalan kerja yang menyokong penghasilan produk. Manfaat mewujudkan rancangan
 kualiti ialah peningkatan keyakinan bahawa keperluan akan dipenuhi, proses berada
 dalam \textbf{kawalan} dan motivasi yang dapat diberikan kepada mereka yang terlibat.
- ISO 10006 Quality management systems - Guidelines for quality management in projects
 terpakai untuk projek kecil mahupun yang besar, mudah mahupun rumit, projek individu
 mahupun sebahagian daripada portfolio projek. ISO 10006 adalah untuk digunakan oleh
 kakitangan yang mengurus projek dan sesiapa yang perlu bagi memastikan bahawa
 organisasi mereka mengguna pakai amalan yang terkandung dalam standard sistem
 pengurusan kualiti ISO.
- ISO 10007 Quality management systems - Guidelines for configuration management
 adalah untuk membantu organisasi yang mengguna pakai pengurusan konfigurasi bagi
 hala tuju teknikal dan pentadbiran sepanjang kitaran hayat produk. Pengurusan
 konfigurasi boleh digunakan untuk memenuhi keperluan pengenalpastian dan
 kebolehkesanan produk yang dinyatakan dalam Standard Antarabangsa ini.

\textbf{:pengurusan:} - ISO 10001 Quality management - Customer satisfaction - Guidelines for codes of
 conduct for organizations menyediakan panduan untuk sesuatu organisasi dalam
 menentukan bahawa peruntukan kepuasan pelanggannya memenuhi keperluan dan
 jangkaan pelanggan. Penggunaannya dapat meningkatkan keyakinan pelanggan
 terhadap sesuatu organisasi dan menambah baik pemahaman pelanggan tentang apa
 yang diharapkan daripada sesuatu organisasi, dan dengan itu mengurangkan
 kemungkinan salah faham dan aduan.
- ISO 10002 Quality management - Customer satisfaction - Guidelines for complaints
 handling in organizations menyediakan panduan tentang proses pengendalian aduan
 dengan mengambil kira dan menangani keperluan dan jangkaan pengadu serta
 menyelesaikan apa-apa aduan yang diterima. ISO 10002 menyediakan proses aduan
 terbuka, berkesan dan mudah untuk digunakan, termasuk latihan modal insan. Ia juga
 menyediakan panduan untuk perniagaan kecil.
- ISO 10003 Quality management - Customer satisfaction - Guidelines for dispute
 resolution external to organizations menyediakan panduan bagi penyelesaian pertikaian
 luar yang cekap dan berkesan untuk aduan berkaitan produk. Penyelesaian pertikaian
 memberikan ruang bagi tebus rugi apabila organisasi tidak membetulkan aduan secara
 dalaman. Kebanyakan aduan boleh diselesaikan dengan jayanya dalam organisasi,
 tanpa prosedur bertentangan.
- ISO 10004 Quality management - Customer satisfaction - Guidelines for monitoring and
 measuring menyediakan garis panduan bagi tindakan untuk meningkatkan kepuasan
 pelanggan dan untuk menentukan peluang bagi penambahbaikan produk, proses dan
 sifat khusus yang dihargai oleh pelanggan. Tindakan sedemikian boleh mengukuhkan
 kesetiaan pelanggan dan membantu mengekalkan pelanggan.
- ISO 10005 Quality management systems - Guidelines for quality plans menyediakan
 panduan bagi mewujudkan dan menggunakan rancangan kualiti sebagai suatu cara
 untuk menghubungkan keperluan proses, produk, projek atau kontrak, dengan kaedah
 dan amalan kerja yang menyokong penghasilan produk. Manfaat mewujudkan rancangan
 kualiti ialah peningkatan keyakinan bahawa keperluan akan dipenuhi, proses berada
 dalam \textbf{kawalan} dan motivasi yang dapat diberikan kepada mereka yang terlibat.
- ISO 10006 Quality management systems - Guidelines for quality management in projects
 terpakai untuk projek kecil mahupun yang besar, mudah mahupun rumit, projek individu
 mahupun sebahagian daripada portfolio projek. ISO 10006 adalah untuk digunakan oleh
 kakitangan yang mengurus projek dan sesiapa yang perlu bagi memastikan bahawa
 organisasi mereka mengguna pakai amalan yang terkandung dalam standard sistem
 \textbf{pengurusan} kualiti ISO.
- ISO 10007 Quality management systems - Guidelines for configuration management
 adalah untuk membantu organisasi yang mengguna pakai \textbf{pengurusan} konfigurasi bagi
 hala tuju teknikal dan pentadbiran sepanjang kitaran hayat produk. \textbf{pengurusan}
 konfigurasi boleh digunakan untuk memenuhi keperluan pengenalpastian dan
 kebolehkesanan produk yang dinyatakan dalam Standard Antarabangsa ini.

\textbf{:pelanggan:} - ISO 10001 Quality management - Customer satisfaction - Guidelines for codes of
 conduct for organizations menyediakan panduan untuk sesuatu organisasi dalam
 menentukan bahawa peruntukan kepuasan \textbf{pelanggan}nya memenuhi keperluan dan
 jangkaan \textbf{pelanggan}. Penggunaannya dapat meningkatkan keyakinan \textbf{pelanggan}
 terhadap sesuatu organisasi dan menambah baik pemahaman \textbf{pelanggan} tentang apa
 yang diharapkan daripada sesuatu organisasi, dan dengan itu mengurangkan
 kemungkinan salah faham dan aduan.
- ISO 10002 Quality management - Customer satisfaction - Guidelines for complaints
 handling in organizations menyediakan panduan tentang proses pengendalian aduan
 dengan mengambil kira dan menangani keperluan dan jangkaan pengadu serta
 menyelesaikan apa-apa aduan yang diterima. ISO 10002 menyediakan proses aduan
 terbuka, berkesan dan mudah untuk digunakan, termasuk latihan modal insan. Ia juga
 menyediakan panduan untuk perniagaan kecil.
- ISO 10003 Quality management - Customer satisfaction - Guidelines for dispute
 resolution external to organizations menyediakan panduan bagi penyelesaian pertikaian
 luar yang cekap dan berkesan untuk aduan berkaitan produk. Penyelesaian pertikaian
 memberikan ruang bagi tebus rugi apabila organisasi tidak membetulkan aduan secara
 dalaman. Kebanyakan aduan boleh diselesaikan dengan jayanya dalam organisasi,
 tanpa prosedur bertentangan.
- ISO 10004 Quality management - Customer satisfaction - Guidelines for monitoring and
 measuring menyediakan garis panduan bagi tindakan untuk meningkatkan kepuasan
 \textbf{pelanggan} dan untuk menentukan peluang bagi penambahbaikan produk, proses dan
 sifat khusus yang dihargai oleh \textbf{pelanggan}. Tindakan sedemikian boleh mengukuhkan
 kesetiaan \textbf{pelanggan} dan membantu mengekalkan \textbf{pelanggan}.
- ISO 10005 Quality management systems - Guidelines for quality plans menyediakan
 panduan bagi mewujudkan dan menggunakan rancangan kualiti sebagai suatu cara
 untuk menghubungkan keperluan proses, produk, projek atau kontrak, dengan kaedah
 dan amalan kerja yang menyokong penghasilan produk. Manfaat mewujudkan rancangan
 kualiti ialah peningkatan keyakinan bahawa keperluan akan dipenuhi, proses berada
 dalam \textbf{kawalan} dan motivasi yang dapat diberikan kepada mereka yang terlibat.
- ISO 10006 Quality management systems - Guidelines for quality management in projects
 terpakai untuk projek kecil mahupun yang besar, mudah mahupun rumit, projek individu
 mahupun sebahagian daripada portfolio projek. ISO 10006 adalah untuk digunakan oleh
 kakitangan yang mengurus projek dan sesiapa yang perlu bagi memastikan bahawa
 organisasi mereka mengguna pakai amalan yang terkandung dalam standard sistem
 \textbf{pengurusan} kualiti ISO.
- ISO 10007 Quality management systems - Guidelines for configuration management
 adalah untuk membantu organisasi yang mengguna pakai \textbf{pengurusan} konfigurasi bagi
 hala tuju teknikal dan pentadbiran sepanjang kitaran hayat produk. \textbf{pengurusan}
 konfigurasi boleh digunakan untuk memenuhi keperluan pengenalpastian dan
 kebolehkesanan produk yang dinyatakan dalam Standard Antarabangsa ini.

\textbf{:penambahbaikan:} - ISO 10001 Quality management - Customer satisfaction - Guidelines for codes of
 conduct for organizations menyediakan panduan untuk sesuatu organisasi dalam
 menentukan bahawa peruntukan kepuasan \textbf{pelanggan}nya memenuhi keperluan dan
 jangkaan \textbf{pelanggan}. Penggunaannya dapat meningkatkan keyakinan \textbf{pelanggan}
 terhadap sesuatu organisasi dan menambah baik pemahaman \textbf{pelanggan} tentang apa
 yang diharapkan daripada sesuatu organisasi, dan dengan itu mengurangkan
 kemungkinan salah faham dan aduan.
- ISO 10002 Quality management - Customer satisfaction - Guidelines for complaints
 handling in organizations menyediakan panduan tentang proses pengendalian aduan
 dengan mengambil kira dan menangani keperluan dan jangkaan pengadu serta
 menyelesaikan apa-apa aduan yang diterima. ISO 10002 menyediakan proses aduan
 terbuka, berkesan dan mudah untuk digunakan, termasuk latihan modal insan. Ia juga
 menyediakan panduan untuk perniagaan kecil.
- ISO 10003 Quality management - Customer satisfaction - Guidelines for dispute
 resolution external to organizations menyediakan panduan bagi penyelesaian pertikaian
 luar yang cekap dan berkesan untuk aduan berkaitan produk. Penyelesaian pertikaian
 memberikan ruang bagi tebus rugi apabila organisasi tidak membetulkan aduan secara
 dalaman. Kebanyakan aduan boleh diselesaikan dengan jayanya dalam organisasi,
 tanpa prosedur bertentangan.
- ISO 10004 Quality management - Customer satisfaction - Guidelines for monitoring and
 measuring menyediakan garis panduan bagi tindakan untuk meningkatkan kepuasan
 \textbf{pelanggan} dan untuk menentukan peluang bagi \textbf{penambahbaikan} produk, proses dan
 sifat khusus yang dihargai oleh \textbf{pelanggan}. Tindakan sedemikian boleh mengukuhkan
 kesetiaan \textbf{pelanggan} dan membantu mengekalkan \textbf{pelanggan}.
- ISO 10005 Quality management systems - Guidelines for quality plans menyediakan
 panduan bagi mewujudkan dan menggunakan rancangan kualiti sebagai suatu cara
 untuk menghubungkan keperluan proses, produk, projek atau kontrak, dengan kaedah
 dan amalan kerja yang menyokong penghasilan produk. Manfaat mewujudkan rancangan
 kualiti ialah peningkatan keyakinan bahawa keperluan akan dipenuhi, proses berada
 dalam \textbf{kawalan} dan motivasi yang dapat diberikan kepada mereka yang terlibat.
- ISO 10006 Quality management systems - Guidelines for quality management in projects
 terpakai untuk projek kecil mahupun yang besar, mudah mahupun rumit, projek individu
 mahupun sebahagian daripada portfolio projek. ISO 10006 adalah untuk digunakan oleh
 kakitangan yang mengurus projek dan sesiapa yang perlu bagi memastikan bahawa
 organisasi mereka mengguna pakai amalan yang terkandung dalam standard sistem
 \textbf{pengurusan} kualiti ISO.
- ISO 10007 Quality management systems - Guidelines for configuration management
 adalah untuk membantu organisasi yang mengguna pakai \textbf{pengurusan} konfigurasi bagi
 hala tuju teknikal dan pentadbiran sepanjang kitaran hayat produk. \textbf{pengurusan}
 konfigurasi boleh digunakan untuk memenuhi keperluan pengenalpastian dan
 kebolehkesanan produk yang dinyatakan dalam Standard Antarabangsa ini.

\textbf{:kompeten:} - ISO 10008 Quality management - Customer satisfaction - Guidelines for business-to-
 consumer electronic commerce transactions memberikan panduan tentang cara
 organisasi boleh melaksanakan sistem urus niaga perdagangan elektronik perniagaan-
 ke-pengguna yang berkesan dan cekap (B2C ECT), dan dengan itu menyediakan asas
 supaya pengguna mempunyai keyakinan yang meningkat terhadap B2C ECT,
 meningkatkan keupayaan organisasi untuk memuaskan hati pengguna dan membantu
 mengurangkan aduan dan pertikaian.
- ISO 10012 Measurement management systems - Requirements for measurement
 processes and measuring menyediakan panduan bagi pengurusan proses pengukuran
 dan pengesahan metrologi peralatan pengukuran yang digunakan untuk menyokong dan
 menunjukkan pematuhan terhadap keperluan metrologi. ISO 10012 menyediakan kriteria
 pengurusan kualiti untuk sistem pengurusan pengukuran bagi memastikan keperluan
 metrologi dipenuhi.
- ISO/TR 10013 Guidelines for quality management system documentation menyediakan
 garis panduan untuk pembangunan dan penyelenggaraan pendokumenan yang
 diperlukan untuk sistem pengurusan kualiti. ISO/TR 10013 boleh digunakan untuk
 mendokumenkan sistem pengurusan selain standard sistem pengurusan kualiti
 ISO, contohnya sistem pengurusan alam sekitar dan sistem pengurusan keselamatan.
- ISO 10014 Quality management - Guidelines for realizing financial and economic
 benefits ditujukan kepada pengurusan atasan. Ia menyediakan garis panduan untuk
 merealisasikan faedah kewangan dan ekonomi melalui penggunaan prinsip pengurusan
 kualiti. Ia memudahkan penggunaan prinsip pengurusan dan pemilihan kaedah dan alat
 yang membolehkan kejayaan mampan sesuatu organisasi.
- ISO 10015 Quality management - Guidelines for training menyediakan garis panduan
 untuk membantu organisasi dalam menangani isu yang berkaitan dengan latihan. ISO
 10015 boleh diguna pakai pada bila-bila masa bimbingan diperlukan untuk mentafsir
 perkataan "pendidikan" dan "latihan" dalam standard sistem pengurusan kualiti ISO. Apa-
 apa perkataan "latihan" termasuk semua jenis pendidikan dan latihan.
- ISO/TR 10017 Guidance on statistical techniques for ISO 9001:2000 memperjelaskan
 teknik statistik yang bermula seawal perubahan yang dapat dilihat dalam tingkah laku dan
 hasil proses, walaupun di bawah keadaan kestabilan nyata. Teknik statistik
 membolehkan penggunaan yang lebih baik terhadap data yang ada untuk membantu
 dalam membuat keputusan, dan dengan itu membantu untuk terus menambah baik kualiti
 produk dan proses bagi mencapai kepuasan pelanggan.
- ISO 10018 Quality management - Guidelines on people involvement and competence
 menyediakan garis panduan yang mempengaruhi pelibatan modal insan dan
 ke\textbf{kompeten}an. Sistem pengurusan kualiti bergantung kepada pelibatan orang yang
 \textbf{kompeten} dan cara mereka diperkenalkan dan disepadukan dalam organisasi. Penting
 untuk menentukan, membangun dan menilai pengetahuan, kemahiran, tingkah laku dan
 persekitaran kerja yang diperlukan.
- ISO 10019 Guidelines for the selection of quality management system consultants and
 use of their services menyediakan panduan tentang pemilihan perunding sistem
 pengurusan kualiti dan penggunaan perkhidmatan mereka. Ia memberikan panduan
 tentang proses penilaian ke\textbf{kompeten}an perunding sistem pengurusan kualiti dan
 memberi keyakinan bahawa keperluan dan jangkaan organisasi bagi perkhidmatan
 perunding akan dipenuhi.

\textbf{:pengurusan:} - ISO 10008 Quality management - Customer satisfaction - Guidelines for business-to-
 consumer electronic commerce transactions memberikan panduan tentang cara
 organisasi boleh melaksanakan sistem urus niaga perdagangan elektronik perniagaan-
 ke-pengguna yang berkesan dan cekap (B2C ECT), dan dengan itu menyediakan asas
 supaya pengguna mempunyai keyakinan yang meningkat terhadap B2C ECT,
 meningkatkan keupayaan organisasi untuk memuaskan hati pengguna dan membantu
 mengurangkan aduan dan pertikaian.
- ISO 10012 Measurement management systems - Requirements for measurement
 processes and measuring menyediakan panduan bagi \textbf{pengurusan} proses pengukuran
 dan pengesahan metrologi peralatan pengukuran yang digunakan untuk menyokong dan
 menunjukkan pematuhan terhadap keperluan metrologi. ISO 10012 menyediakan kriteria
 \textbf{pengurusan} kualiti untuk sistem \textbf{pengurusan} pengukuran bagi memastikan keperluan
 metrologi dipenuhi.
- ISO/TR 10013 Guidelines for quality management system documentation menyediakan
 garis panduan untuk pembangunan dan penyelenggaraan pendokumenan yang
 diperlukan untuk sistem \textbf{pengurusan} kualiti. ISO/TR 10013 boleh digunakan untuk
 mendokumenkan sistem \textbf{pengurusan} selain standard sistem \textbf{pengurusan} kualiti
 ISO, contohnya sistem \textbf{pengurusan} alam sekitar dan sistem \textbf{pengurusan} keselamatan.
- ISO 10014 Quality management - Guidelines for realizing financial and economic
 benefits ditujukan kepada \textbf{pengurusan} atasan. Ia menyediakan garis panduan untuk
 merealisasikan faedah kewangan dan ekonomi melalui penggunaan prinsip \textbf{pengurusan}
 kualiti. Ia memudahkan penggunaan prinsip \textbf{pengurusan} dan pemilihan kaedah dan alat
 yang membolehkan kejayaan mampan sesuatu organisasi.
- ISO 10015 Quality management - Guidelines for training menyediakan garis panduan
 untuk membantu organisasi dalam menangani isu yang berkaitan dengan latihan. ISO
 10015 boleh diguna pakai pada bila-bila masa bimbingan diperlukan untuk mentafsir
 perkataan "pendidikan" dan "latihan" dalam standard sistem \textbf{pengurusan} kualiti ISO. Apa-
 apa perkataan "latihan" termasuk semua jenis pendidikan dan latihan.
- ISO/TR 10017 Guidance on statistical techniques for ISO 9001:2000 memperjelaskan
 teknik statistik yang bermula seawal perubahan yang dapat dilihat dalam tingkah laku dan
 hasil proses, walaupun di bawah keadaan kestabilan nyata. Teknik statistik
 membolehkan penggunaan yang lebih baik terhadap data yang ada untuk membantu
 dalam membuat keputusan, dan dengan itu membantu untuk terus menambah baik kualiti
 produk dan proses bagi mencapai kepuasan pelanggan.
- ISO 10018 Quality management - Guidelines on people involvement and competence
 menyediakan garis panduan yang mempengaruhi pelibatan modal insan dan
 ke\textbf{kompeten}an. Sistem \textbf{pengurusan} kualiti bergantung kepada pelibatan orang yang
 \textbf{kompeten} dan cara mereka diperkenalkan dan disepadukan dalam organisasi. Penting
 untuk menentukan, membangun dan menilai pengetahuan, kemahiran, tingkah laku dan
 persekitaran kerja yang diperlukan.
- ISO 10019 Guidelines for the selection of quality management system consultants and
 use of their services menyediakan panduan tentang pemilihan perunding sistem
 \textbf{pengurusan} kualiti dan penggunaan perkhidmatan mereka. Ia memberikan panduan
 tentang proses penilaian ke\textbf{kompeten}an perunding sistem \textbf{pengurusan} kualiti dan
 memberi keyakinan bahawa keperluan dan jangkaan organisasi bagi perkhidmatan
 perunding akan dipenuhi.

\textbf{:ukur:} - ISO 10008 Quality management - Customer satisfaction - Guidelines for business-to-
 consumer electronic commerce transactions memberikan panduan tentang cara
 organisasi boleh melaksanakan sistem urus niaga perdagangan elektronik perniagaan-
 ke-pengguna yang berkesan dan cekap (B2C ECT), dan dengan itu menyediakan asas
 supaya pengguna mempunyai keyakinan yang meningkat terhadap B2C ECT,
 meningkatkan keupayaan organisasi untuk memuaskan hati pengguna dan membantu
 mengurangkan aduan dan pertikaian.
- ISO 10012 Measurement management systems - Requirements for measurement
 processes and measuring menyediakan panduan bagi \textbf{pengurusan} proses peng\textbf{ukur}an
 dan pengesahan metrologi peralatan peng\textbf{ukur}an yang digunakan untuk menyokong dan
 menunjukkan pematuhan terhadap keperluan metrologi. ISO 10012 menyediakan kriteria
 \textbf{pengurusan} kualiti untuk sistem \textbf{pengurusan} peng\textbf{ukur}an bagi memastikan keperluan
 metrologi dipenuhi.
- ISO/TR 10013 Guidelines for quality management system documentation menyediakan
 garis panduan untuk pembangunan dan penyelenggaraan pendokumenan yang
 diperlukan untuk sistem \textbf{pengurusan} kualiti. ISO/TR 10013 boleh digunakan untuk
 mendokumenkan sistem \textbf{pengurusan} selain standard sistem \textbf{pengurusan} kualiti
 ISO, contohnya sistem \textbf{pengurusan} alam sekitar dan sistem \textbf{pengurusan} keselamatan.
- ISO 10014 Quality management - Guidelines for realizing financial and economic
 benefits ditujukan kepada \textbf{pengurusan} atasan. Ia menyediakan garis panduan untuk
 merealisasikan faedah kewangan dan ekonomi melalui penggunaan prinsip \textbf{pengurusan}
 kualiti. Ia memudahkan penggunaan prinsip \textbf{pengurusan} dan pemilihan kaedah dan alat
 yang membolehkan kejayaan mampan sesuatu organisasi.
- ISO 10015 Quality management - Guidelines for training menyediakan garis panduan
 untuk membantu organisasi dalam menangani isu yang berkaitan dengan latihan. ISO
 10015 boleh diguna pakai pada bila-bila masa bimbingan diperlukan untuk mentafsir
 perkataan "pendidikan" dan "latihan" dalam standard sistem \textbf{pengurusan} kualiti ISO. Apa-
 apa perkataan "latihan" termasuk semua jenis pendidikan dan latihan.
- ISO/TR 10017 Guidance on statistical techniques for ISO 9001:2000 memperjelaskan
 teknik statistik yang bermula seawal perubahan yang dapat dilihat dalam tingkah laku dan
 hasil proses, walaupun di bawah keadaan kestabilan nyata. Teknik statistik
 membolehkan penggunaan yang lebih baik terhadap data yang ada untuk membantu
 dalam membuat keputusan, dan dengan itu membantu untuk terus menambah baik kualiti
 produk dan proses bagi mencapai kepuasan pelanggan.
- ISO 10018 Quality management - Guidelines on people involvement and competence
 menyediakan garis panduan yang mempengaruhi pelibatan modal insan dan
 ke\textbf{kompeten}an. Sistem \textbf{pengurusan} kualiti bergantung kepada pelibatan orang yang
 \textbf{kompeten} dan cara mereka diperkenalkan dan disepadukan dalam organisasi. Penting
 untuk menentukan, membangun dan menilai pengetahuan, kemahiran, tingkah laku dan
 persekitaran kerja yang diperlukan.
- ISO 10019 Guidelines for the selection of quality management system consultants and
 use of their services menyediakan panduan tentang pemilihan perunding sistem
 \textbf{pengurusan} kualiti dan penggunaan perkhidmatan mereka. Ia memberikan panduan
 tentang proses penilaian ke\textbf{kompeten}an perunding sistem \textbf{pengurusan} kualiti dan
 memberi keyakinan bahawa keperluan dan jangkaan organisasi bagi perkhidmatan
 perunding akan dipenuhi.

\textbf{:pelanggan:} - ISO 10008 Quality management - Customer satisfaction - Guidelines for business-to-
 consumer electronic commerce transactions memberikan panduan tentang cara
 organisasi boleh melaksanakan sistem urus niaga perdagangan elektronik perniagaan-
 ke-pengguna yang berkesan dan cekap (B2C ECT), dan dengan itu menyediakan asas
 supaya pengguna mempunyai keyakinan yang meningkat terhadap B2C ECT,
 meningkatkan keupayaan organisasi untuk memuaskan hati pengguna dan membantu
 mengurangkan aduan dan pertikaian.
- ISO 10012 Measurement management systems - Requirements for measurement
 processes and measuring menyediakan panduan bagi \textbf{pengurusan} proses peng\textbf{ukur}an
 dan pengesahan metrologi peralatan peng\textbf{ukur}an yang digunakan untuk menyokong dan
 menunjukkan pematuhan terhadap keperluan metrologi. ISO 10012 menyediakan kriteria
 \textbf{pengurusan} kualiti untuk sistem \textbf{pengurusan} peng\textbf{ukur}an bagi memastikan keperluan
 metrologi dipenuhi.
- ISO/TR 10013 Guidelines for quality management system documentation menyediakan
 garis panduan untuk pembangunan dan penyelenggaraan pendokumenan yang
 diperlukan untuk sistem \textbf{pengurusan} kualiti. ISO/TR 10013 boleh digunakan untuk
 mendokumenkan sistem \textbf{pengurusan} selain standard sistem \textbf{pengurusan} kualiti
 ISO, contohnya sistem \textbf{pengurusan} alam sekitar dan sistem \textbf{pengurusan} keselamatan.
- ISO 10014 Quality management - Guidelines for realizing financial and economic
 benefits ditujukan kepada \textbf{pengurusan} atasan. Ia menyediakan garis panduan untuk
 merealisasikan faedah kewangan dan ekonomi melalui penggunaan prinsip \textbf{pengurusan}
 kualiti. Ia memudahkan penggunaan prinsip \textbf{pengurusan} dan pemilihan kaedah dan alat
 yang membolehkan kejayaan mampan sesuatu organisasi.
- ISO 10015 Quality management - Guidelines for training menyediakan garis panduan
 untuk membantu organisasi dalam menangani isu yang berkaitan dengan latihan. ISO
 10015 boleh diguna pakai pada bila-bila masa bimbingan diperlukan untuk mentafsir
 perkataan "pendidikan" dan "latihan" dalam standard sistem \textbf{pengurusan} kualiti ISO. Apa-
 apa perkataan "latihan" termasuk semua jenis pendidikan dan latihan.
- ISO/TR 10017 Guidance on statistical techniques for ISO 9001:2000 memperjelaskan
 teknik statistik yang bermula seawal perubahan yang dapat dilihat dalam tingkah laku dan
 hasil proses, walaupun di bawah keadaan kestabilan nyata. Teknik statistik
 membolehkan penggunaan yang lebih baik terhadap data yang ada untuk membantu
 dalam membuat keputusan, dan dengan itu membantu untuk terus menambah baik kualiti
 produk dan proses bagi mencapai kepuasan \textbf{pelanggan}.
- ISO 10018 Quality management - Guidelines on people involvement and competence
 menyediakan garis panduan yang mempengaruhi pelibatan modal insan dan
 ke\textbf{kompeten}an. Sistem \textbf{pengurusan} kualiti bergantung kepada pelibatan orang yang
 \textbf{kompeten} dan cara mereka diperkenalkan dan disepadukan dalam organisasi. Penting
 untuk menentukan, membangun dan menilai pengetahuan, kemahiran, tingkah laku dan
 persekitaran kerja yang diperlukan.
- ISO 10019 Guidelines for the selection of quality management system consultants and
 use of their services menyediakan panduan tentang pemilihan perunding sistem
 \textbf{pengurusan} kualiti dan penggunaan perkhidmatan mereka. Ia memberikan panduan
 tentang proses penilaian ke\textbf{kompeten}an perunding sistem \textbf{pengurusan} kualiti dan
 memberi keyakinan bahawa keperluan dan jangkaan organisasi bagi perkhidmatan
 perunding akan dipenuhi.

\textbf{:kompeten:} - ISO 19011 Guidelines for auditing management systems menyediakan panduan tentang
 pengurusan program audit, perancangan dan pengendalian audit sistem pengurusan,
 serta tentang ke\textbf{kompeten}an dan penilaian juruaudit dan pasukan audit. ISO 19011
 bertujuan untuk diguna pakai oleh juruaudit, organisasi yang melaksanakan sistem
 pengurusan dan organisasi yang perlu menjalankan audit sistem pengurusan.

\textbf{:pengurusan:} - ISO 19011 Guidelines for auditing management systems menyediakan panduan tentang
 \textbf{pengurusan} program audit, perancangan dan pengendalian audit sistem \textbf{pengurusan},
 serta tentang ke\textbf{kompeten}an dan penilaian juruaudit dan pasukan audit. ISO 19011
 bertujuan untuk diguna pakai oleh juruaudit, organisasi yang melaksanakan sistem
 \textbf{pengurusan} dan organisasi yang perlu menjalankan audit sistem \textbf{pengurusan}.

\textbf{:pengurusan:} Ahli Jawatankuasa Teknikal \textbf{pengurusan} Kualiti dan Penentuan Kualiti (TC2) mengenai
Sistem Kualiti
Name Organisation

\textbf{:pengurusan:} Encik Mohd Azani Jabar Unit Pemodenan Tadbiran dan
Perancangan \textbf{pengurusan} Malaysia

\textbf{:pengurusan:} Unit Pemodenan Tadbiran dan
Perancangan \textbf{pengurusan} Malaysia
\newline
\begin{table}[]
\begin{tabular}{lll} \hline


Keyword & Count & Weightage \\ \hline
1. rekod & 2 & 1\\
2. objektif & 11 & 1\\
3. dasar & 20 & 1\\
4. kompeten & 11 & 0.7\\
5. kawalan & 31 & 0.7\\
6. pengurusan & 122 & 0.7\\
7. komunikasi & 18 & 0.5\\
8. komitmen & 7 & 0.5\\
9. prestasi & 22 & 0.5\\
10. ukur & 25 & 0.3\\
11. tanggungjawab & 17 & 0.3\\
12. senarai & 2 & 0.3\\
13. pelanggan & 55 & 0.2\\
14. ketakakuran & 14 & 0.2\\
15. penambahbaikan & 18 & 0.2\\
\end{tabular}
\caption{Keyword Tally}
\label{table:1}
\end{table}
Doc. Index: High (201.9) \\
Model: \\
rekod\hspace{5mm}\textasciitilde\hspace{5mm}1(objektif) + 1(dasar) + 0.7(kompeten) + 0.7(kawalan) + \\
\hspace{5mm}0.7(pengurusan) + 0.5(komunikasi) + 0.5(komitmen) + 0.5(prestasi) +\\
\hspace{5mm}0.3(ukur) + 0.3(tanggungjawab) + 0.3(senarai) + 0.2(pelanggan) +\\
\hspace{5mm}0.2(ketakakuran) + 0.2(penambahbaikan) + 0.2()
\end{document}
