% Options for packages loaded elsewhere
\PassOptionsToPackage{unicode}{hyperref}
\PassOptionsToPackage{hyphens}{url}
%
\documentclass[
]{article}
\usepackage{lmodern}
\usepackage{amssymb,amsmath}
\usepackage{ifxetex,ifluatex}
\ifnum 0\ifxetex 1\fi\ifluatex 1\fi=0 % if pdftex
  \usepackage[T1]{fontenc}
  \usepackage[utf8]{inputenc}
  \usepackage{textcomp} % provide euro and other symbols
\else % if luatex or xetex
  \usepackage{unicode-math}
  \defaultfontfeatures{Scale=MatchLowercase}
  \defaultfontfeatures[\rmfamily]{Ligatures=TeX,Scale=1}
\fi
% Use upquote if available, for straight quotes in verbatim environments
\IfFileExists{upquote.sty}{\usepackage{upquote}}{}
\IfFileExists{microtype.sty}{% use microtype if available
  \usepackage[]{microtype}
  \UseMicrotypeSet[protrusion]{basicmath} % disable protrusion for tt fonts
}{}
\makeatletter
\@ifundefined{KOMAClassName}{% if non-KOMA class
  \IfFileExists{parskip.sty}{%
    \usepackage{parskip}
  }{% else
    \setlength{\parindent}{0pt}
    \setlength{\parskip}{6pt plus 2pt minus 1pt}}
}{% if KOMA class
  \KOMAoptions{parskip=half}}
\makeatother
\usepackage{xcolor}
\IfFileExists{xurl.sty}{\usepackage{xurl}}{} % add URL line breaks if available
\IfFileExists{bookmark.sty}{\usepackage{bookmark}}{\usepackage{hyperref}}
\hypersetup{
  hidelinks,
  pdfcreator={LaTeX via pandoc}}
\urlstyle{same} % disable monospaced font for URLs
\usepackage[margin=1in]{geometry}
\usepackage{graphicx,grffile}
\makeatletter
\def\maxwidth{\ifdim\Gin@nat@width>\linewidth\linewidth\else\Gin@nat@width\fi}
\def\maxheight{\ifdim\Gin@nat@height>\textheight\textheight\else\Gin@nat@height\fi}
\makeatother
% Scale images if necessary, so that they will not overflow the page
% margins by default, and it is still possible to overwrite the defaults
% using explicit options in \includegraphics[width, height, ...]{}
\setkeys{Gin}{width=\maxwidth,height=\maxheight,keepaspectratio}
% Set default figure placement to htbp
\makeatletter
\def\fps@figure{htbp}
\makeatother
\setlength{\emergencystretch}{3em} % prevent overfull lines
\providecommand{\tightlist}{%
  \setlength{\itemsep}{0pt}\setlength{\parskip}{0pt}}
\setcounter{secnumdepth}{-\maxdimen} % remove section numbering

\author{}
\date{\vspace{-2.5em}}

\begin{document}

MALAYSIAN MS ISO 9001:2015 (BM)

STANDARD

Sistem pengurusan kualiti - Keperluan

(Semakan kedua)

(ISO 9001:2015, IDT)

(Diterbitkan oleh Jabatan Standard Malaysia pada tahun 2017)

ICS: 03.120.

JABATAN STANDARD MALAYSIA

PEMBANGUNAN MALAYSIAN STANDARD

Jabatan Standard Malaysia (STANDARDS MALAYSIA) ialah badan standard dan
akreditasi kebangsaan.

Fungsi utama Jabatan Standard Malaysia adalah untuk merangsang dan
menggalakkan standard, penstandardan dan akreditasi sebagai cara bagi
memajukan ekonomi negara, menggalakkan kecekapan dan pembangunan
industri yang bermanfaat kepada kesihatan dan keselamatan awam,
melindungi pengguna, memudahkan perdagangan dalam negeri dan
antarabangsa serta melanjutkan kerjasama antarabangsa berhubung dengan
standard dan penstandardan.

Malaysian Standard (MS) dibangunkan melalui sepersetujuan
jawatankuasa-jawatankuasa yang dianggotai oleh perwakilan yang seimbang
daripada pengeluar, pengguna dan pihak lain yang kepentingannya relevan,
sebagaimana yang sesuai dengan perkara yang sedang diusahakan. Malaysian
Standard adalah sejajar atau diterima guna daripada standard
antarabangsa, seboleh mungkin. Kelulusan sesuatu standard sebagai
Malaysian Standard ditentukan oleh Akta Standard Malaysia 1996 {[}Akta
549{]}. Malaysian Standard dikaji semula secara berkala. Penggunaan
Malaysian Standard adalah secara sukarela, melainkan diwajibkan oleh
pihak berkuasa yang mengawal selia melalui peraturan, undang-undang
kecil tempatan atau apa-apa cara lain yang serupa.

Untuk tujuan Malaysian Standard, definisi-definisi berikut diguna pakai:

Semakan: Proses di mana Malaysian Standard yang sedia ada dikaji semula
dan dikemaskini yang menjurus kepada penerbitan edisi baharu Malaysian
Standard.

MS yang disahkan: Malaysian Standard yang telah dikaji semula oleh
jawatankuasa yang bertanggungjawab dan mengesahkan bahawa kandungannya
adalah terkini.

Pindaan: Proses di mana peruntukan-peruntukan dalam Malaysian Standard
sedia ada diubah. Perubahan-perubahan dinyatakan dalam halaman pindaan
yang dimasukkan ke dalam Malaysian Standard sedia ada. Pindaan-pindaan
boleh dalam bentuk teknikal atau editorial.

Corrigendum teknikal: Cetakan semula yang telah dibetulkan bagi edisi
terkini yang dikeluarkan untuk membuat pembetulan kepada kesilapan
teknikal atau kekeliruan dalam Malaysian Standard yang diwujudkan dengan
tidak sengaja semasa mendraf atau percetakan yang menyebabkan penggunaan
Malaysian Standard yang tidak betul atau tidak selamat.

NOTA: Corrigendum teknikal bukan untuk membetulkan kesilapan yang boleh
dianggap mendatangkan akibat semasa penggunaan Malaysian Standard,
sebagai contoh kesilapan kecil percetakan.

Jabatan Standard Malaysia melantik SIRIM Berhad sebagai ejen bagi
membangunkan Malaysian Standard. Jabatan itu juga melantik SIRIM Berhad
sebagai ejen pengedaran dan penjualan Malaysian Standard.

Untuk maklumat lanjut berkaitan dengan Malaysian Standard, sila hubungi:

Jabatan Standard Malaysia ATAU SIRIM Berhad Kementerian Sains, Teknologi
dan Inovasi (No.~Syarikat 367474-V) Aras 1 \& 2, Blok 2300, Century
Square 1, Persiaran Dato' Menteri Jalan Usahawan Seksyen 2, Peti Surat
7035, 63000 Cyberjaya 40700 Shah Alam Selangor Darul Ehsan Selangor
Darul Ehsan MALAYSIA MALAYSIA

Tel.: 60 3 8318 0002 Tel.: 60 3 5544 6000 Faks: 60 3 8319 3131 Faks: 60
3 5510 8095 \url{http://www.jsm.gov.my} \url{http://www.sirim.my} E-mel:
\href{mailto:central@jsm.gov.my}{\nolinkurl{central@jsm.gov.my}} E-mel:
\href{mailto:msonline@sirim.my}{\nolinkurl{msonline@sirim.my}}

Perwakilan jawatankuasa

Jawatankuasa Standard Perindustrian mengenai Pengurusan Kualiti dan
Penentuan Kualiti (ISC Y) yang di bawah kuasanya Malaysian Standard ini
diterima guna dianggotai oleh wakil daripada organisasi yang berikut:

Biro Pengawalan Farmaseutikal Kebangsaan Dewan Perdagangan dan Industri
Antarabangsa Malaysia Institut Jurutera Malaysia Institut Kimia Malaysia
Institut Penyelidikan Sains dan Teknologi Pertahanan Institute of
Quality Malaysia Jabatan Standard Malaysia Lembaga Pembangunan Industri
Pembinaan Malaysia Perbadanan Produktiviti Malaysia Persatuan Elektrik
dan Elektronik Malaysia Persatuan Pengguna-Pengguna Standard Malaysia
Persekutuan Pekilang-Pekilang Malaysia Pihak Berkuasa Peranti Perubatan,
Kementerian Kesihatan Malaysia SIRIM Berhad SIRIM Berhad (National
Metrology Institute of Malaysia) SIRIM Berhad (Sekretariat) SIRIM QAS
International Sdn Bhd Universiti Utara Malaysia

Jawatankuasa Teknikal mengenai Pengurusan Kualiti dan Penentuan Kualiti
(TC2) mengenai Sistem Kualiti yang mengesyorkan penerimagunaan Standard
ISO dianggotai oleh wakil daripada organisasi yang berikut:

Dewan Perdagangan dan Industri Antarabangsa Malaysia Gabungan
Pembekal-pembekal Perkhidmatan Malaysia Institut Penyelidikan Sains dan
Teknologi Pertahanan Institute of Quality Malaysia Jabatan Standard
Malaysia Lembaga Pembangunan Industri Pembinaan Malaysia SIRIM Berhad
(Sekretariat) SIRIM QAS International Sdn Bhd Unit Pemodenan Tadbiran
dan Perancangan Pengurusan Malaysia Universiti Malaya Universiti Utara
Malaysia

Kumpulan Kerja mengenai Terjemahan MS ISO 9001:2015 dianggotai oleh
wakil daripada organisasi yang berikut:

Dewan Bahasa dan Pustaka Malaysia Institut Terjemahan \& Buku Malaysia
Institute of Quality Malaysia Jabatan Standard Malaysia Perbadanan
Produktiviti Malaysia SIRIM Berhad (Sekretariat) SIRIM QAS International
Sdn Bhd Unit Pemodenan Tadbiran dan Perancangan Pengurusan Malaysia
Universiti Utara Malaysia

Prakata kebangsaan

Penerimagunaan Standard ISO sebagai Malaysian Standard telah disyorkan
oleh Jawatankuasa Teknikal mengenai Pengurusan Kualiti dan Penentuan
Kualiti (TC2) mengenai Sistem Kualiti di bawah kuasa Jawatankuasa
Standard Perindustrian mengenai Pengurusan Kualiti dan Penentuan
Kualiti.

Malaysian Standard ini serupa dengan ISO 9001:20 15 , Quality Management
Systems - Requirements, diterbitkan oleh International Organization for
Standardization (ISO). Walau bagaimanapun, bagi maksud Malaysian
Standard ini, perkara-perkara yang berikut diguna pakai:

\begin{enumerate}
\def\labelenumi{\alph{enumi})}
\item
  dalam teks sumber, ``Standard Antarabangsa ini'' hendaklah dibaca
  ``Malaysian Standard ini'';
\item
  tanda koma yang digunakan sebagai tanda perpuluhan (jika ada),
  hendaklah dibaca sebagai tanda noktah; dan
\item
  rujukan Standard Antarabangsa hendaklah digantikan dengan Malaysian
  Standard sepadan seperti yang berikut:
\end{enumerate}

Rujukan Standard Antarabangsa Malaysian Standard sepadan

ISO 9000:2015, Quality management

systems - Fundamentals and vocabulary

MS ISO 9000:2015, Quality management

systems - Fundamentals and vocabulary

Malaysian Standard ini membatalkan dan menggantikan MS ISO 9001:2008,
Quality management systems - Requirements (First revision).

Versi bahasa Malaysia ini ialah terjemahan daripada versi asal dalam
bahasa Inggeris, iaitu MS ISO 9001:20 15 , Quality management systems -
Requirements. Jika terdapat apa-apa pertikaian semasa penggunaan
standard ini, versi bahasa Inggeris mengatasi versi ini.

Pematuhan Malaysian Standard tidak dengan sendirinya memberikan
kekebalan daripada obligasi undang-undang.

NOTA. IDT pada kulit depan menunjukkan standard yang serupa, iaitu satu
standard yang kandungan

teknikal, struktur, perkataan (atau terjemahan yang serupa) Malaysian
Standard adalah benar-benar

sama seperti dalam Standard Antarabangsa atau serupa dari segi kandungan
teknikal dan struktur walaupun ia mungkin mengandungi perubahan
editorial yang minimum seperti yang dinyatakan dalam 4.2 ISO/IEC Guide
21-1.

iv

Prakata

International Organization for Standardization (ISO) atau Pertubuhan
Penstandardan Antarabangsa ialah persekutuan badan standard kebangsaan
(badan anggota ISO) di seluruh dunia. Kerja menyediakan Standard
Antarabangsa lazimnya dilakukan melalui jawatankuasa teknikal ISO.
Setiap badan anggota yang berminat mengenai sesuatu perkara, yang untuk
maksud itu suatu jawatankuasa teknikal telah diwujudkan, berhak diwakili
dalam jawatankuasa itu. Pertubuhan antarabangsa kerajaan dan bukan
kerajaan, melalui hubungan dengan ISO, juga mengambil bahagian dalam
kerja ini. ISO bekerjasama rapat dengan International Electrotechnical
Commission (IEC) atau Suruhanjaya Elektroteknikal Antarabangsa dalam
segala perkara mengenai perstandardan elektroteknikal.

Prosedur yang digunakan dalam pembangunan dokumen ini dan yang
dimaksudkan untuk penyelenggaraan seterusnya diperihalkan dalam ISO/IEC
Directives, Part 1. Khususnya, kriteria kelulusan berbeza diperlukan
untuk jenis dokumen ISO berbeza sepatutnya diambil maklum. Standard
Antarabangsa disediakan selaras dengan peraturan yang diberikan dalam
ISO/IEC Directives, Part 2 (lihat
\href{http://www.iso.org/directives}{http://www.iso.org/directives).}.)

Perhatian perlu diberikan terhadap kemungkinan terdapat beberapa unsur
dalam dokumen ini boleh menjadi hal perkara hak paten. ISO tidak boleh
dipertanggungjawabkan untuk mengenal pasti apa-apa atau semua hak paten
itu. Perincian mana-mana hak paten yang dikenal pasti semasa pembangunan
dokumen akan dimasukkan ke dalam Pengenalan dan/atau senarai
perisytiharan paten ISO yang diterima (lihat
\href{http://www.iso.org/patents}{http://www.iso.org/patents).}.)

Apa-apa nama dagang yang digunakan dalam dokumen ini ialah maklumat yang
diberikan untuk kemudahan pengguna dan tidak menjadi sebahagian daripada
pengendorsan.

Bagi penjelasan makna istilah khusus ISO dan ungkapan berkaitan dengan
penilaian keakuran, serta maklumat tentang pematuhan ISO kepada prinsip
Pertubuhan Perdagangan Antarabangsa (WTO) dalam Technical Barriers to
Trade (TBT), lihat URL: \url{http://www.iso.org/iso/foreword.html.}

Jawatankuasa yang bertanggungjawab tentang dokumen ini ialah Technical
Committee ISO/TC 176, Quality management and quality assurance,
Subcommittee SC 2, Quality systems.

Edisi kelima ini membatalkan dan menggantikan edisi keempat (ISO
9001:2008), yang telah disemak semula dari segi teknikal, melalui
penerimagunaan kajian semula urutan klausa dan penyesuaian kajian semula
prinsip pengurusan kualiti dan konsep baharu. Edisi ini juga membatalkan
dan menggantikan Technical Corrigendum ISO 9001:2008/Cor.1:2009.

\hypertarget{pengenalan}{%
\subsubsection{Pengenalan}\label{pengenalan}}

\hypertarget{am}{%
\subsubsection{0.1 Am}\label{am}}

Penerimagunaan sistem pengurusan kualiti ialah suatu keputusan strategik
untuk sesebuah organisasi yang boleh membantu untuk meningkatkan
prestasi secara menyeluruh dan menyediakan asas yang kukuh bagi
inisiatif pembangunan mampan.

Faedah yang mungkin kepada organisasi dalam melaksanakan sistem
pengurusan kualiti berdasarkan Standard Antarabangsa ini adalah:

\begin{enumerate}
\def\labelenumi{\alph{enumi})}
\item
  keupayaan untuk menyediakan secara tekal produk dan perkhidmatan yang
  memenuhi keperluan pelanggan, serta keperluan berkanun dan peraturan
  yang diguna pakai;
\item
  memudahkan peluang untuk meningkatkan kepuasan pelanggan;
\item
  menyatakan risiko dan peluang yang berkaitan dengan konteks dan
  matlamatnya;
\item
  keupayaan untuk menunjukkan keakuran terhadap keperluan sistem
  pengurusan kualiti yang ditetapkan.
\end{enumerate}

Standard Antarabangsa ini boleh digunakan oleh pihak dalaman dan luaran.

Standard Antarabangsa ini tidak bertujuan untuk menggambarkan keperluan
bagi:

\begin{itemize}
\tightlist
\item
  keseragaman dalam struktur sistem pengurusan kualiti yang berbeza;
\item
  keselarasan pendokumenan dengan struktur klausa Standard Antarabangsa
  ini;
\item
  penggunaan istilah khusus Standard Antarabangsa ini dalam organisasi.
\end{itemize}

Keperluan sistem pengurusan kualiti yang dinyatakan dalam Standard
Antarabangsa ini merupakan pelengkap kepada keperluan untuk produk dan
perkhidmatan.

Standard Antarabangsa ini menggunakan pendekatan proses, yang
menggabungkan kitaran PDCA (Plan-Do-Check-Act -
Rancang-Lakukan-Semak-Bertindak) dan pemikiran berasaskan risiko.

Pendekatan proses membolehkan sesuatu organisasi merancang proses dan
saling tindaknya.

Kitaran PDCA membolehkan sesuatu organisasi memastikan yang prosesnya
disumberkan dan diuruskan secukupnya, dan peluang untuk penambahbaikan
ditentukan dan diambil tindakan.

Pemikiran berasaskan risiko membolehkan sesuatu organisasi menentukan
faktor yang boleh menyebabkan proses dan sistem pengurusan kualitinya
menyimpang daripada hasil yang dirancang, menyediakan kawalan pencegahan
untuk meminimumkan kesan negatif dan untuk menggunakan secara maksimum
peluang yang wujud (lihat Klausa A.4).

vi

Memenuhi kehendak serta menangani keperluan dan jangkaan masa depan
secara tekal menjadi cabaran kepada organisasi dalam persekitaran yang
semakin dinamik dan kompleks. Bagi mencapai matlamat ini, organisasi
mungkin perlu mengguna terima pelbagai bentuk penambahbaikan sebagai
tambahan kepada pembetulan dan penambahbaikan berterusan, seperti
perubahan besar, inovasi dan penyusunan semula organisasi.

Dalam Standard Antarabangsa ini, bentuk kata kerja yang berikut
digunakan:

\begin{itemize}
\tightlist
\item
  ``hendaklah'' menandakan keperluan;
\item
  ``harus'' menandakan cadangan;
\item
  ``bolehlah'' menandakan kebenaran;
\item
  ``boleh'' menandakan suatu kemungkinan atau keupayaan.
\end{itemize}

Maklumat yang bertandakan ``NOTA'' adalah sebagai panduan bagi memahami
atau memperjelaskan keperluan yang berkaitan.

\hypertarget{prinsip-pengurusan-kualiti}{%
\subsubsection{0.2 Prinsip pengurusan
kualiti}\label{prinsip-pengurusan-kualiti}}

Standard Antarabangsa ini berdasarkan prinsip pengurusan kualiti yang
diperihalkan dalam ISO 9000. Pemerihalan termasuk pernyataan tentang
setiap prinsip, rasional tentang sebab prinsip itu penting bagi
organisasi, beberapa contoh manfaat yang berkaitan dengan prinsip itu
dan contoh tindakan biasa untuk menambah baik prestasi organisasi
apabila menggunakan prinsip itu.

Prinsip pengurusan kualiti adalah seperti yang berikut:

\begin{itemize}
\tightlist
\item
  fokus pelanggan;
\item
  kepimpinan;
\end{itemize}

− pelibatan orang;

− pendekatan proses;

− penambahbaikan;

− pembuatan keputusan berasaskan bukti;

− pengurusan perhubungan.

\hypertarget{pendekatan-proses}{%
\subsubsection{0.3 Pendekatan proses}\label{pendekatan-proses}}

0.3.1 Am

Standard Antarabangsa ini menggalakkan penerimagunaan pendekatan proses
apabila membangunkan, melaksanakan dan menambah baik keberkesanan sistem
pengurusan kualiti, bagi meningkatkan kepuasan pelanggan dengan cara
memenuhi keperluan pelanggan. Keperluan khusus yang dianggap penting
dalam penerimagunaan pendekatan proses dimasukkan dalam 4 .4.

vii

Memahami dan mengurus proses saling berkait sebagai suatu sistem adalah
menyumbang kepada keberkesanan dan kecekapan organisasi dalam mencapai
hasil yang dimaksudkan. Pendekatan ini membolehkan organisasi mengawal
saling hubungan dan saling kebergantungan antara proses sistem itu,
supaya prestasi keseluruhan organisasi boleh dipertingkatkan.

Pendekatan proses melibatkan pentakrifan dan pengurusan proses yang
sistematik, dan saling tindaknya, untuk mencapai hasil yang dimaksudkan
selaras dengan dasar kualiti dan hala tuju strategik organisasi.
Pengurusan proses dan sistem secara keseluruhan boleh dicapai dengan
menggunakan kitaran PDCA (lihat 0.3.2) dengan tumpuan keseluruhan
terhadap pemikiran berasaskan risiko (lihat 0.3.3) yang bertujuan untuk
memanfaatkan peluang dan mencegah keputusan yang tidak diingini.

Pemakaian pendekatan proses dalam sistem pengurusan kualiti membolehkan:

\begin{enumerate}
\def\labelenumi{\alph{enumi})}
\item
  pemahaman dan ketekalan dalam memenuhi keperluan;
\item
  pertimbangan proses dari segi nilai ditambah;
\item
  pencapaian prestasi proses yang berkesan;
\item
  penambahbaikan proses berdasarkan penilaian data dan maklumat.
\end{enumerate}

Rajah 1 memberi gambaran skematik proses dan menunjukkan saling tindak
unsurnya. Pemantauan dan pengukuran titik semak, yang diperlukan untuk
kawalan, adalah khusus untuk setiap proses dan berubah, bergantung pada
risiko yang berkaitan.

0.3. 3 Pemikiran berasaskan risiko

Pemikiran berasaskan risiko (lihat Klausa A.4) adalah penting bagi
mencapai sistem pengurusan kualiti yang berkesan. Konsep pemikiran
berasaskan risiko yang tersirat dalam edisi terdahulu Standard
Antarabangsa ini termasuk, sebagai contoh, menjalankan tindakan
pencegahan untuk menghapuskan ketakakuran yang mungkin berlaku,
menganalisis apa-apa ketakakuran yang berlaku, dan mengambil tindakan
untuk mencegah berulangnya ketakakuran, sesuai dengan kesan daripada
ketakakuran itu.

Sesuatu organisasi perlu merancang dan melaksanakan tindakan untuk
menangani risiko dan peluang supaya akur dengan keperluan Standard
Antarabangsa ini. Penanganan kedua-dua risiko dan peluang dapat
menetapkan asas bagi meningkatkan keberkesanan sistem pengurusan
kualiti, mencapai hasil yang lebih baik dan mencegah kesan negatif.

Peluang boleh muncul sebagai hasil daripada keadaan yang sesuai untuk
mencapai hasil yang dimaksudkan, sebagai contoh, suatu keadaan yang
membolehkan organisasi menarik pelanggan, membangunkan produk dan
perkhidmatan baharu, mengurangkan pembaziran atau menambah baik
produktiviti. Tindakan untuk menangani peluang boleh juga termasuk
pertimbangan terhadap risiko yang berkaitan. Risiko ialah kesan
ketakpastian, dan apa-apa ketakpastian itu boleh mempunyai kesan positif
atau negatif. Sisihan positif yang timbul daripada risiko boleh memberi
peluang, tetapi bukan semua kesan positif risiko menghasilkan peluang.

\hypertarget{hubungan-dengan-standard-sistem-pengurusan-yang-lain}{%
\subsubsection{0.4 Hubungan dengan standard sistem pengurusan yang
lain}\label{hubungan-dengan-standard-sistem-pengurusan-yang-lain}}

Standard Antarabangsa ini menggunakan rangka kerja yang dibangunkan oleh
ISO untuk menambah baik keselarasan antara Standard Antarabangsa bagi
sistem pengurusannya (lihat Klausa A.1).

Standard Antarabangsa ini membolehkan sesuatu organisasi menggunakan
pendekatan proses, ditambah pula dengan kitaran PDCA dan pemikiran
berasaskan risiko, untuk menyelaraskan atau mengintegrasikan sistem
pengurusan kualitinya dengan keperluan standard sistem pengurusan yang
lain.

Standard Antarabangsa ini berkait dengan ISO 9000 dan ISO 9004 seperti
yang berikut:

\begin{itemize}
\tightlist
\item
  ISO 9000 Quality management systems - Fundamentals and vocabulary
  menyediakan latar belakang penting untuk pemahaman dan pelaksanaan
  yang lebih sesuai bagi Standard Antarabangsa ini;
\item
  ISO 9004 Managing for the sustained success of an organization - A
  quality management approach menyediakan panduan bagi organisasi yang
  memilih untuk maju melangkaui keperluan Standard Antarabangsa ini.
\end{itemize}

Lampiran B menyediakan perincian tentang Standard Antarabangsa
pengurusan kualiti dan sistem pengurusan kualiti lain yang telah
dibangunkan oleh ISO/TC 176.

x

Standard Antarabangsa ini tidak termasuk keperluan khusus untuk sistem
pengurusan lain, seperti pengurusan alam sekitar, pengurusan keselamatan
dan kesihatan pekerjaan, atau pengurusan kewangan.

Standard sistem pengurusan kualiti sektor khusus yang berdasarkan
keperluan Standard Antarabangsa ini telah dibangunkan untuk beberapa
sektor. Sesetengah Standard ini menetapkan keperluan sistem pengurusan
kualiti tambahan, manakala yang lainnya terhad kepada menyediakan
panduan terhadap pemakaian Standard Antarabangsa ini dalam sektor
tertentu.

Matriks yang menunjukkan hubung kait antara klausa dalam edisi Standard
Antarabangsa ini dengan edisi sebelumnya (ISO 9001:2008) boleh didapati
dalam tapak sesawang akses terbuka ISO/TC 176/SC 2 di:
\url{http://www.iso.org/tc176/sc02/public.}

\hypertarget{sistem-pengurusan-kualiti---keperluan}{%
\subsubsection{Sistem pengurusan kualiti -
Keperluan}\label{sistem-pengurusan-kualiti---keperluan}}

\hypertarget{skop}{%
\subsubsection{1 Skop}\label{skop}}

Standard Antarabangsa ini menetapkan keperluan bagi satu sistem
pengurusan kualiti apabila sesuatu organisasi:

\begin{enumerate}
\def\labelenumi{\alph{enumi})}
\item
  perlu menunjukkan keupayaannya menyediakan secara tekal produk dan
  perkhidmatan yang memenuhi keperluan pelanggan serta keperluan
  berkanun dan peraturan yang diguna pakai, dan
\item
  mempunyai tujuan untuk meningkatkan kepuasan pelanggan melalui
  pemakaian sistem yang berkesan, termasuk proses penambahbaikan untuk
  sistem itu dan jaminan keakuran terhadap keperluan pelanggan serta
  keperluan berkanun dan peraturan yang diguna pakai.
\end{enumerate}

Semua keperluan Standard Antarabangsa ini adalah generik dan bermaksud
untuk diguna pakai oleh mana-mana organisasi, tanpa mengambil kira jenis
atau saiz, atau produk dan perkhidmatan yang disediakan.

NOTA 1. Dalam Standard Antarabangsa ini, istilah ``produk'' atau
``perkhidmatan'' hanya diguna pakai kepada produk dan perkhidmatan yang
dimaksudkan untuk, atau diperlukan oleh, pelanggan.

NOTA 2. Keperluan berkanun dan peraturan boleh diungkapkan sebagai
keperluan undang-undang.

\hypertarget{rujukan-normatif}{%
\subsubsection{2 Rujukan normatif}\label{rujukan-normatif}}

Keseluruhan atau sebahagian daripada dokumen yang berikut, dirujuk
secara normatif dalam dokumen ini dan sangat diperlukan untuk
penggunaannya. Bagi rujukan bertarikh, hanya edisi yang disebut diguna
pakai. Bagi rujukan tidak bertarikh, edisi terkini dokumen yang dirujuk
(termasuk sebarang pindaan) diguna pakai.

MS ISO 9000 (BM), Sistem pengurusan kualiti - Asas dan kosa kata

\hypertarget{istilah-dan-takrifan}{%
\subsubsection{3 Istilah dan takrifan}\label{istilah-dan-takrifan}}

Bagi tujuan dokumen ini, istilah dan takrifan yang diberikan dalam ISO
9000:2015 adalah diguna pakai.

\hypertarget{konteks-organisasi}{%
\subsubsection{4 Konteks organisasi}\label{konteks-organisasi}}

\hypertarget{memahami-organisasi-dan-konteksnya}{%
\paragraph{4.1 Memahami organisasi dan
konteksnya}\label{memahami-organisasi-dan-konteksnya}}

Organisasi hendaklah menentukan isu luaran dan dalaman yang relevan
dengan tujuan dan haluan strategiknya serta yang memberi kesan kepada
kebolehannya untuk mencapai hasil yang dimaksudkan daripada sistem
pengurusan kualitinya.

Organisasi hendaklah memantau dan mengkaji semula maklumat tentang isu
luaran dan dalaman ini.

NOTA 1. Isu boleh termasuk faktor atau keadaan positif dan negatif untuk
dipertimbangkan.

NOTA 2. Memahami konteks luaran boleh dipermudahkan dengan
mempertimbangkan isu yang timbul daripada persekitaran undang-undang,
teknologi, daya saing, pasaran, budaya, sosial dan ekonomi, sama ada
yang berupa antarabangsa, dalam negara, serantau atau tempatan.

NOTA 3. Memahami konteks dalaman boleh dipermudahkan dengan
mempertimbangkan isu yang berkaitan dengan nilai, budaya, pengetahuan
dan prestasi organisasi.

\hypertarget{memahami-keperluan-dan-jangkaan-pihak-yang-berkepentingan}{%
\paragraph{4.2 Memahami keperluan dan jangkaan pihak yang
berkepentingan}\label{memahami-keperluan-dan-jangkaan-pihak-yang-berkepentingan}}

Disebabkan oleh kesan atau kesan yang mungkin wujud terhadap keupayaan
organisasi untuk menyediakan secara tekal produk dan perkhidmatan yang
memenuhi keperluan pelanggan, serta keperluan berkanun dan peraturan
yang diguna pakai, organisasi hendaklah menentukan:

\begin{enumerate}
\def\labelenumi{\alph{enumi})}
\item
  pihak yang berkepentingan yang relevan dengan sistem pengurusan
  kualiti;
\item
  keperluan pihak yang berkepentingan ini yang relevan dengan sistem
  pengurusan kualiti.
\end{enumerate}

Organisasi hendaklah memantau dan mengkaji semula maklumat tentang pihak
yang berkepentingan ini dan keperluan mereka yang relevan.

\hypertarget{menentukan-skop-sistem-pengurusan-kualiti}{%
\paragraph{4.3 Menentukan skop sistem pengurusan
kualiti}\label{menentukan-skop-sistem-pengurusan-kualiti}}

Organisasi hendaklah menentukan sempadan dan kesesuaian sistem
pengurusan kualiti untuk mewujudkan skopnya.

Apabila menentukan skop ini, organisasi hendaklah mempertimbangkan:

\begin{enumerate}
\def\labelenumi{\alph{enumi})}
\item
  isu luaran dan dalaman yang dirujuk dalam 4.1;
\item
  keperluan pihak yang berkepentingan yang relevan dan dirujuk dalam
  4.2;
\item
  produk dan perkhidmatan organisasi.
\end{enumerate}

Organisasi hendaklah mengguna pakai semua keperluan Standard
Antarabangsa ini jika ia terpakai dalam skop yang ditetapkan dalam
sistem pengurusan kualitinya.

Skop sistem pengurusan kualiti organisasi hendaklah tersedia dan
diselenggarakan sebagai maklumat didokumentasikan. Skop ini hendaklah
menyatakan jenis produk dan perkhidmatan yang diliputinya, dan
memberikan justifikasi bagi apa-apa keperluan Standard Antarabangsa ini
yang ditentukan oleh organisasi sebagai tidak terpakai untuk skop sistem
pengurusan kualitinya.

Keakuran terhadap Standard Antarabangsa ini hanya boleh diakui, jika
keperluan ditentukan sebagai tidak terpakai, tidak menjejaskan keupayaan
atau tanggungjawab organisasi bagi memastikan keakuran produk dan
perkhidmatannya dan peningkatan kepuasan pelanggan.

\hypertarget{sistem-pengurusan-kualiti-dan-prosesnya}{%
\paragraph{4.4 Sistem pengurusan kualiti dan
prosesnya}\label{sistem-pengurusan-kualiti-dan-prosesnya}}

\hypertarget{organisasi-hendaklah-mewujudkan-melaksanakan-menyelenggarakan-dan}{%
\subparagraph{4.4.1 Organisasi hendaklah mewujudkan, melaksanakan,
menyelenggarakan
dan}\label{organisasi-hendaklah-mewujudkan-melaksanakan-menyelenggarakan-dan}}

menambah baik secara berterusan sistem pengurusan kualiti, termasuk
proses yang diperlukan dan saling tindaknya, selaras dengan keperluan
Standard Antarabangsa ini.

Organisasi hendaklah menentukan proses yang diperlukan untuk sistem
pengurusan kualiti dan pemakaiannya dalam seluruh organisasi, dan
hendaklah:

\begin{enumerate}
\def\labelenumi{\alph{enumi})}
\item
  menentukan input yang diperlukan dan output yang dijangkakan daripada
  proses ini;
\item
  menentukan urutan dan saling tindak proses ini;
\item
  menentukan dan mengguna pakai kriteria dan kaedah (termasuk
  pemantauan, pengukuran dan petunjuk prestasi yang berkaitan) yang
  diperlukan bagi memastikan keberkesanan operasi dan kawalan proses
  ini;
\item
  menentukan sumber yang diperlukan untuk proses ini dan memastikan
  ketersediaannya;
\item
  menetapkan tanggungjawab dan bidang kuasa untuk proses ini;
\item
  menyatakan risiko dan peluang sebagaimana yang ditentukan selaras
  dengan keperluan dalam 6.1;
\item
  menilai proses ini dan melaksanakan apa-apa perubahan yang diperlukan
  bagi memastikan bahawa proses ini mencapai hasil yang dimaksudkan;
\item
  menambah baik proses dan sistem pengurusan kualiti.
\end{enumerate}

\hypertarget{menyelenggarakan-maklumat-didokumentasikan}{%
\subparagraph{4.4.2 menyelenggarakan maklumat
didokumentasikan}\label{menyelenggarakan-maklumat-didokumentasikan}}

4.4.2 Setakat yang perlu, organisasi hendaklah:

\begin{enumerate}
\def\labelenumi{\alph{enumi})}
\item
  menyelenggarakan maklumat didokumentasikan bagi menyokong operasi
  prosesnya;
\item
  menyimpan maklumat didokumentasikan supaya mempunyai keyakinan bahawa
  proses dijalankan seperti yang dirancang.
\end{enumerate}

\hypertarget{kepimpinan}{%
\subsubsection{5 Kepimpinan}\label{kepimpinan}}

\hypertarget{kepimpinan-dan-komitmen}{%
\paragraph{5.1 Kepimpinan dan komitmen}\label{kepimpinan-dan-komitmen}}

\hypertarget{am-1}{%
\subparagraph{5.1.1 Am}\label{am-1}}

Pengurusan atasan hendaklah menunjukkan kepimpinan dan komitmen tentang
sistem pengurusan kualiti dengan:

\begin{enumerate}
\def\labelenumi{\alph{enumi})}
\item
  mengambil kebertanggungjawaban terhadap keberkesanan sistem pengurusan
  kualiti;
\item
  memastikan dasar kualiti dan objektif kualiti diwujudkan untuk sistem
  pengurusan kualiti dan adalah bersesuaian dengan konteks dan hala tuju
  strategik organisasi;
\item
  memastikan integrasi keperluan sistem pengurusan kualiti ke dalam
  proses perniagaan organisasi;
\item
  menggalakkan penggunaan pendekatan proses dan pemikiran berasaskan
  risiko;
\item
  memastikan sumber yang diperlukan untuk sistem pengurusan kualiti
  adalah tersedia;
\item
  mengkomunikasikan pentingnya pengurusan kualiti yang berkesan dan
  keakuran terhadap keperluan sistem pengurusan kualiti;
\item
  memastikan bahawa sistem pengurusan kualiti mencapai hasil yang
  dimaksudkan;
\item
  melibatkan, mengarah dan menyokong pekerja supaya menyumbang kepada
  keberkesanan sistem pengurusan kualiti;
\item
  menggalakkan penambahbaikan;
\item
  menyokong peranan pengurusan lain yang relevan untuk menunjukkan
  kepimpinannya, seperti yang terpakai dalam bidang tanggungjawabnya.
\end{enumerate}

NOTA. Kata ``perniagaan'' dalam Standard Antarabangsa ini boleh
ditafsirkan secara meluas yang bermakna aktiviti teras kepada tujuan
kewujudan organisasi, sama ada organisasi itu berupa organisasi
kerajaan, swasta, untuk keuntungan atau bukan untuk keuntungan.

\hypertarget{fokus-kepada-pelanggan}{%
\subparagraph{5.1.2 Fokus kepada
pelanggan}\label{fokus-kepada-pelanggan}}

Pengurusan atasan hendaklah menunjukkan kepimpinan dan komitmen
berkaitan fokus kepada pelanggan dengan memastikan bahawa:

\begin{enumerate}
\def\labelenumi{\alph{enumi})}
\item
  keperluan pelanggan, serta keperluan berkanun dan peraturan yang
  terpakai ditentukan, difahami serta dipenuhi secara tekal;
\item
  risiko dan peluang yang boleh memberi kesan kepada keakuran produk dan
  perkhidmatan dan keupayaan untuk meningkatkan kepuasan pelanggan
  ditentukan dan dinyatakan;
\item
  fokus untuk meningkatkan kepuasan pelanggan dikekalkan.
\end{enumerate}

\hypertarget{dasar}{%
\paragraph{5.2 Dasar}\label{dasar}}

\hypertarget{membangunkan-dasar-kualiti}{%
\subparagraph{5.2.1 Membangunkan dasar
kualiti}\label{membangunkan-dasar-kualiti}}

Pengurusan atasan hendaklah mewujudkan, melaksanakan dan
menyelenggarakan dasar kualiti yang:

\begin{enumerate}
\def\labelenumi{\alph{enumi})}
\item
  sesuai dengan tujuan dan konteks organisasi serta menyokong hala tuju
  strategiknya;
\item
  menyediakan rangka kerja untuk menetapkan objektif kualiti;
\item
  mengandungi komitmen untuk memenuhi keperluan berkenaan;
\item
  mengandungi komitmen untuk menambah baik secara berterusan sistem
  pengurusan kualiti.
\end{enumerate}

\hypertarget{mengkomunikasikan-dasar-kualiti}{%
\subparagraph{5.2.2 Mengkomunikasikan dasar
kualiti}\label{mengkomunikasikan-dasar-kualiti}}

Dasar kualiti hendaklah:

\begin{enumerate}
\def\labelenumi{\alph{enumi})}
\item
  tersedia dan diselenggarakan sebagai maklumat didokumentasikan;
\item
  dikomunikasikan, difahami dan diguna pakai dalam organisasi;
\item
  tersedia untuk pihak yang berkepentingan yang relevan, mengikut
  kesesuaian.
\end{enumerate}

\hypertarget{peranan-tanggungjawab-dan-bidang-kuasa-organisasi}{%
\paragraph{5.3 Peranan, tanggungjawab dan bidang kuasa
organisasi}\label{peranan-tanggungjawab-dan-bidang-kuasa-organisasi}}

Pengurusan atasan hendaklah memastikan bahawa tanggungjawab dan bidang
kuasa bagi peranan yang relevan ditetapkan, dikomunikasikan dan difahami
dalam organisasi itu.

Pengurusan atasan hendaklah menetapkan tanggungjawab dan bidang kuasa
bagi:

\begin{enumerate}
\def\labelenumi{\alph{enumi})}
\item
  memastikan sistem pengurusan kualiti akur terhadap keperluan Standard
  Antarabangsa ini;
\item
  memastikan proses memberikan output yang dimaksudkan;
\item
  melaporkan prestasi sistem pengurusan kualiti dan peluang untuk
  penambahbaikan (lihat 10.1), khususnya kepada pengurusan atasan;
\item
  memastikan penggalakan bagi fokus terhadap pelanggan dalam seluruh
  organisasi;
\item
  memastikan integriti sistem pengurusan kualiti dikekalkan apabila
  perubahan sistem pengurusan kualiti dirancang dan dilaksanakan.
\end{enumerate}

\hypertarget{perancangan}{%
\subsubsection{6 Perancangan}\label{perancangan}}

\hypertarget{tindakan-menyatakan-risiko-dan-peluang}{%
\paragraph{6.1 Tindakan menyatakan risiko dan
peluang}\label{tindakan-menyatakan-risiko-dan-peluang}}

\hypertarget{apabila-merancang-untuk-sistem-pengurusan-kualiti-organisasi-hendaklah}{%
\subparagraph{6.1.1 Apabila merancang untuk sistem pengurusan kualiti,
organisasi
hendaklah}\label{apabila-merancang-untuk-sistem-pengurusan-kualiti-organisasi-hendaklah}}

mempertimbangkan isu yang disebutkan dalam 4.1 dan keperluan yang
disebutkan dalam 4. dan menentukan risiko dan peluang yang perlu
dinyatakan untuk:

\begin{enumerate}
\def\labelenumi{\alph{enumi})}
\item
  memberi jaminan bahawa sistem pengurusan kualiti boleh mencapai hasil
  yang dimaksudkan;
\item
  meningkatkan kesan yang diingini;
\item
  mencegah, atau mengurangkan kesan yang tidak diingini;
\item
  mencapai penambahbaikan.
\end{enumerate}

\hypertarget{organisasi-hendaklah-merancang}{%
\subparagraph{6.1.2 Organisasi hendaklah
merancang:}\label{organisasi-hendaklah-merancang}}

\begin{enumerate}
\def\labelenumi{\alph{enumi})}
\item
  tindakan menyatakan risiko dan peluang;
\item
  cara untuk:
\end{enumerate}

\begin{enumerate}
\def\labelenumi{\arabic{enumi})}
\item
  mengintegrasikan, dan melaksanakan tindakan itu ke dalam proses sistem
  pengurusan kualiti (lihat 4.4);
\item
  menilai keberkesanan tindakan ini.
\end{enumerate}

Tindakan yang diambil untuk menyatakan risiko dan peluang hendaklah
setimpal dengan kesan yang mungkin wujud terhadap keakuran produk dan
perkhidmatan.

NOTA 1. Pilihan untuk menyatakan risiko boleh termasuk mengelakkan
risiko, mengambil risiko untuk mengejar peluang, menghapuskan punca
risiko, mengubah kemungkinan atau akibat, berkongsi risiko, atau
mengekalkan risiko berdasarkan keputusan bermaklumat.

NOTA 2. Peluang boleh membawa kepada penerimagunaan amalan baharu,
pelancaran produk baharu, pembukaan pasaran baharu, menangani pelanggan
baharu, membina perkongsian, menggunakan teknologi baharu, dan
kemungkinan lain yang diingini dan berdaya maju bagi menangani keperluan
organisasi atau pelanggannya.

\hypertarget{objektif-kualiti-dan-perancangan-untuk-mencapainya}{%
\paragraph{6.2 Objektif kualiti dan perancangan untuk
mencapainya}\label{objektif-kualiti-dan-perancangan-untuk-mencapainya}}

\hypertarget{organisasi-hendaklah-mewujudkan-objektif-kualiti-pada-fungsi-aras-dan-proses-yang-relevan-yang-diperlukan-untuk-sistem-pengurusan-kualiti}{%
\subparagraph{6.2.1 Organisasi hendaklah mewujudkan objektif kualiti
pada fungsi, aras dan proses yang relevan yang diperlukan untuk sistem
pengurusan
kualiti}\label{organisasi-hendaklah-mewujudkan-objektif-kualiti-pada-fungsi-aras-dan-proses-yang-relevan-yang-diperlukan-untuk-sistem-pengurusan-kualiti}}

Objektif kualiti hendaklah:

\begin{enumerate}
\def\labelenumi{\alph{enumi})}
\item
  tekal dengan dasar kualiti;
\item
  boleh diukur;
\item
  mengambil kira keperluan yang diterima pakai;
\item
  relevan dengan keakuran produk dan perkhidmatan, dan peningkatan
  kepuasan pelanggan;
\item
  dipantau;
\item
  dikomunikasikan;
\item
  dikemas kini mengikut kesesuaian.
\end{enumerate}

Organisasi hendaklah menyelenggara maklumat didokumentasikan tentang
objektif kualiti.

\hypertarget{apabila-merancang-cara-untuk-mencapai-objektif-kualiti-organisasi-hendaklah-menentukan}{%
\subparagraph{6.2.2 Apabila merancang cara untuk mencapai objektif
kualiti, organisasi hendaklah
menentukan:}\label{apabila-merancang-cara-untuk-mencapai-objektif-kualiti-organisasi-hendaklah-menentukan}}

\begin{enumerate}
\def\labelenumi{\alph{enumi})}
\item
  apa yang akan dilakukan;
\item
  sumber yang diperlukan;
\item
  siapa yang bertanggungjawab;
\item
  bila ia akan disiapkan;
\item
  cara hasil akan dinilai.
\end{enumerate}

\hypertarget{merancang-perubahan}{%
\paragraph{6.3 Merancang perubahan}\label{merancang-perubahan}}

Apabila organisasi menentukan keperluan untuk mengubah sistem pengurusan
kualiti, perubahan itu hendaklah dilaksanakan dengan cara yang terancang
(lihat 4.4).

Organisasi hendaklah memberi pertimbangan terhadap:

\begin{enumerate}
\def\labelenumi{\alph{enumi})}
\item
  tujuan perubahan dan kesan yang mungkin timbul;
\item
  integriti sistem pengurusan kualiti;
\item
  ketersediaan sumber;
\item
  pengagihan atau pengagihan semula tanggungjawab dan bidang kuasa.
\end{enumerate}

\hypertarget{sokongan}{%
\subsubsection{7 Sokongan}\label{sokongan}}

\hypertarget{sumber}{%
\paragraph{7.1 Sumber}\label{sumber}}

\hypertarget{am-2}{%
\subparagraph{7.1.1 Am}\label{am-2}}

Organisasi hendaklah menentukan dan menyediakan sumber yang diperlukan
bagi mewujudkan, melaksanakan, menyelenggarakan, dan menambah baik
secara berterusan sistem pengurusan kualiti.

Organisasi hendaklah memberi pertimbangan terhadap:

\begin{enumerate}
\def\labelenumi{\alph{enumi})}
\item
  keupayaan dan kekangan sumber dalaman sedia ada;
\item
  apa yang perlu diperoleh daripada penyedia luar.
\end{enumerate}

\hypertarget{modal-insan}{%
\subparagraph{7.1.2 Modal insan}\label{modal-insan}}

Organisasi hendaklah menentukan dan menyediakan modal insan yang
diperlukan untuk pelaksanaan sistem pengurusan kualiti yang berkesan dan
untuk operasi dan kawalan prosesnya.

\hypertarget{prasarana}{%
\subparagraph{7.1.3 Prasarana}\label{prasarana}}

Organisasi hendaklah menentukan, menyediakan dan menyelenggarakan
prasarana yang diperlukan untuk operasi prosesnya dan untuk mencapai
keakuran produk dan perkhidmatan.

NOTA. Prasarana boleh termasuk:

\begin{enumerate}
\def\labelenumi{\alph{enumi})}
\item
  bangunan dan utiliti yang berkaitan;
\item
  peralatan, termasuk perkakasan dan perisian;
\item
  sumber pengangkutan;
\item
  teknologi maklumat dan komunikasi.
\end{enumerate}

\hypertarget{persekitaran-untuk-operasi-proses}{%
\subparagraph{7.1.4 Persekitaran untuk operasi
proses}\label{persekitaran-untuk-operasi-proses}}

Organisasi hendaklah menentukan, menyediakan dan menyelenggarakan
persekitaran yang diperlukan untuk operasi prosesnya dan untuk mencapai
keakuran produk dan perkhidmatan.

NOTA. Persekitaran yang sesuai boleh merupakan gabungan faktor manusia
dan fizikal, seperti:

\begin{enumerate}
\def\labelenumi{\alph{enumi})}
\item
  sosial (contohnya tanpa diskriminasi, tenang, bukan konfrontasi);
\item
  psikologi (contohnya mengurangkan tekanan, mencegah keletihan diri,
  bersifat melindungi dari segi emosi);
\item
  fizikal (contohnya suhu, haba, kelembapan, cahaya, aliran udara,
  kebersihan, bunyi bising).
\end{enumerate}

Faktor ini boleh berbeza dengan ketara bergantung pada produk dan
perkhidmatan yang disediakan.

\hypertarget{sumber-pemantauan-dan-pengukuran}{%
\subparagraph{7.1.5 Sumber pemantauan dan
pengukuran}\label{sumber-pemantauan-dan-pengukuran}}

7.1.5.1 Am

Organisasi hendaklah menentukan dan menyediakan sumber yang diperlukan
untuk memastikan keputusan yang sah dan boleh dipercayai, apabila
pemantauan atau pengukuran digunakan untuk menentu sah keakuran produk
dan perkhidmatan terhadap keperluan.

Organisasi hendaklah memastikan sumber yang disediakan:

\begin{enumerate}
\def\labelenumi{\alph{enumi})}
\item
  adalah sesuai untuk jenis aktiviti pemantauan dan pengukuran tertentu
  yang dilaksanakan;
\item
  diselenggara bagi memastikan kesesuaian untuk maksudnya yang
  berterusan.
\end{enumerate}

Organisasi hendaklah menyimpan maklumat didokumentasikan yang sesuai
sebagai bukti kesesuaian untuk maksud tentang sumber pemantauan dan
pengukuran.

7.1.5.2 Kebolehkesanan pengukuran

Apabila kebolehkesanan pengukuran ialah suatu keperluan, atau dianggap
suatu perkara yang penting oleh organisasi untuk memberikan keyakinan
dalam kesahan hasil pengukuran, peralatan mengukur hendaklah:

\begin{enumerate}
\def\labelenumi{\alph{enumi})}
\item
  ditentukur atau ditentusahkan, atau kedua-duanya, pada sela waktu yang
  ditetapkan, atau sebelum digunakan, dengan standard pengukuran yang
  boleh dikesan daripada standard pengukuran antarabangsa atau
  kebangsaan; apabila standard sedemikian tidak wujud, asas yang
  digunakan bagi tentukuran atau penentusahan hendaklah disimpan sebagai
  maklumat didokumentasikan;
\item
  dikenal pasti untuk menentukan statusnya;
\item
  dilindungi daripada pelarasan, kerosakan atau kemerosotan yang akan
  mentaksahkan status tentukuran dan hasil pengukuran seterusnya.
\end{enumerate}

Organisasi hendaklah menentukan sama ada kesahan hasil pengukuran
terdahulu terjejas teruk apabila peralatan mengukur didapati tidak
sesuai untuk maksudnya yang ditetapkan, dan hendaklah mengambil tindakan
sewajarnya seperti yang diperlukan.

\hypertarget{pengetahuan-organisasi}{%
\subparagraph{7.1.6 Pengetahuan
organisasi}\label{pengetahuan-organisasi}}

Organisasi hendaklah menentukan pengetahuan yang diperlukan untuk
operasi prosesnya dan bagi mencapai keakuran produk dan perkhidmatan.

Pengetahuan ini hendaklah diselenggarakan dan tersedia setakat yang
perlu.

Apabila menangani perubahan keperluan dan trend, organisasi hendaklah
mengambil kira pengetahuan semasanya dan menentukan cara memperoleh atau
mengakses apa-apa pengetahuan tambahan dan pengetahuan dikemas kini yang
perlu ada.

NOTA 1. Pengetahuan organisasi ialah pengetahuan khusus bagi organisasi
itu; ia diperoleh melalui pengalaman. Ia merupakan maklumat yang
digunakan dan dikongsi bagi mencapai objektif organisasi.

NOTA 2. Pengetahuan organisasi boleh berdasarkan:

\begin{enumerate}
\def\labelenumi{\alph{enumi})}
\item
  sumber dalaman (contoh, harta intelek; pengetahuan yang diperoleh
  melalui pengalaman; pengajaran daripada kegagalan dan projek yang
  berjaya; perakaman dan perkongsian pengetahuan dan pengalaman yang
  tidak didokumentasikan; hasil penambahbaikan dalam proses, produk dan
  perkhidmatan);
\item
  sumber luaran (contoh, standard; akademia; persidangan; pengumpulan
  pengetahuan daripada pelanggan atau penyedia luar).
\end{enumerate}

\hypertarget{kekompetenan}{%
\paragraph{7.2 Kekompetenan}\label{kekompetenan}}

Organisasi hendaklah:

\begin{enumerate}
\def\labelenumi{\alph{enumi})}
\item
  menentukan kekompetenan yang diperlukan oleh orang yang melakukan
  kerja di bawah kawalannya memberi kesan kepada prestasi dan
  keberkesanan sistem pengurusan kualiti;
\item
  memastikan bahawa orang tersebut kompeten berdasarkan pendidikan,
  latihan, atau pengalaman yang sesuai;
\item
  mengambil tindakan, jika berkenaan, untuk memperoleh kekompetenan yang
  diperlukan, dan menilai keberkesanan tindakan yang diambil;
\item
  menyimpan maklumat didokumentasikan yang sesuai sebagai bukti
  kekompetenan.
\end{enumerate}

NOTA. Tindakan yang terpakai boleh termasuk, sebagai contoh, penyediaan
latihan, pementoran, atau penugasan semula pekerja sedia ada; atau
pengambilan atau pengkontrakan orang yang kompeten.

\hypertarget{kesedaran}{%
\paragraph{7.3 Kesedaran}\label{kesedaran}}

Organisasi hendaklah memastikan bahawa orang yang melakukan kerja di
bawah kawalannya mengetahui tentang:

\begin{enumerate}
\def\labelenumi{\alph{enumi})}
\item
  dasar kualiti;
\item
  objektif kualiti yang relevan;
\item
  sumbangan mereka kepada keberkesanan sistem pengurusan kualiti,
  termasuk manfaat prestasi yang ditambah baik;
\item
  implikasi jika tidak mengakuri keperluan sistem pengurusan kualiti.
\end{enumerate}

\hypertarget{komunikasi}{%
\paragraph{7.4 Komunikasi}\label{komunikasi}}

Organisasi hendaklah menentukan komunikasi dalaman dan luaran yang
relevan dengan sistem pengurusan kualiti, termasuk:

\begin{enumerate}
\def\labelenumi{\alph{enumi})}
\item
  perkara yang akan dikomunikasikan;
\item
  bila perlu berkomunikasi;
\item
  dengan siapa untuk berkomunikasi;
\item
  cara untuk berkomunikasi;
\item
  siapa yang berkomunikasi.
\end{enumerate}

\hypertarget{maklumat-didokumentasikan}{%
\paragraph{7.5 Maklumat
didokumentasikan}\label{maklumat-didokumentasikan}}

\hypertarget{am-4}{%
\subparagraph{7.5.1 Am}\label{am-4}}

Sistem pengurusan kualiti organisasi hendaklah termasuk:

\begin{enumerate}
\def\labelenumi{\alph{enumi})}
\item
  maklumat didokumentasikan yang diperlukan oleh Standard Antarabangsa
  ini;
\item
  maklumat didokumentasikan yang ditentukan perlu oleh organisasi bagi
  keberkesanan sistem pengurusan kualiti.
\end{enumerate}

NOTA. Tahap maklumat didokumentasikan untuk sistem pengurusan kualiti
boleh berbeza antara satu organisasi dengan yang lain disebabkan oleh:

\begin{itemize}
\tightlist
\item
  saiz organisasi dan jenis aktiviti, proses, produk dan
  perkhidmatannya;
\item
  kerumitan proses dan saling tindaknya;
\item
  kekompetenan orang.
\end{itemize}

\hypertarget{mewujudkan-dan-mengemas-kini}{%
\subparagraph{7.5.2 Mewujudkan dan mengemas
kini}\label{mewujudkan-dan-mengemas-kini}}

Apabila mewujudkan dan mengemas kini maklumat didokumentasikan,
organisasi hendaklah memastikan kesesuaian:

\begin{enumerate}
\def\labelenumi{\alph{enumi})}
\item
  pengenalpastian dan perihalan (contoh, tajuk, tarikh, pengarang, atau
  nombor rujukan);
\item
  format (contoh, bahasa, versi perisian, grafik) dan media (contoh,
  kertas, elektronik);
\item
  kajian semula dan kelulusan bagi kesesuaian dan kecukupan.
\end{enumerate}

\hypertarget{kawalan-maklumat-didokumentasikan}{%
\subparagraph{7.5.3 Kawalan maklumat
didokumentasikan}\label{kawalan-maklumat-didokumentasikan}}

7.5.3.1 Maklumat didokumentasikan yang diperlukan oleh sistem pengurusan
kualiti dan

Standard Antarabangsa ini hendaklah dikawal bagi memastikan:

\begin{enumerate}
\def\labelenumi{\alph{enumi})}
\item
  ia tersedia dan sesuai untuk digunakan, jika dan apabila diperlukan;
\item
  ia dilindungi secukupnya (contoh, daripada kehilangan kerahsiaan,
  penggunaan yang tidak betul, atau kehilangan integriti).
\end{enumerate}

7.5.3.2 Bagi mengawal maklumat didokumentasikan, organisasi hendaklah
menyatakan

aktiviti berikut, jika terpakai:

\begin{enumerate}
\def\labelenumi{\alph{enumi})}
\item
  pengedaran, akses, dapatan semula dan penggunaan;
\item
  penyimpanan dan pemeliharaan, termasuk pemeliharaan kemudahbacaan;
\item
  kawalan perubahan (contoh, kawalan versi);
\item
  penyimpanan dan pelupusan.
\end{enumerate}

Maklumat didokumentasikan yang berasal dari luar yang ditentukan penting
oleh organisasi bagi perancangan dan operasi sistem pengurusan kualiti
hendaklah dikenal pasti sewajarnya, dan dikawal.

Maklumat didokumentasikan yang disimpan sebagai bukti keakuran hendaklah
dilindungi daripada pengubahan tidak disengajakan.

NOTA. Akses boleh bermaksud suatu keputusan mengenai kebenaran untuk
melihat maklumat didokumentasikan sahaja, atau kebenaran dan kuasa untuk
melihat dan menukar maklumat didokumentasikan.

\hypertarget{operasi}{%
\subsubsection{8 Operasi}\label{operasi}}

\hypertarget{perancangan-dan-kawalan-operasi}{%
\paragraph{8.1 Perancangan dan kawalan
operasi}\label{perancangan-dan-kawalan-operasi}}

Organisasi hendaklah merancang, melaksanakan dan mengawal proses (lihat
4.4) yang diperlukan bagi memenuhi keperluan untuk penyediaan produk dan
perkhidmatan, dan bagi melaksanakan tindakan yang ditentukan dalam
Klausa 6 , dengan cara:

\begin{enumerate}
\def\labelenumi{\alph{enumi})}
\item
  menentukan keperluan untuk produk dan perkhidmatan;
\item
  mewujudkan kriteria untuk:
\end{enumerate}

\begin{enumerate}
\def\labelenumi{\arabic{enumi})}
\item
  proses;
\item
  penerimaan produk dan perkhidmatan;
\end{enumerate}

\begin{enumerate}
\def\labelenumi{\alph{enumi})}
\setcounter{enumi}{2}
\item
  menentukan sumber yang diperlukan untuk mencapai keakuran terhadap
  keperluan produk dan perkhidmatan;
\item
  melaksanakan kawalan proses selaras dengan kriteria;
\item
  menentukan, menyelenggarakan dan mengekalkan maklumat didokumentasikan
  setakat yang perlu:
\end{enumerate}

\begin{enumerate}
\def\labelenumi{\arabic{enumi})}
\item
  untuk memberi keyakinan bahawa proses telah dilaksanakan seperti yang
  dirancang;
\item
  untuk menunjukkan keakuran produk dan perkhidmatan terhadap
  keperluannya.
\end{enumerate}

Output perancangan ini hendaklah sesuai untuk operasi organisasi.

Organisasi hendaklah mengawal perubahan terancang dan mengkaji semula
kesan perubahan yang tidak diingini, dengan mengambil tindakan untuk
mengurangkan apa-apa kesan buruk, seperti yang diperlukan.

Organisasi hendaklah memastikan proses disumber luar dikawal (lihat
8.4).

\hypertarget{keperluan-untuk-produk-dan-perkhidmatan}{%
\paragraph{8.2 Keperluan untuk produk dan
perkhidmatan}\label{keperluan-untuk-produk-dan-perkhidmatan}}

\hypertarget{komunikasi-dengan-pelanggan}{%
\subparagraph{8.2.1 Komunikasi dengan
pelanggan}\label{komunikasi-dengan-pelanggan}}

Komunikasi dengan pelanggan hendaklah termasuk:

\begin{enumerate}
\def\labelenumi{\alph{enumi})}
\item
  menyediakan maklumat yang berkaitan dengan produk dan perkhidmatan;
\item
  mengendalikan pertanyaan, kontrak atau pesanan, termasuk perubahan;
\item
  mendapatkan maklum balas pelanggan berkaitan dengan produk dan
  perkhidmatan, termasuk aduan pelanggan;
\item
  mengendalikan atau mengawal harta pelanggan;
\item
  mewujudkan keperluan khusus untuk tindakan luar jangka, jika relevan.
\end{enumerate}

\hypertarget{menentukan-keperluan-untuk-produk-dan-perkhidmatan}{%
\subparagraph{8.2.2 Menentukan keperluan untuk produk dan
perkhidmatan}\label{menentukan-keperluan-untuk-produk-dan-perkhidmatan}}

Apabila menentukan keperluan untuk produk dan perkhidmatan yang akan
ditawarkan kepada pelanggan, organisasi hendaklah memastikan bahawa:

\begin{enumerate}
\def\labelenumi{\alph{enumi})}
\tightlist
\item
  keperluan untuk produk dan perkhidmatan ditetapkan, termasuk:
\end{enumerate}

\begin{enumerate}
\def\labelenumi{\arabic{enumi})}
\item
  apa-apa keperluan berkanun dan peraturan yang diguna pakai;
\item
  apa-apa yang difikirkan perlu oleh organisasi;
\end{enumerate}

\begin{enumerate}
\def\labelenumi{\alph{enumi})}
\setcounter{enumi}{1}
\tightlist
\item
  organisasi boleh memenuhi akuan untuk produk dan perkhidmatan yang
  ditawarkan.
\end{enumerate}

\hypertarget{kajian-semula-keperluan-untuk-produk-dan-perkhidmatan}{%
\subparagraph{8.2.3 Kajian semula keperluan untuk produk dan
perkhidmatan}\label{kajian-semula-keperluan-untuk-produk-dan-perkhidmatan}}

8.2.3.1 Organisasi hendaklah memastikan bahawa ia mempunyai keupayaan
untuk

memenuhi keperluan produk dan perkhidmatan yang ditawarkan kepada
pelanggan. Organisasi hendaklah menjalankan kajian semula sebelum
memberikan komitmen untuk membekalkan produk dan perkhidmatan kepada
pelanggan, termasuk:

\begin{enumerate}
\def\labelenumi{\alph{enumi})}
\item
  keperluan yang ditetapkan oleh pelanggan, termasuk keperluan hantar
  serah dan aktiviti selepas hantar serah;
\item
  keperluan yang tidak dinyatakan oleh pelanggan, tetapi adalah perlu
  bagi kegunaan yang ditetapkan atau yang dimaksudkan, jika hal itu
  diketahui;
\item
  keperluan yang ditetapkan oleh organisasi;
\item
  keperluan berkanun dan peraturan yang diguna pakai bagi produk dan
  perkhidmatan itu;
\item
  keperluan kontrak atau pesanan yang berbeza daripada yang dinyatakan
  terdahulu.
\end{enumerate}

Organisasi hendaklah memastikan bahawa keperluan kontrak atau pesanan
yang berbeza daripada yang ditetapkan terdahulu diselesaikan.

Keperluan pelanggan hendaklah disahkan oleh organisasi sebelum
penerimaan, apabila pelanggan tidak menyediakan pernyataan
didokumentasikan tentang keperluan mereka.

NOTA. Dalam sesetengah keadaan, seperti jualan melalui internet, kajian
semula secara formal adalah tidak praktikal bagi setiap pesanan.
Sebaliknya, kajian semula boleh meliputi maklumat produk yang relevan,
seperti katalog atau bahan pengiklanan.

8.2.3.2 Organisasi hendaklah menyimpan maklumat didokumentasikan,
seperti yang

berkenaan:

\begin{enumerate}
\def\labelenumi{\alph{enumi})}
\item
  tentang hasil kajian semula;
\item
  tentang apa-apa keperluan baharu untuk produk dan perkhidmatan.
\end{enumerate}

\hypertarget{perubahan-keperluan-untuk-produk-dan-perkhidmatan}{%
\subparagraph{8.2.4 Perubahan keperluan untuk produk dan
perkhidmatan}\label{perubahan-keperluan-untuk-produk-dan-perkhidmatan}}

Organisasi hendaklah memastikan bahawa maklumat didokumentasikan yang
relevan dipinda, dan bahawa orang yang berkaitan dimaklumkan tentang
keperluan yang berubah, apabila keperluan untuk produk dan perkhidmatan
diubah.

\hypertarget{reka-bentuk-dan-pembangunan-produk-dan-perkhidmatan}{%
\paragraph{8.3 Reka bentuk dan pembangunan produk dan
perkhidmatan}\label{reka-bentuk-dan-pembangunan-produk-dan-perkhidmatan}}

\hypertarget{am-5}{%
\subparagraph{8.3.1 Am}\label{am-5}}

Organisasi hendaklah mewujudkan, melaksanakan dan menyelenggarakan
proses reka bentuk dan pembangunan yang sesuai bagi memastikan produk
dan perkhidmatan seterusnya disediakan.

\hypertarget{perancangan-reka-bentuk-dan-pembangunan}{%
\subparagraph{8.3.2 Perancangan reka bentuk dan
pembangunan}\label{perancangan-reka-bentuk-dan-pembangunan}}

Dalam menentukan tahap dan kawalan untuk reka bentuk dan pembangunan,
organisasi hendaklah mengambil kira:

\begin{enumerate}
\def\labelenumi{\alph{enumi})}
\item
  keadaan, tempoh dan kerumitan aktiviti reka bentuk dan pembangunan;
\item
  tahap proses yang diperlukan, termasuk kajian semula reka bentuk dan
  pembangunan yang berkenaan;
\item
  aktiviti penentusahan dan pengesahan reka bentuk dan pembangunan yang
  diperlukan;
\item
  tanggungjawab dan kuasa yang terlibat dalam proses reka bentuk dan
  pembangunan;
\item
  keperluan sumber dalaman dan luaran bagi reka bentuk dan pembangunan
  produk dan perkhidmatan;
\item
  keperluan untuk mengawal antara muka dalam kalangan orang yang
  terlibat dalam proses reka bentuk dan pembangunan;
\item
  keperluan untuk pelibatan pelanggan dan pengguna dalam proses reka
  bentuk dan pembangunan;
\item
  keperluan untuk produk dan perkhidmatan seterusnya disediakan;
\item
  tahap kawalan yang dijangkakan untuk proses reka bentuk dan
  pembangunan oleh pelanggan dan pihak berkepentingan lain yang relevan;
\item
  maklumat didokumentasikan yang diperlukan untuk menunjukkan bahawa
  keperluan reka bentuk dan pembangunan telah dipenuhi.
\end{enumerate}

\hypertarget{input-reka-bentuk-dan-pembangunan}{%
\subparagraph{8.3.3 Input reka bentuk dan
pembangunan}\label{input-reka-bentuk-dan-pembangunan}}

Organisasi hendaklah menentukan keperluan yang penting untuk jenis
produk dan perkhidmatan tertentu yang akan direka bentuk dan
dibangunkan. Organisasi hendaklah mengambil kira:

\begin{enumerate}
\def\labelenumi{\alph{enumi})}
\item
  keperluan fungsian dan prestasi;
\item
  maklumat yang diperoleh daripada aktiviti reka bentuk dan pembangunan
  terdahulu yang serupa;
\item
  keperluan berkanun dan peraturan;
\item
  standard atau kod amalan yang organisasi telah memberikan komitmen
  untuk dilaksanakan;
\item
  akibat kegagalan yang mungkin timbul disebabkan sifat produk dan
  perkhidmatan.
\end{enumerate}

Input hendaklah, mencukupi bagi tujuan reka bentuk dan pembangunan,
lengkap dan jelas.

Reka bentuk dan pembangunan input yang bercanggah hendaklah
diselesaikan.

Organisasi hendaklah mengekalkan maklumat didokumentasikan tentang input
reka bentuk dan pembangunan.

\hypertarget{kawalan-reka-bentuk-dan-pembangunan}{%
\subparagraph{8.3.4 Kawalan reka bentuk dan
pembangunan}\label{kawalan-reka-bentuk-dan-pembangunan}}

Organisasi hendaklah mengenakan kawalan untuk proses reka bentuk dan
pembangunan bagi memastikan bahawa:

\begin{enumerate}
\def\labelenumi{\alph{enumi})}
\item
  hasil yang hendak dicapai ditetapkan;
\item
  kajian semula dijalankan bagi menilai keupayaan hasil reka bentuk dan
  pembangunan untuk memenuhi keperluan;
\item
  aktiviti penentusahan dijalankan bagi memastikan bahawa output reka
  bentuk dan pembangunan memenuhi keperluan input;
\item
  aktiviti pengesahan dijalankan bagi memastikan produk dan perkhidmatan
  yang dihasilkan memenuhi keperluan untuk pemakaian yang tertentu atau
  kegunaan yang dimaksudkan;
\item
  apa-apa tindakan yang perlu, diambil terhadap masalah yang dikenal
  pasti semasa kajian semula, atau aktiviti penentusahan dan pengesahan;
\item
  maklumat didokumentasikan tentang aktiviti ini dikekalkan.
\end{enumerate}

NOTA. Kajian semula reka bentuk dan pembangunan, penentusahan serta
pengesahan mempunyai tujuan yang berbeza. Perkara ini boleh dijalankan
secara berasingan atau dalam apa-apa gabungan yang sesuai bagi produk
dan perkhidmatan organisasi.

\hypertarget{output-reka-bentuk-dan-pembangunan}{%
\subparagraph{8.3.5 Output reka bentuk dan
pembangunan}\label{output-reka-bentuk-dan-pembangunan}}

Organisasi hendaklah memastikan bahawa output reka bentuk dan
pembangunan:

\begin{enumerate}
\def\labelenumi{\alph{enumi})}
\item
  memenuhi keperluan input;
\item
  mencukupi untuk proses seterusnya bagi penyediaan produk dan
  perkhidmatan;
\item
  merangkumi atau merujuk keperluan pemantauan dan pengukuran,
  sebagaimana yang sesuai, dan kriteria penerimaan;
\item
  menetapkan ciri-ciri penting produk dan perkhidmatan untuk tujuan yang
  dimaksudkan dan penyediaannya yang selamat dan sesuai.
\end{enumerate}

Organisasi hendaklah mengekalkan maklumat didokumentasikan tentang
output reka bentuk dan pembangunan.

\hypertarget{perubahan-reka-bentuk-dan-pembangunan}{%
\subparagraph{8.3.6 Perubahan reka bentuk dan
pembangunan}\label{perubahan-reka-bentuk-dan-pembangunan}}

Organisasi hendaklah mengenal pasti, menyemak semula dan mengawal
perubahan yang dibuat semasa, atau selepas, reka bentuk dan pembangunan
produk dan perkhidmatan, setakat yang perlu bagi memastikan tidak ada
impak yang bertentangan dengan keakuran terhadap keperluan.

Organisasi hendaklah mengekalkan maklumat didokumentasikan tentang:

\begin{enumerate}
\def\labelenumi{\alph{enumi})}
\item
  perubahan reka bentuk dan pembangunan;
\item
  hasil kajian semula;
\item
  kebenaran atas perubahan;
\item
  tindakan yang diambil untuk mencegah impak yang bertentangan.
\end{enumerate}

\hypertarget{kawalan-terhadap-proses-produk-dan-perkhidmatan-sediaan-luar}{%
\paragraph{8.4 Kawalan terhadap proses, produk dan perkhidmatan sediaan
luar}\label{kawalan-terhadap-proses-produk-dan-perkhidmatan-sediaan-luar}}

\hypertarget{am-6}{%
\subparagraph{8.4.1 Am}\label{am-6}}

Organisasi hendaklah memastikan bahawa proses, produk dan perkhidmatan
sediaan luar akur terhadap keperluan.

Organisasi hendaklah menentukan kawalan yang akan diguna pakai bagi
proses, produk dan perkhidmatan sediaan luar apabila:

\begin{enumerate}
\def\labelenumi{\alph{enumi})}
\item
  produk dan perkhidmatan daripada penyedia luar bermaksud untuk
  dijadikan sebahagian daripada produk dan perkhidmatan organisasi itu
  sendiri;
\item
  produk dan perkhidmatan disediakan secara terus kepada pelanggan oleh
  penyedia luar bagi pihak organisasi;
\item
  proses, atau sebahagian daripada proses, disediakan oleh penyedia luar
  berdasarkan keputusan yang dibuat oleh organisasi.
\end{enumerate}

Organisasi hendaklah menentukan dan mengguna pakai kriteria penilaian,
pemilihan, pemantauan prestasi, dan penilaian semula penyedia luar,
berdasarkan keupayaan mereka untuk menyediakan proses atau produk dan
perkhidmatan selaras dengan keperluan. Organisasi hendaklah mengekalkan
maklumat didokumentasikan berkaitan aktiviti ini dan apa-apa tindakan
perlu yang berpunca daripada penilaian.

\hypertarget{jenis-dan-takat-kawalan}{%
\subparagraph{8.4.2 Jenis dan takat
kawalan}\label{jenis-dan-takat-kawalan}}

Organisasi hendaklah memastikan bahawa proses, produk dan perkhidmatan
sediaan luar tidak memberi kesan bertentangan terhadap keupayaan
organisasi untuk membekalkan secara tekal produk dan perkhidmatan yang
akur dengan keperluan kepada pelanggannya.

Organisasi hendaklah:

\begin{enumerate}
\def\labelenumi{\alph{enumi})}
\item
  memastikan bahawa proses sediaan luar kekal dalam kawalan sistem
  pengurusan kualitinya;
\item
  menetapkan kawalan yang dimaksudkan untuk diguna pakai terhadap
  kedua-dua penyedia luar dan output yang terhasil;
\item
  mengambil kira:
\end{enumerate}

\begin{enumerate}
\def\labelenumi{\arabic{enumi})}
\item
  impak yang mungkin terhasil daripada proses, produk dan perkhidmatan
  sediaan luar terhadap keupayaan organisasi untuk memenuhi secara tekal
  keperluan pelanggan, serta keperluan berkanun dan peraturan yang
  terpakai;
\item
  keberkesanan kawalan yang diguna pakai oleh penyedia luar;
\end{enumerate}

\begin{enumerate}
\def\labelenumi{\alph{enumi})}
\setcounter{enumi}{3}
\tightlist
\item
  menentukan aktiviti penentusahan, atau aktiviti lain, yang perlu bagi
  memastikan bahawa proses, produk dan perkhidmatan sediaan luar
  memenuhi keperluan.
\end{enumerate}

\hypertarget{maklumat-untuk-penyedia-luar}{%
\subparagraph{8.4.3 Maklumat untuk penyedia
luar}\label{maklumat-untuk-penyedia-luar}}

Organisasi hendaklah memastikan kecukupan keperluan sebelum
dikomunikasikan kepada penyedia luar.

Organisasi hendaklah mengkomunikasikan kepada penyedia luar keperluannya
tentang:

\begin{enumerate}
\def\labelenumi{\alph{enumi})}
\item
  proses, produk dan perkhidmatan yang akan disediakan;
\item
  kelulusan untuk:
\end{enumerate}

\begin{enumerate}
\def\labelenumi{\arabic{enumi})}
\item
  produk dan perkhidmatan;
\item
  kaedah, proses dan peralatan;
\item
  pelepasan produk dan perkhidmatan;
\end{enumerate}

\begin{enumerate}
\def\labelenumi{\alph{enumi})}
\setcounter{enumi}{2}
\item
  kekompetenan, termasuk apa-apa kelayakan yang diperlukan oleh
  seseorang;
\item
  saling tindak penyedia luar dengan organisasi;
\item
  kawalan dan pemantauan prestasi penyedia luar yang akan diguna pakai
  oleh organisasi;
\item
  aktiviti penentusahan atau pengesahan yang organisasi, atau
  pelanggannya, bermaksud untuk melaksanakan di premis penyedia luar.
\end{enumerate}

\hypertarget{penyediaan-pengeluaran-dan-perkhidmatan}{%
\paragraph{8.5 Penyediaan pengeluaran dan
perkhidmatan}\label{penyediaan-pengeluaran-dan-perkhidmatan}}

\hypertarget{kawalan-penyediaan-pengeluaran-dan-perkhidmatan}{%
\subparagraph{8.5.1 Kawalan penyediaan pengeluaran dan
perkhidmatan}\label{kawalan-penyediaan-pengeluaran-dan-perkhidmatan}}

Organisasi hendaklah melaksanakan penyediaan pengeluaran dan
perkhidmatan dalam keadaan terkawal.

Keadaan terkawal hendaklah termasuk, jika berkenaan:

\begin{enumerate}
\def\labelenumi{\alph{enumi})}
\tightlist
\item
  ketersediaan maklumat didokumentasikan yang menetapkan:
\end{enumerate}

\begin{enumerate}
\def\labelenumi{\arabic{enumi})}
\item
  ciri-ciri produk yang akan dihasilkan, perkhidmatan yang akan
  disediakan, atau aktiviti yang akan dilaksanakan;
\item
  hasil yang hendak dicapai;
\end{enumerate}

\begin{enumerate}
\def\labelenumi{\alph{enumi})}
\setcounter{enumi}{1}
\item
  ketersediaan dan penggunaan sumber pemantauan dan pengukuran yang
  sesuai;
\item
  pelaksanaan aktiviti pemantauan dan pengukuran pada tahap yang sesuai
  untuk menentusahkan bahawa kriteria kawalan proses atau output, dan
  kriteria penerimaan produk dan perkhidmatan telah dipenuhi;
\item
  penggunaan prasarana dan persekitaran yang sesuai untuk operasi
  proses;
\item
  pelantikan orang yang kompeten, termasuk apa-apa kelayakan yang
  diperlukan;
\item
  pengesahan dan pengesahan semula secara berkala, terhadap keupayaan
  untuk mencapai hasil yang dirancang daripada proses bagi penyediaan
  pengeluaran dan perkhidmatan, jika output yang dihasilkan tidak boleh
  ditentusahkan melalui pemantauan atau pengukuran berikutnya;
\item
  pelaksanaan tindakan bagi mencegah kesilapan manusia;
\item
  pelaksanaan aktiviti pelepasan, hantar serah dan selepas hantar serah.
\end{enumerate}

\hypertarget{pengenalpastian-dan-kebolehkesanan}{%
\subparagraph{8.5.2 Pengenalpastian dan
kebolehkesanan}\label{pengenalpastian-dan-kebolehkesanan}}

Organisasi hendaklah menggunakan cara yang sesuai untuk mengenal pasti
output apabila perlu bagi memastikan keakuran produk dan perkhidmatan.

Organisasi hendaklah mengenal pasti status output berkenaan dengan
keperluan pemantauan dan pengukuran sepanjang penyediaan pengeluaran dan
perkhidmatan.

Organisasi hendaklah mengawal pengenalpastian unik output apabila
kebolehkesanan merupakan satu keperluan, dan hendaklah menyimpan
maklumat didokumentasikan yang perlu untuk membolehkan kebolehkesanan.

\hypertarget{harta-kepunyaan-pelanggan-atau-penyedia-luar}{%
\subparagraph{8.5.3 Harta kepunyaan pelanggan atau penyedia
luar}\label{harta-kepunyaan-pelanggan-atau-penyedia-luar}}

Organisasi hendaklah memelihara harta kepunyaan pelanggan atau penyedia
luar semasa harta itu di bawah kawalan organisasi atau digunakan oleh
organisasi itu.

Organisasi hendaklah mengenal pasti, menentusahkan, menjaga serta
melindungi harta pelanggan atau penyedia luar yang disediakan untuk
kegunaan atau untuk dijadikan sebahagian daripada produk dan
perkhidmatan.

Apabila harta pelanggan atau penyedia luar hilang, rosak atau selainnya
didapati tidak sesuai untuk digunakan, organisasi hendaklah melaporkan
hal itu kepada pelanggan atau penyedia luar, dan menyimpan maklumat
didokumentasikan tentang hal yang telah berlaku.

NOTA. Harta pelanggan atau penyedia luar boleh termasuk bahan, komponen,
alat dan peralatan, premis, harta intelek dan data peribadi.

\hypertarget{pemeliharaan}{%
\subparagraph{8.5.4 Pemeliharaan}\label{pemeliharaan}}

Organisasi hendaklah memelihara output semasa penyediaan pengeluaran dan
perkhidmatan, setakat yang perlu bagi memastikan keakuran terhadap
keperluan.

NOTA. Pemeliharaan boleh termasuk pengenalpastian, pengendalian, kawalan
pencemaran, pembungkusan, penyimpanan, penghantaran atau pengangkutan,
dan penjagaan.

\hypertarget{aktiviti-selepas-hantar-serah}{%
\subparagraph{8.5.5 Aktiviti selepas hantar
serah}\label{aktiviti-selepas-hantar-serah}}

Organisasi hendaklah memenuhi keperluan bagi aktiviti selepas hantar
serah yang berkaitan dengan produk dan perkhidmatan.

Dalam menentukan tahap aktiviti yang diperlukan selepas hantar serah,
organisasi hendaklah mengambil kira:

\begin{enumerate}
\def\labelenumi{\alph{enumi})}
\item
  keperluan berkanun dan peraturan;
\item
  akibat tidak diingini yang mungkin timbul berkaitan produk dan
  perkhidmatannya;
\item
  keadaan, kegunaan serta hayat produk dan perkhidmatan yang
  dimaksudkan;
\item
  keperluan pelanggan;
\item
  maklum balas pelanggan.
\end{enumerate}

NOTA. Aktiviti selepas hantar serah boleh termasuk tindakan di bawah
peruntukan waranti, obligasi kontraktual seperti perkhidmatan
penyelenggaraan, dan perkhidmatan tambahan seperti kitar semula atau
pelupusan akhir.

\hypertarget{kawalan-perubahan}{%
\subparagraph{8.5.6 Kawalan perubahan}\label{kawalan-perubahan}}

Organisasi hendaklah menyemak semula dan mengawal perubahan terhadap
pengeluaran atau penyediaan perkhidmatan, setakat yang perlu bagi
memastikan keakuran berterusan terhadap keperluan.

Organisasi hendaklah mengekalkan maklumat didokumentasikan yang
memerihalkan hasil kajian semula perubahan, orang yang membenarkan
perubahan, dan apa-apa tindakan yang perlu hasil daripada kajian semula
itu.

\hypertarget{pelepasan-produk-dan-perkhidmatan}{%
\paragraph{8.6 Pelepasan produk dan
perkhidmatan}\label{pelepasan-produk-dan-perkhidmatan}}

Organisasi hendaklah melaksanakan perkiraan terancang, pada tahap yang
sesuai, untuk menentusahkan bahawa keperluan produk dan perkhidmatan
telah dipenuhi.

Pelepasan produk dan perkhidmatan kepada pelanggan tidak perlu
diteruskan sehingga perkiraan terancang disiapkan dengan memuaskan,
melainkan jika diluluskan sebaliknya oleh pihak berkuasa yang relevan
dan, jika berkenaan, oleh pelanggan.

Organisasi hendaklah mengekalkan maklumat didokumentasikan tentang
pelepasan produk dan perkhidmatan. Maklumat didokumentasikan ini
hendaklah termasuk:

\begin{enumerate}
\def\labelenumi{\alph{enumi})}
\item
  bukti keakuran terhadap kriteria penerimaan;
\item
  kebolehkesanan orang yang membenarkan pelepasan.
\end{enumerate}

\hypertarget{kawalan-output-tak-akur}{%
\paragraph{8.7 Kawalan output tak akur}\label{kawalan-output-tak-akur}}

\hypertarget{organisasi-hendaklah-memastikan-bahawa-output-tak-akur-terhadap-keperluannya}{%
\subparagraph{8.7.1 Organisasi hendaklah memastikan bahawa output tak
akur terhadap
keperluannya}\label{organisasi-hendaklah-memastikan-bahawa-output-tak-akur-terhadap-keperluannya}}

dikenal pasti dan dikawal bagi mencegah penggunaan atau hantar serah
yang tidak dimaksudkan.

Organisasi hendaklah mengambil tindakan yang sesuai berdasarkan keadaan
ketakakuran dan kesannya terhadap keakuran produk dan perkhidmatan.
Perkara ini hendaklah juga terpakai bagi produk dan perkhidmatan tak
akur yang dikesan selepas hantar serah produk, semasa atau selepas
penyediaan perkhidmatan.

Organisasi hendaklah mengurus output tak akur dengan satu atau lebih
daripada cara yang berikut:

\begin{enumerate}
\def\labelenumi{\alph{enumi})}
\item
  membuat pembetulan;
\item
  membuat pengasingan, pembendungan, pemulangan atau penggantungan
  penyediaan produk dan perkhidmatan;
\item
  memaklumkan pelanggan;
\item
  mendapatkan kebenaran untuk penerimaan di bawah konsesi.
\end{enumerate}

Keakuran terhadap keperluan hendaklah ditentusahkan apabila output tak
akur diperbetulkan.

\hypertarget{organisasi-hendaklah-mengekalkan-maklumat-didokumentasikan-yang}{%
\subparagraph{8.7.2 Organisasi hendaklah mengekalkan maklumat
didokumentasikan
yang:}\label{organisasi-hendaklah-mengekalkan-maklumat-didokumentasikan-yang}}

\begin{enumerate}
\def\labelenumi{\alph{enumi})}
\item
  memerihalkan ketakakuran;
\item
  memerihalkan tindakan yang diambil;
\item
  memerihalkan apa-apa konsesi yang diperoleh;
\item
  mengenal pasti kuasa yang memutuskan tindakan berkenaan dengan
  ketakakuran.
\end{enumerate}

\hypertarget{penilaian-prestasi}{%
\subsubsection{9 Penilaian prestasi}\label{penilaian-prestasi}}

\hypertarget{pemantauan-pengukuran-analisis-dan-penilaian}{%
\paragraph{9.1 Pemantauan, pengukuran, analisis dan
penilaian}\label{pemantauan-pengukuran-analisis-dan-penilaian}}

\hypertarget{am-7}{%
\subparagraph{9.1.1 Am}\label{am-7}}

Organisasi hendaklah menentukan:

\begin{enumerate}
\def\labelenumi{\alph{enumi})}
\item
  apa yang perlu dipantau dan diukur;
\item
  kaedah pemantauan, pengukuran, analisis dan penilaian yang diperlukan
  bagi memastikan hasil yang sah;
\item
  bila pemantauan dan pengukuran hendak dilaksanakan;
\item
  bila hasil pemantauan dan pengukuran hendak dianalisis dan dinilai.
\end{enumerate}

Organisasi hendaklah menilai prestasi dan keberkesanan sistem pengurusan
kualiti.

Organisasi hendaklah mengekalkan maklumat didokumentasikan yang sesuai
sebagai bukti hasil.

\hypertarget{kepuasan-pelanggan}{%
\subparagraph{9.1.2 Kepuasan pelanggan}\label{kepuasan-pelanggan}}

Organisasi hendaklah memantau tanggapan pelanggan tentang tahap
keperluan dan jangkaan mereka yang telah dipenuhi. Organisasi hendaklah
menentukan kaedah untuk memperoleh, memantau dan menyemak semula
maklumat ini.

NOTA. Contoh pemantauan tanggapan pelanggan boleh termasuk kaji selidik
pelanggan, maklum balas pelanggan mengenai produk dan perkhidmatan yang
dihantar serah, mesyuarat dengan pelanggan, analisis bahagian pasaran,
pujian, tuntutan waranti dan laporan wakil penjual.

\hypertarget{analisis-dan-penilaian}{%
\subparagraph{9.1.3 Analisis dan
penilaian}\label{analisis-dan-penilaian}}

Organisasi hendaklah menganalisis dan menilai data dan maklumat yang
sesuai hasil daripada pemantauan dan pengukuran.

Hasil analisis hendaklah digunakan untuk menilai:

\begin{enumerate}
\def\labelenumi{\alph{enumi})}
\item
  keakuran produk dan perkhidmatan;
\item
  tahap kepuasan pelanggan;
\item
  prestasi dan keberkesanan sistem pengurusan kualiti;
\item
  sama ada perancangan telah dilaksanakan dengan berkesan;
\item
  keberkesanan tindakan yang diambil bagi menangani risiko dan peluang;
\item
  prestasi penyedia luar;
\item
  keperluan untuk menambah baik sistem pengurusan kualiti.
\end{enumerate}

NOTA. Kaedah untuk menganalisis data boleh termasuk teknik statistik.

\hypertarget{audit-dalaman}{%
\paragraph{9.2 Audit dalaman}\label{audit-dalaman}}

\hypertarget{organisasi-hendaklah-menjalankan-audit-dalaman-secara-berkala-bagi-menyediakan}{%
\subparagraph{9.2.1 Organisasi hendaklah menjalankan audit dalaman
secara berkala bagi
menyediakan}\label{organisasi-hendaklah-menjalankan-audit-dalaman-secara-berkala-bagi-menyediakan}}

maklumat bahawa sistem pengurusan kualiti sama ada atau tidak:

\begin{enumerate}
\def\labelenumi{\alph{enumi})}
\tightlist
\item
  akur terhadap:
\end{enumerate}

\begin{enumerate}
\def\labelenumi{\arabic{enumi})}
\item
  keperluan sistem pengurusan kualiti organisasi itu sendiri;
\item
  keperluan Standard Antarabangsa ini;
\end{enumerate}

\begin{enumerate}
\def\labelenumi{\alph{enumi})}
\setcounter{enumi}{1}
\tightlist
\item
  dilaksanakan dan diselenggarakan secara berkesan.
\end{enumerate}

\hypertarget{organisasi-hendaklah-merancang-mewujudkan-melaksanakan-dan-menyelenggarakan-program-audit}{%
\subparagraph{9.2.2 Organisasi hendaklah merancang, mewujudkan,
melaksanakan dan menyelenggarakan program
audit}\label{organisasi-hendaklah-merancang-mewujudkan-melaksanakan-dan-menyelenggarakan-program-audit}}

\begin{enumerate}
\def\labelenumi{\alph{enumi})}
\item
  merancang, mewujudkan, melaksanakan dan menyelenggarakan program
  audit, termasuk kekerapan, kaedah, tanggungjawab, keperluan
  perancangan dan pelaporan, yang hendaklah mengambil kira kepentingan
  proses berkenaan, perubahan yang memberi kesan kepada organisasi, dan
  keputusan audit terdahulu;
\item
  menentukan kriteria audit dan skop bagi setiap audit;
\item
  memilih juruaudit dan menjalankan audit bagi memastikan keobjektifan
  dan kesaksamaan proses audit;
\item
  memastikan bahawa keputusan audit dilaporkan kepada pihak pengurusan
  yang relevan;
\item
  membuat pembetulan dan mengambil tindakan pembetulan yang sesuai,
  tanpa kelengahan tidak wajar;
\item
  mengekalkan maklumat didokumentasikan sebagai bukti pelaksanaan
  program audit dan keputusan audit.
\end{enumerate}

NOTA. Lihat ISO 19011 untuk panduan.

\hypertarget{kajian-semula-pengurusan}{%
\paragraph{9.3 Kajian semula
pengurusan}\label{kajian-semula-pengurusan}}

\hypertarget{am-8}{%
\subparagraph{9.3.1 Am}\label{am-8}}

Pengurusan atasan hendaklah mengkaji semula sistem pengurusan kualiti
organisasi, secara berkala, bagi memastikan kesesuaian, kecukupan,
keberkesanan dan keselarasan yang berterusan dengan hala tuju strategik
organisasi.

\hypertarget{input-kajian-semula-pengurusan}{%
\subparagraph{9.3.2 Input kajian semula
pengurusan}\label{input-kajian-semula-pengurusan}}

Kajian semula pengurusan hendaklah dirancang dan dijalankan dengan
mengambil kira:

\begin{enumerate}
\def\labelenumi{\alph{enumi})}
\item
  status tindakan daripada kajian semula pengurusan yang terdahulu;
\item
  perubahan dalam isu-isu luaran dan dalaman yang relevan dengan sistem
  pengurusan kualiti;
\item
  maklumat tentang prestasi dan keberkesanan sistem pengurusan kualiti,
  termasuk trend dalam:
\end{enumerate}

\begin{enumerate}
\def\labelenumi{\arabic{enumi})}
\item
  kepuasan pelanggan dan maklum balas daripada pihak berkepentingan yang
  relevan;
\item
  takat pencapaian objektif kualiti;
\item
  prestasi proses dan keakuran produk dan perkhidmatan;
\item
  ketakakuran dan tindakan pembetulan;
\item
  hasil pemantauan dan pengukuran;
\item
  keputusan audit;
\item
  prestasi penyedia luar;
\end{enumerate}

\begin{enumerate}
\def\labelenumi{\alph{enumi})}
\setcounter{enumi}{3}
\item
  kecukupan sumber;
\item
  keberkesanan tindakan yang diambil bagi menangani risiko dan peluang
  (lihat 6.1);
\item
  peluang untuk penambahbaikan.
\end{enumerate}

\hypertarget{output-kajian-semula-pengurusan}{%
\subparagraph{9.3.3 Output kajian semula
pengurusan}\label{output-kajian-semula-pengurusan}}

Output kajian semula pengurusan hendaklah termasuk keputusan dan
tindakan yang berkaitan dengan:

\begin{enumerate}
\def\labelenumi{\alph{enumi})}
\item
  peluang untuk penambahbaikan;
\item
  apa-apa keperluan untuk perubahan kepada sistem pengurusan kualiti;
\item
  keperluan sumber.
\end{enumerate}

Organisasi hendaklah mengekalkan maklumat didokumentasikan sebagai bukti
hasil kajian semula pengurusan.

\hypertarget{penambahbaikan}{%
\subsubsection{10 Penambahbaikan}\label{penambahbaikan}}

\hypertarget{am-9}{%
\paragraph{10.1 Am}\label{am-9}}

Organisasi hendaklah menentukan dan memilih peluang untuk
penambahbaikan, dan melaksanakan apa-apa tindakan yang perlu, bagi
memenuhi keperluan pelanggan dan meningkatkan kepuasan pelanggan.

Perkara ini hendaklah termasuk:

\begin{enumerate}
\def\labelenumi{\alph{enumi})}
\item
  menambah baik produk dan perkhidmatan bagi memenuhi keperluan serta
  bagi menangani keperluan dan jangkaan masa depan;
\item
  membetulkan, mencegah atau mengurangkan kesan yang tidak diingini;
\item
  menambah baik prestasi dan keberkesanan sistem pengurusan kualiti.
\end{enumerate}

NOTA. Contoh penambahbaikan boleh termasuk pembetulan, tindakan
pembetulan, penambahbaikan berterusan, perubahan kejayaan besar, inovasi
dan penyusunan semula.

\hypertarget{ketakakuran-dan-tindakan-pembetulan}{%
\paragraph{10.2 Ketakakuran dan tindakan
pembetulan}\label{ketakakuran-dan-tindakan-pembetulan}}

\hypertarget{apabila-ketakakuran-berlaku-termasuk-apa-apa-yang-timbul-daripada-aduan}{%
\subparagraph{10.2.1 Apabila ketakakuran berlaku, termasuk apa-apa yang
timbul daripada
aduan,}\label{apabila-ketakakuran-berlaku-termasuk-apa-apa-yang-timbul-daripada-aduan}}

organisasi hendaklah:

\begin{enumerate}
\def\labelenumi{\alph{enumi})}
\tightlist
\item
  bertindak balas terhadap ketakakuran itu dan, jika berkenaan:
\end{enumerate}

\begin{enumerate}
\def\labelenumi{\arabic{enumi})}
\item
  mengambil tindakan untuk mengawal dan membetulkannya;
\item
  menguruskan akibatnya;
\end{enumerate}

\begin{enumerate}
\def\labelenumi{\alph{enumi})}
\setcounter{enumi}{1}
\tightlist
\item
  menilai keperluan untuk mengambil tindakan menghapuskan penyebab
  ketakakuran, supaya tidak berulang atau berlaku di tempat lain, dengan
  cara:
\end{enumerate}

\begin{enumerate}
\def\labelenumi{\arabic{enumi})}
\item
  menyemak semula dan menganalisis ketakakuran;
\item
  menentukan penyebab ketakakuran;
\item
  menentukan jika ketakakuran serupa wujud, atau mungkin boleh berlaku;
\end{enumerate}

\begin{enumerate}
\def\labelenumi{\alph{enumi})}
\setcounter{enumi}{2}
\item
  melaksanakan apa-apa tindakan yang diperlukan;
\item
  menyemak semula keberkesanan apa-apa tindakan pembetulan yang diambil;
\item
  mengemas kini risiko dan peluang yang ditentukan semasa perancangan,
  jika perlu;
\item
  membuat perubahan dalam sistem pengurusan kualiti, jika perlu.
\end{enumerate}

Tindakan pembetulan hendaklah bersesuaian dengan kesan ketakakuran yang
dihadapi.

\hypertarget{organisasi-hendaklah-mengekalkan-maklumat-didokumentasikan-sebagai-bukti}{%
\subparagraph{10.2.2 Organisasi hendaklah mengekalkan maklumat
didokumentasikan sebagai
bukti:}\label{organisasi-hendaklah-mengekalkan-maklumat-didokumentasikan-sebagai-bukti}}

\begin{enumerate}
\def\labelenumi{\alph{enumi})}
\item
  keadaan ketakakuran dan apa-apa tindakan susulan yang diambil;
\item
  hasil apa-apa tindakan pembetulan.
\end{enumerate}

\hypertarget{penambahbaikan-berterusan}{%
\paragraph{10.3 Penambahbaikan
berterusan}\label{penambahbaikan-berterusan}}

Organisasi hendaklah secara berterusan menambah baik kesesuaian,
kecukupan dan keberkesanan sistem pengurusan kualiti.

Organisasi hendaklah mengambil kira keputusan analisis dan penilaian,
dan output daripada kajian semula pengurusan, bagi menentukan jika
terdapat keperluan atau peluang yang sepatutnya ditangani sebagai
sebahagian daripada penambahbaikan berterusan.

\hypertarget{lampiran-a}{%
\subsubsection{Lampiran A}\label{lampiran-a}}

\hypertarget{a.1-struktur-dan-istilah}{%
\paragraph{A.1 Struktur dan istilah}\label{a.1-struktur-dan-istilah}}

Struktur klausa (iaitu urutan klausa) dan beberapa istilah edisi
Standard Antarabangsa ini, berbanding dengan edisi terdahulu (ISO
9001:2008), telah diubah untuk menambah baik sejajar dengan standard
sistem pengurusan yang lain.

Dalam Standard Antarabangsa ini, tidak ada keperluan supaya struktur dan
istilahnya diguna pakai terhadap maklumat didokumentasikan dalam sistem
pengurusan kualiti organisasi.

Struktur klausa bermaksud menyediakan pembentangan keperluan yang
koheren, bukannya model untuk mendokumentasikan dasar, matlamat dan
proses sesuatu organisasi. Struktur dan kandungan maklumat
didokumentasikan yang berkaitan dengan sistem pengurusan kualiti sering
boleh menjadi lebih relevan kepada penggunanya jika ia berkaitan dengan
proses yang dikendalikan oleh organisasi dan maklumat yang
diselenggarakan untuk maksud lain.

Istilah yang digunakan oleh organisasi tidak perlu digantikan dengan
istilah yang digunakan dalam Standard Antarabangsa ini bagi menetapkan
keperluan sistem pengurusan kualiti. Organisasi boleh memilih untuk
menggunakan istilah yang sesuai dengan operasinya (contohnya,
menggunakan ``rekod'', ``pendokumenan'' atau ``protokol'' dan bukannya
``maklumat didokumentasikan''; atau ``pembekal'', ``rakan kongsi'' atau
``penjual'' dan bukannya ``penyedia luar''). Jadual A.1 menunjukkan
perbezaan utama dalam istilah antara edisi Standard Antarabangsa ini
dengan edisi terdahulu.

\hypertarget{a.2-produk-dan-perkhidmatan}{%
\paragraph{A.2 Produk dan
perkhidmatan}\label{a.2-produk-dan-perkhidmatan}}

ISO 9001:2008 menggunakan istilah ``produk'' bagi merangkumi semua
kategori output. Edisi Standard Antarabangsa ini menggunakan ``produk
dan perkhidmatan''. Istilah ``produk dan perkhidmatan'' termasuk semua
kategori output (perkakasan, perkhidmatan, perisian dan bahan diproses).

Kemasukan khusus istilah ``perkhidmatan'' bermaksud untuk menunjukkan
ada perbezaan antara produk dengan perkhidmatan dalam pemakaian
sesetengah keperluan. Ciri-ciri perkhidmatan ialah sekurang-kurangnya
sebahagian daripada output dihasilkan pada antara muka dengan pelanggan.
Ini bermakna, sebagai contoh, bahawa keakuran terhadap keperluan tidak
semestinya disahkan sebelum hantar serah perkhidmatan.

Dalam kebanyakan hal, produk dan perkhidmatan digunakan bersama-sama.
Kebanyakan output yang disediakan organisasi kepada pelanggan, atau
dibekalkan kepada mereka oleh pembekal luar, termasuk kedua-dua produk
dan perkhidmatan. Sebagai contoh, produk ketara atau tak ketara boleh
mempunyai beberapa perkhidmatan yang berkaitan atau sesuatu perkhidmatan
boleh mempunyai beberapa produk ketara atau tak ketara yang berkaitan.

\hypertarget{a.3-memahami-keperluan-dan-jangkaan-pihak-berkepentingan}{%
\paragraph{A.3 Memahami keperluan dan jangkaan pihak
berkepentingan}\label{a.3-memahami-keperluan-dan-jangkaan-pihak-berkepentingan}}

Subklausa 4.2 menetapkan keperluan untuk organisasi menentukan pihak
berkepentingan yang relevan dengan sistem pengurusan kualiti dan
keperluan pihak berkepentingan tersebut. Walau bagaimanapun, 4.2 tidak
membayangkan tambahan keperluan sistem pengurusan kualiti di luar skop
Standard Antarabangsa ini. Seperti yang dinyatakan dalam skop, Standard
Antarabangsa ini diguna pakai jika organisasi perlu menunjukkan
keupayaannya menyediakan secara tekal produk dan perkhidmatan yang
memenuhi keperluan pelanggan serta keperluan berkanun dan peraturan yang
diguna pakai, dan bertujuan untuk meningkatkan kepuasan pelanggan.

Dalam Standard Antarabangsa ini, tidak ada keperluan supaya organisasi
mengambil kira pihak berkepentingan jika telah diputuskan bahawa pihak
berkepentingan tersebut tidak relevan dengan sistem pengurusan
kualitinya. Terpulang kepada organisasi untuk memutuskan jika keperluan
tertentu pihak berkepentingan yang relevan adalah berkaitan dengan
sistem pengurusan kualitinya.

\hypertarget{a.4-pemikiran-berasaskan-risiko}{%
\paragraph{A.4 Pemikiran berasaskan
risiko}\label{a.4-pemikiran-berasaskan-risiko}}

Konsep pemikiran berasaskan risiko tersirat dalam edisi terdahulu
Standard Antarabangsa ini, contohnya melalui keperluan untuk merancang,
menyemak semula dan menambah baik. Standard Antarabangsa ini menetapkan
keperluan bagi organisasi memahami konteksnya (lihat 4.1) dan menentukan
risiko sebagai asas perancangan (lihat 6.1). Hal ini menggambarkan
penggunaan pemikiran berasaskan risiko untuk merancang dan melaksanakan
proses sistem pengurusan kualiti (lihat 4.4) dan membantu dalam
menentukan tahap maklumat didokumentasikan.

Satu daripada matlamat utama sistem pengurusan kualiti adalah untuk
berfungsi sebagai alat pencegahan. Oleh itu, Standard Antarabangsa ini
tidak mempunyai suatu klausa atau subklausa berasingan tentang tindakan
pencegahan. Konsep tindakan pencegahan dinyatakan melalui penggunaan
pemikiran berasaskan risiko dalam merangka keperluan sistem pengurusan
kualiti.

Pemikiran berasaskan risiko yang diguna pakai dalam Standard
Antarabangsa ini membolehkan sedikit pengurangan dalam keperluan
preskriptif dan penggantiannya dengan keperluan berasaskan prestasi.
Terdapat keluwesan lebih besar berbanding yang terkandung dalam ISO
9001:2008 dalam keperluan untuk proses, maklumat didokumentasikan dan
tanggungjawab organisasi.

Walaupun 6.1 menetapkan bahawa organisasi hendaklah merancang tindakan
bagi menyatakan risiko, tidak ada keperluan terhadap kaedah formal untuk
pengurusan risiko atau proses pengurusan risiko yang didokumentasikan.
Organisasi boleh memutuskan sama ada atau tidak untuk membangunkan
metodologi pengurusan risiko yang lebih luas daripada yang diperlukan
oleh Standard Antarabangsa ini, contohnya melalui penggunaan panduan
atau standard lain.

Tidak semua proses sistem pengurusan kualiti mewakili tahap risiko yang
sama dari segi keupayaan organisasi mencapai matlamatnya, dan kesan
ketaktentuan adalah tidak sama bagi semua organisasi. Di bawah keperluan
6.1, organisasi bertanggungjawab atas penggunaan pemikiran berasaskan
risiko dan tindakan yang diambil oleh organisasi untuk menyatakan
risiko, termasuk sama ada atau tidak untuk mengekalkan maklumat
didokumentasikan sebagai bukti penentuan risikonya.

\hypertarget{a.5-kebolehgunaan}{%
\paragraph{A.5 Kebolehgunaan}\label{a.5-kebolehgunaan}}

Standard Antarabangsa ini tidak menyebut ``tidak dimasukkan'' berhubung
dengan kebolehgunaan keperluannya kepada sistem pengurusan kualiti
organisasi. Walau bagaimanapun, sesuatu organisasi boleh menyemak semula
kebolehgunaan keperluan disebabkan oleh saiz atau kerumitan organisasi,
model pengurusan yang diterima guna, pelbagai aktiviti organisasi dan
jenis risiko dan peluang yang dihadapi.

Keperluan untuk kebolehgunaan dinyatakan dalam 4.3, yang menetapkan
keadaan yang sesuatu organisasi boleh memutuskan bahawa sesuatu
keperluan tidak boleh diguna pakai untuk mana-mana proses dalam skop
sistem pengurusan kualitinya. Organisasi hanya boleh memutuskan bahawa
sesuatu keperluan itu tidak terpakai jika keputusannya tidak akan
mengakibatkan kegagalan untuk mencapai keakuran produk dan perkhidmatan.

\hypertarget{a.6-maklumat-didokumentasikan}{%
\paragraph{A.6 Maklumat
didokumentasikan}\label{a.6-maklumat-didokumentasikan}}

Sebagai sebahagian daripada kesejajaran dengan standard sistem
pengurusan yang lain, klausa umum tentang ``maklumat didokumentasikan''
telah diterima guna tanpa perubahan atau tambahan ketara (lihat 7.5).
Jika sesuai, teks lain dalam Standard Antarabangsa ini telah
disejajarkan dengan keperluannya. Oleh itu, ``maklumat
didokumentasikan'' digunakan untuk semua keperluan dokumen.

Jika ISO 9001:2008 menggunakan istilah khusus seperti ``dokumen'' atau
``prosedur didokumentasikan'', ``manual kualiti'' atau ``rancangan
kualiti'', edisi Standard Antarabangsa ini menetapkan keperluan untuk
``menyelenggara maklumat didokumentasikan''.

Jika ISO 9001:2008 menggunakan istilah ``rekod'' bagi menandakan dokumen
yang diperlukan untuk menyediakan bukti keakuran terhadap keperluan, hal
ini kini dinyatakan sebagai keperluan untuk ``mengekalkan maklumat
didokumentasikan''. Organisasi bertanggungjawab menentukan maklumat
didokumentasikan yang perlu dikekalkan, tempoh pengekalannya dan media
yang akan digunakan untuk mengekalkannya.

Keperluan untuk ``menyelenggara'' maklumat didokumentasikan tidak
menolak kemungkinan bahawa organisasi juga mungkin perlu untuk
``mengekalkan'' maklumat didokumentasikan yang sama untuk tujuan
tertentu, contohnya menyimpan versinya yang terdahulu.

Jika Standard Antarabangsa ini menyebut ``maklumat'' dan bukannya
``maklumat didokumentasikan'' (contohnya dalam 4.1: ``Organisasi
hendaklah memantau dan menyemak semula maklumat tentang isu luaran dan
dalaman ini''), tidak ada keperluan supaya maklumat ini
didokumentasikan. Dalam keadaan ini, organisasi boleh memutuskan sama
ada atau tidak perlu atau wajar menyelenggarakan maklumat
didokumentasikan.

\hypertarget{a.7-pengetahuan-organisasi}{%
\paragraph{A.7 Pengetahuan
organisasi}\label{a.7-pengetahuan-organisasi}}

Dalam 7.1.6, Standard Antarabangsa ini menyatakan keperluan untuk
menentukan dan mengurus pengetahuan yang diselenggarakan oleh
organisasi, bagi memastikan bahawa keakuran produk dan perkhidmatan
boleh dicapai.

Keperluan tentang pengetahuan organisasi diperkenalkan bagi maksud:

\begin{enumerate}
\def\labelenumi{\alph{enumi})}
\tightlist
\item
  melindungi organisasi daripada kehilangan pengetahuan, contohnya
\end{enumerate}

\begin{itemize}
\tightlist
\item
  melalui pusing ganti kakitangan;
\item
  kegagalan untuk mendapatkan dan berkongsi maklumat;
\end{itemize}

\begin{enumerate}
\def\labelenumi{\alph{enumi})}
\setcounter{enumi}{1}
\tightlist
\item
  menggalakkan organisasi memperoleh pengetahuan, contohnya
\end{enumerate}

\begin{itemize}
\tightlist
\item
  belajar daripada pengalaman;
\item
  pementoran;
\item
  penandaarasan.
\end{itemize}

\hypertarget{a.8-kawalan-proses-produk-dan-perkhidmatan-sediaan-luar}{%
\paragraph{A.8 Kawalan proses, produk dan perkhidmatan sediaan
luar}\label{a.8-kawalan-proses-produk-dan-perkhidmatan-sediaan-luar}}

Semua bentuk proses, produk dan perkhidmatan sediaan luar dinyatakan
dalam 8.4, contohnya melalui:

\begin{enumerate}
\def\labelenumi{\alph{enumi})}
\item
  pembelian daripada pembekal;
\item
  perkiraan dengan syarikat bersekutu;
\item
  penyumberan luar proses kepada penyedia luar.
\end{enumerate}

Penyumberan luar lazimnya mempunyai ciri-ciri penting perkhidmatan,
kerana ia mempunyai sekurang-kurangnya satu aktiviti yang semestinya
dilakukan pada antara muka antara penyedia dengan organisasi.

Kawalan yang diperlukan untuk penyediaan luar boleh sangat berbeza
bergantung kepada keadaan proses, produk dan perkhidmatan. Organisasi
boleh mengguna pakai pemikiran berasaskan risiko bagi menentukan jenis
dan takat kawalan yang sesuai untuk penyedia luar dan proses, produk dan
perkhidmatan sediaan luar yang tertentu.

\hypertarget{lampiran-b}{%
\subsubsection{Lampiran B}\label{lampiran-b}}

Standard Antarabangsa lain tentang pengurusan kualiti dan sistem

pengurusan kualiti yang dibangunkan oleh ISO/TC 176

Standard Antarabangsa yang diperihalkan dalam lampiran ini telah
dibangunkan oleh ISO/TC 176 untuk menyediakan maklumat sokongan kepada
organisasi yang menggunakan Standard Antarabangsa ini, dan untuk memberi
panduan kepada organisasi yang memilih untuk maju melampaui
keperluannya. Panduan atau keperluan yang terkandung dalam dokumen yang
tersenarai dalam lampiran ini tidak menambah, atau mengubah suai,
keperluan Standard Antarabangsa ini.

Jadual B.1 menunjukkan hubungan antara standard berkenaan dengan klausa
yang relevan dalam Standard Antarabangsa ini.

Lampiran ini tidak termasuk rujukan standard sistem pengurusan kualiti
sektor khusus yang dibangunkan oleh ISO/TC 176.

Standard Antarabangsa ini adalah satu daripada tiga standard teras yang
dibangunkan oleh ISO/TC 176.

\begin{itemize}
\tightlist
\item
  ISO 9000 Quality management systems - Fundamentals and vocabulary
  menyediakan latar belakang yang penting untuk pemahaman dan
  pelaksanaan yang sesuai bagi Standard Antarabangsa ini. Prinsip
  pengurusan kualiti diperihalkan secara terperinci dalam ISO 9000 dan
  telah diambil kira semasa pembangunan Standard Antarabangsa ini.
  Prinsip ini bukanlah keperluan sebenar, tetapi merupakan asas bagi
  keperluan yang ditetapkan oleh Standard Antarabangsa ini. ISO 9000
  juga menetapkan terma, takrifan dan konsep yang digunakan dalam
  Standard Antarabangsa ini.
\item
  ISO 9001 (Standard Antarabangsa ini) menetapkan keperluan yang tujuan
  utamanya adalah untuk memberi keyakinan terhadap produk dan
  perkhidmatan yang disediakan oleh sesuatu organisasi dan dengan itu
  meningkatkan kepuasan pelanggan. Pelaksanaannya yang sesuai juga boleh
  dijangka akan membawa manfaat lain kepada organisasi, seperti
  komunikasi dalaman yang bertambah baik, pemahaman dan kawalan proses
  organisasi yang lebih baik.
\item
  ISO 9004 Managing for the sustained success of an organization - A
  quality management approach menyediakan panduan kepada organisasi yang
  memilih untuk maju melampaui keperluan Standard Antarabangsa ini, bagi
  menangani topik yang lebih meluas yang boleh membawa kepada
  penambahbaikan prestasi keseluruhan organisasi. ISO 9004 termasuk
  panduan tentang metodologi swapenilaian yang membolehkan sesebuah
  organisasi menilai tahap kematangan sistem pengurusan kualitinya.
\end{itemize}

Standard Antarabangsa yang digariskan di bawah boleh membantu organisasi
apabila organisasi mewujudkan atau ingin menambah baik sistem kualiti
pengurusan, proses atau aktivitinya.

\begin{itemize}
\item
  ISO 10001 Quality management - Customer satisfaction - Guidelines for
  codes of conduct for organizations menyediakan panduan untuk sesuatu
  organisasi dalam menentukan bahawa peruntukan kepuasan pelanggannya
  memenuhi keperluan dan jangkaan pelanggan. Penggunaannya dapat
  meningkatkan keyakinan pelanggan terhadap sesuatu organisasi dan
  menambah baik pemahaman pelanggan tentang apa yang diharapkan daripada
  sesuatu organisasi, dan dengan itu mengurangkan kemungkinan salah
  faham dan aduan.
\item
  ISO 10002 Quality management - Customer satisfaction - Guidelines for
  complaints handling in organizations menyediakan panduan tentang
  proses pengendalian aduan dengan mengambil kira dan menangani
  keperluan dan jangkaan pengadu serta menyelesaikan apa-apa aduan yang
  diterima. ISO 10002 menyediakan proses aduan terbuka, berkesan dan
  mudah untuk digunakan, termasuk latihan modal insan. Ia juga
  menyediakan panduan untuk perniagaan kecil.
\item
  ISO 10003 Quality management - Customer satisfaction - Guidelines for
  dispute resolution external to organizations menyediakan panduan bagi
  penyelesaian pertikaian luar yang cekap dan berkesan untuk aduan
  berkaitan produk. Penyelesaian pertikaian memberikan ruang bagi tebus
  rugi apabila organisasi tidak membetulkan aduan secara dalaman.
  Kebanyakan aduan boleh diselesaikan dengan jayanya dalam organisasi,
  tanpa prosedur bertentangan.
\item
  ISO 10004 Quality management - Customer satisfaction - Guidelines for
  monitoring and measuring menyediakan garis panduan bagi tindakan untuk
  meningkatkan kepuasan pelanggan dan untuk menentukan peluang bagi
  penambahbaikan produk, proses dan sifat khusus yang dihargai oleh
  pelanggan. Tindakan sedemikian boleh mengukuhkan kesetiaan pelanggan
  dan membantu mengekalkan pelanggan.
\item
  ISO 10005 Quality management systems - Guidelines for quality plans
  menyediakan panduan bagi mewujudkan dan menggunakan rancangan kualiti
  sebagai suatu cara untuk menghubungkan keperluan proses, produk,
  projek atau kontrak, dengan kaedah dan amalan kerja yang menyokong
  penghasilan produk. Manfaat mewujudkan rancangan kualiti ialah
  peningkatan keyakinan bahawa keperluan akan dipenuhi, proses berada
  dalam kawalan dan motivasi yang dapat diberikan kepada mereka yang
  terlibat.
\item
  ISO 10006 Quality management systems - Guidelines for quality
  management in projects terpakai untuk projek kecil mahupun yang besar,
  mudah mahupun rumit, projek individu mahupun sebahagian daripada
  portfolio projek. ISO 10006 adalah untuk digunakan oleh kakitangan
  yang mengurus projek dan sesiapa yang perlu bagi memastikan bahawa
  organisasi mereka mengguna pakai amalan yang terkandung dalam standard
  sistem pengurusan kualiti ISO.
\item
  ISO 10007 Quality management systems - Guidelines for configuration
  management adalah untuk membantu organisasi yang mengguna pakai
  pengurusan konfigurasi bagi hala tuju teknikal dan pentadbiran
  sepanjang kitaran hayat produk. Pengurusan konfigurasi boleh digunakan
  untuk memenuhi keperluan pengenalpastian dan kebolehkesanan produk
  yang dinyatakan dalam Standard Antarabangsa ini.
\item
  ISO 10008 Quality management - Customer satisfaction - Guidelines for
  business-to- consumer electronic commerce transactions memberikan
  panduan tentang cara organisasi boleh melaksanakan sistem urus niaga
  perdagangan elektronik perniagaan- ke-pengguna yang berkesan dan cekap
  (B2C ECT), dan dengan itu menyediakan asas supaya pengguna mempunyai
  keyakinan yang meningkat terhadap B2C ECT, meningkatkan keupayaan
  organisasi untuk memuaskan hati pengguna dan membantu mengurangkan
  aduan dan pertikaian.
\item
  ISO 10012 Measurement management systems - Requirements for
  measurement processes and measuring menyediakan panduan bagi
  pengurusan proses pengukuran dan pengesahan metrologi peralatan
  pengukuran yang digunakan untuk menyokong dan menunjukkan pematuhan
  terhadap keperluan metrologi. ISO 10012 menyediakan kriteria
  pengurusan kualiti untuk sistem pengurusan pengukuran bagi memastikan
  keperluan metrologi dipenuhi.
\item
  ISO/TR 10013 Guidelines for quality management system documentation
  menyediakan garis panduan untuk pembangunan dan penyelenggaraan
  pendokumenan yang diperlukan untuk sistem pengurusan kualiti. ISO/TR
  10013 boleh digunakan untuk mendokumenkan sistem pengurusan selain
  standard sistem pengurusan kualiti ISO, contohnya sistem pengurusan
  alam sekitar dan sistem pengurusan keselamatan.
\item
  ISO 10014 Quality management - Guidelines for realizing financial and
  economic benefits ditujukan kepada pengurusan atasan. Ia menyediakan
  garis panduan untuk merealisasikan faedah kewangan dan ekonomi melalui
  penggunaan prinsip pengurusan kualiti. Ia memudahkan penggunaan
  prinsip pengurusan dan pemilihan kaedah dan alat yang membolehkan
  kejayaan mampan sesuatu organisasi.
\item
  ISO 10015 Quality management - Guidelines for training menyediakan
  garis panduan untuk membantu organisasi dalam menangani isu yang
  berkaitan dengan latihan. ISO 10015 boleh diguna pakai pada bila-bila
  masa bimbingan diperlukan untuk mentafsir perkataan ``pendidikan'' dan
  ``latihan'' dalam standard sistem pengurusan kualiti ISO. Apa- apa
  perkataan ``latihan'' termasuk semua jenis pendidikan dan latihan.
\item
  ISO/TR 10017 Guidance on statistical techniques for ISO 9001:2000
  memperjelaskan teknik statistik yang bermula seawal perubahan yang
  dapat dilihat dalam tingkah laku dan hasil proses, walaupun di bawah
  keadaan kestabilan nyata. Teknik statistik membolehkan penggunaan yang
  lebih baik terhadap data yang ada untuk membantu dalam membuat
  keputusan, dan dengan itu membantu untuk terus menambah baik kualiti
  produk dan proses bagi mencapai kepuasan pelanggan.
\item
  ISO 10018 Quality management - Guidelines on people involvement and
  competence menyediakan garis panduan yang mempengaruhi pelibatan modal
  insan dan kekompetenan. Sistem pengurusan kualiti bergantung kepada
  pelibatan orang yang kompeten dan cara mereka diperkenalkan dan
  disepadukan dalam organisasi. Penting untuk menentukan, membangun dan
  menilai pengetahuan, kemahiran, tingkah laku dan persekitaran kerja
  yang diperlukan.
\item
  ISO 10019 Guidelines for the selection of quality management system
  consultants and use of their services menyediakan panduan tentang
  pemilihan perunding sistem pengurusan kualiti dan penggunaan
  perkhidmatan mereka. Ia memberikan panduan tentang proses penilaian
  kekompetenan perunding sistem pengurusan kualiti dan memberi keyakinan
  bahawa keperluan dan jangkaan organisasi bagi perkhidmatan perunding
  akan dipenuhi.
\item
  ISO 19011 Guidelines for auditing management systems menyediakan
  panduan tentang pengurusan program audit, perancangan dan pengendalian
  audit sistem pengurusan, serta tentang kekompetenan dan penilaian
  juruaudit dan pasukan audit. ISO 19011 bertujuan untuk diguna pakai
  oleh juruaudit, organisasi yang melaksanakan sistem pengurusan dan
  organisasi yang perlu menjalankan audit sistem pengurusan.
\end{itemize}

\hypertarget{bibliografi}{%
\subsubsection{Bibliografi}\label{bibliografi}}

{[}1{]} ISO 9004, Managing for the sustained success of an organization
- A quality management approach

{[}2{]} ISO 10001, Quality management - Customer satisfaction -
Guidelines for codes of conduct for organizations

{[}3{]} ISO 10002, Quality management - Customer satisfaction -
Guidelines for complaints handling in organizations

{[}4{]} ISO 10003, Quality management - Customer satisfaction -
Guidelines for dispute resolution external to organizations

{[}5{]} ISO 10004, Quality management - Customer satisfaction -
Guidelines for monitoring and measuring

{[}6{]} ISO 10005, Quality management systems - Guidelines for quality
plans

{[}7{]} ISO 10006, Quality management systems - Guidelines for quality
management in projects

{[}8{]} ISO 10007, Quality management systems - Guidelines for
configuration management

{[}9{]} ISO 10008, Quality management - Customer satisfaction -
Guidelines for business-to- consumer electronic commerce transactions

{[}10{]} ISO 10012, Measurement management systems - Requirements for
measurement processes and measuring equipment

{[}11{]} ISO/TR 10013, Guidelines for quality management system
documentation

{[}12{]} ISO 10014, Quality management - Guidelines for realizing
financial and economic benefits

{[}13{]} ISO 10015, Quality management - Guidelines for training

{[}14{]} ISO/TR 10017, Guidance on statistical techniques for ISO
9001:2000

{[}15{]} ISO 10018, Quality management - Guidelines on people
involvement and competence

{[}16{]} ISO 10019, Guidelines for the selection of quality management
system consultants and use of their services

{[}17{]} ISO 14001, Environmental management systems - Requirements with
guidance for use

{[}18{]} ISO 19011, Guidelines for auditing management systems

{[}19{]} ISO 31000, Risk management - Principles and guidelines

{[}20{]} ISO 37500, Guidance on outsourcing

{[}21{]} ISO/IEC 90003, Software engineering - Guidelines for the
application of ISO 9001:2008 to computer software

{[}22{]} IEC 60300-1, Dependability management - Part 1: Guidance for
management and application

{[}23{]} IEC 61160, Design review

{[}24{]} Quality management principles, ISO1)\^{}1

{[}25{]} Selection and use of the ISO 9000 family of standards,
ISO1)\^{}

{[}26{]} ISO 9001 for Small Businesses - What to do, ISO1)\^{}

{[}27{]} Integrated use of management system standards, ISO1)\^{}

{[}28{]} \url{http://www.iso.org/tc176/sc02/public}

{[}29{]} \url{http://www.iso.org/tc176/ISO9001AuditingPracticesGroup}

(\^{}1) Tersedia dalam tapak sesawang: \url{http://www.iso.org.}

\hypertarget{penghargaan}{%
\subsubsection{Penghargaan}\label{penghargaan}}

Ahli Jawatankuasa Teknikal Pengurusan Kualiti dan Penentuan Kualiti
(TC2) mengenai Sistem Kualiti Name Organisation

Encik Parama Iswara (Pengerusi) SIRIM QAS International Sdn Bhd

Puan Nageswary S Iyampillai (Setiausaha) SIRIM Berhad

Dr Loi Kheng Min Dewan Perdagangan dan Industri Antarabangsa Malaysia

Datuk Parmjit Singh Gabungan Pembekal-pembekal Perkhidmatan Malaysia

Encik Wong Siew Kwan Institut Penyelidikan Sains dan Teknologi
Pertahanan

Encik Kamarul Ariffin A Karim Institute of Quality Malaysia

Encik Benardos Bingkang Jabatan Standard Malaysia

Encik Chuang Kuang Hong Lembaga Pembangunan Industri Pembinaan Malaysia

Encik Mohd Azani Jabar Unit Pemodenan Tadbiran dan Perancangan
Pengurusan Malaysia

Dr Ahmad Zabidi Abd Razak Universiti Malaya

Prof Dr Rushami Zien Yusoff Universiti Utara Malaysia

Ahli Kumpulan Kerja mengenai Terjemahan MS ISO 9001:20 15

Name Organisation

Prof Dr Rushami Zien Yusoff (Pengerusi) Universiti Utara Malaysia

Puan Farida Rizal Mohamad Farid (Setiausaha) SIRIM Berhad

Encik Md Fadzli Tajuid Puan Hendun Ibrahim Puan Ana Mohamad Bess

Dewan Bahasa dan Pustaka Malaysia

Cik Che Norhasimah Abd Hamid Puan Nik Nadiah Ramiz Yaaziz Encik Yusof
Cha

Institut Terjemahan \& Buku Malaysia

Encik Kamarul Ariffin A Karim Institute of Quality Malaysia

Encik Nik Mohd Syazwan Nik Mohd Zambi Jabatan Standard Malaysia

Encik Mokhzani Aris Mohd Yusof Perbadanan Produktiviti Malaysia

Puan Rohana Saub Puan Hanida Ghazali

SIRIM QAS International Sdn Bhd

Encik Mohd Azani Jabar Encik Magintharan Rengiah

Unit Pemodenan Tadbiran dan Perancangan Pengurusan Malaysia

Prof Madya Dr Sany Sanuri Mohd Mokhtar Universiti Utara Malaysia

\end{document}
