% Options for packages loaded elsewhere
\PassOptionsToPackage{unicode}{hyperref}
\PassOptionsToPackage{hyphens}{url}
%
\documentclass[
]{article}
\usepackage{lmodern}
\usepackage{amssymb,amsmath}
\usepackage{ifxetex,ifluatex}
\ifnum 0\ifxetex 1\fi\ifluatex 1\fi=0 % if pdftex
  \usepackage[T1]{fontenc}
  \usepackage[utf8]{inputenc}
  \usepackage{textcomp} % provide euro and other symbols
\else % if luatex or xetex
  \usepackage{unicode-math}
  \defaultfontfeatures{Scale=MatchLowercase}
  \defaultfontfeatures[\rmfamily]{Ligatures=TeX,Scale=1}
\fi
% Use upquote if available, for straight quotes in verbatim environments
\IfFileExists{upquote.sty}{\usepackage{upquote}}{}
\IfFileExists{microtype.sty}{% use microtype if available
  \usepackage[]{microtype}
  \UseMicrotypeSet[protrusion]{basicmath} % disable protrusion for tt fonts
}{}
\makeatletter
\@ifundefined{KOMAClassName}{% if non-KOMA class
  \IfFileExists{parskip.sty}{%
    \usepackage{parskip}
  }{% else
    \setlength{\parindent}{0pt}
    \setlength{\parskip}{6pt plus 2pt minus 1pt}}
}{% if KOMA class
  \KOMAoptions{parskip=half}}
\makeatother
\usepackage{xcolor}
\IfFileExists{xurl.sty}{\usepackage{xurl}}{} % add URL line breaks if available
\IfFileExists{bookmark.sty}{\usepackage{bookmark}}{\usepackage{hyperref}}
\hypersetup{
  pdftitle={REKOD KUALITI},
  pdfauthor={OpenApps QMS},
  hidelinks,
  pdfcreator={LaTeX via pandoc}}
\urlstyle{same} % disable monospaced font for URLs
\usepackage[margin=1in]{geometry}
\usepackage{longtable,booktabs}
% Correct order of tables after \paragraph or \subparagraph
\usepackage{etoolbox}
\makeatletter
\patchcmd\longtable{\par}{\if@noskipsec\mbox{}\fi\par}{}{}
\makeatother
% Allow footnotes in longtable head/foot
\IfFileExists{footnotehyper.sty}{\usepackage{footnotehyper}}{\usepackage{footnote}}
\makesavenoteenv{longtable}
\usepackage{graphicx,grffile}
\makeatletter
\def\maxwidth{\ifdim\Gin@nat@width>\linewidth\linewidth\else\Gin@nat@width\fi}
\def\maxheight{\ifdim\Gin@nat@height>\textheight\textheight\else\Gin@nat@height\fi}
\makeatother
% Scale images if necessary, so that they will not overflow the page
% margins by default, and it is still possible to overwrite the defaults
% using explicit options in \includegraphics[width, height, ...]{}
\setkeys{Gin}{width=\maxwidth,height=\maxheight,keepaspectratio}
% Set default figure placement to htbp
\makeatletter
\def\fps@figure{htbp}
\makeatother
\setlength{\emergencystretch}{3em} % prevent overfull lines
\providecommand{\tightlist}{%
  \setlength{\itemsep}{0pt}\setlength{\parskip}{0pt}}
\setcounter{secnumdepth}{-\maxdimen} % remove section numbering

\title{REKOD KUALITI}
\author{OpenApps QMS}
\date{}

\begin{document}
\maketitle

{
\setcounter{tocdepth}{4}
\tableofcontents
}
\newpage

\begin{longtable}[]{@{}lr@{}}
\toprule
\begin{minipage}[b]{0.66\columnwidth}\raggedright
Prosedur Kualiti ~ ~ ~ ~ ~ ~ ~ ~ ~ ~~ ~ ~ ~ ~ ~ ~ ~ ~ ~~ ~ ~ ~ ~ ~ ~ ~ ~
~~ ~ ~ ~ ~ ~ ~ ~ ~ ~\strut
\end{minipage} & \begin{minipage}[b]{0.28\columnwidth}\raggedleft
PK(W).OA.02\strut
\end{minipage}\tabularnewline
\midrule
\endhead
\bottomrule
\end{longtable}

\begin{figure}
\centering
\includegraphics[width=0.3\textwidth,height=\textheight]{media/openapps-logo.png}
\caption{OA Logo}
\end{figure}

\begin{longtable}[]{@{}c@{}}
\toprule
\endhead
REKOD KUALITI\tabularnewline
\bottomrule
\end{longtable}

\hypertarget{objektik}{%
\section{1.0 Objektik}\label{objektik}}

Prosedur ini bertujuan memberi garis panduan bagi memastikan pengurusan
Rekod Kualiti termasuk Sistem Pengurusan Fail dan Dasar Keselamatan
Maklumat dipatuhi dan dilaksanakan secara cekap dan berkesan.

\hypertarget{skop}{%
\section{2.0 Skop}\label{skop}}

Prosedur ini digunapakai oleh warga Syarikat semasa menguruskan fail dan
rekod-rekod kualiti Sistem Pengurusan Kualiti dan Keselamatan Maklumat.
Sistem pengurusan ini termasuk juga pengurusan rekod dan maklumat yang
disimpan secara elektronik. Rekod kualiti berkenaan perlu disimpan
secara sistematik untuk rujukan sebagai bukti objektif kepada
aktiviti-aktiviti kualiti.

\hypertarget{rujukan}{%
\section{3.0 Rujukan}\label{rujukan}}

\begin{verbatim}
1.  MK.OA.02 -- Seksyen 4.2.4 (Kawalan Rekod Kualiti).
\end{verbatim}

3.2 Akta Arkib Negara, 1971

3.3 Dasar Keselamatan Maklumat -- DKM.OA.01

3.4 FAFSI 1961

3.5 Arahan Keselamatan

\hypertarget{definisi}{%
\section{4.0 Definisi}\label{definisi}}

\begin{verbatim}
2.  Mengindeks
\end{verbatim}

Proses memberikan nombor-nombor rujukan kepada rekod-rekod kualiti yang
dikenalpasti

\begin{enumerate}
\def\labelenumi{\arabic{enumi}.}
\setcounter{enumi}{2}
\tightlist
\item
  Pelupusan
\end{enumerate}

Proses pelupusan rekod-rekod kualiti mengikut prosedur pelupusan yang
ditetapkan sebagaimana Akta Arkib Negara 1971.

\begin{enumerate}
\def\labelenumi{\arabic{enumi}.}
\setcounter{enumi}{3}
\tightlist
\item
  Rekod
\end{enumerate}

Semua data dan maklumat yang diterima,dikeluarkan, dikumpulkan dan
direkodkan secara sistematik, sebagai bukti objektif dalam pelaksanaan
Sistem Pengurusan Kualiti.

\begin{enumerate}
\def\labelenumi{\arabic{enumi}.}
\setcounter{enumi}{4}
\tightlist
\item
  Menyelenggara
\end{enumerate}

Proses penjagaan rekod-rekod agar ianya dalam keadaan selamat, teratur
dan sentiasa dikemaskini.

\begin{enumerate}
\def\labelenumi{\arabic{enumi}.}
\setcounter{enumi}{5}
\tightlist
\item
  Mengedar
\end{enumerate}

Proses pergerakan rekod-rekod kepada pegawai-pegawai cawangan yang
dikenalpasti mengikut prosedur yang ditetapkan dalam Manual Kualiti dan
Prosedur Kualiti.

\begin{enumerate}
\def\labelenumi{\arabic{enumi}.}
\setcounter{enumi}{6}
\tightlist
\item
  Rekod Kualiti
\end{enumerate}

Dokumen hasil daripada aktiviti-aktiviti yang dilakukan mengikut Manual
Kualiti dan Prosedur Kualiti.

\begin{enumerate}
\def\labelenumi{\arabic{enumi}.}
\setcounter{enumi}{7}
\tightlist
\item
  Fail
\end{enumerate}

Kegunaan untuk menyimpan dokumen dan rekod.

4.8 Utusan/Kawat

Berita tentera (maklumat) yang diterima dari pasukan yang lain dalam
bentuk utusan ringkas.

\begin{enumerate}
\def\labelenumi{\arabic{enumi}.}
\setcounter{enumi}{8}
\tightlist
\item
  Surat Perkhidmatan Biasa
\end{enumerate}

Surat rasmi atau tidak rasmi yang dihantar secara pos awam atau khidmat
tentera berkaitan dengan perkhidmatan sahaja.

\begin{enumerate}
\def\labelenumi{\arabic{enumi}.}
\setcounter{enumi}{4}
\tightlist
\item
  Singkatan
\end{enumerate}

5.1 BSPP - Bahagian Staf Perisikan Pertahanan.

5.2 Syarikat - Syarikat.

5.3 Pengurus Besar - Pengurus Besar Rekod dan Pencen.

5.4 PS 1 - Pegawai Staf 1.

5.5 PS 2 - Pegawai Staf 2.

5.6 PS 3 - Pegawai Staf 3.

5.7 PAR - Pembantu Am Rendah.

5.8 Sjn Tad - Sarjan Tadbir.

5.9 PC - Penyelia Cawangan.

5.10 KS - Ketua Sel.

5.11 PT - Pembantu Tadbir.

\hypertarget{tanggungjawab-dan-tindakan}{%
\section{6.0 Tanggungjawab dan
Tindakan}\label{tanggungjawab-dan-tindakan}}

\begin{longtable}[]{@{}lll@{}}
\toprule
\begin{minipage}[b]{0.30\columnwidth}\raggedright
TANGGUNGJAWAB \textbar{}\strut
\end{minipage} & \begin{minipage}[b]{0.30\columnwidth}\raggedright
\textbar{} TIN\strut
\end{minipage} & \begin{minipage}[b]{0.30\columnwidth}\raggedright
DAKAN\strut
\end{minipage}\tabularnewline
\midrule
\endhead
\begin{minipage}[t]{0.30\columnwidth}\raggedright
\strut
\end{minipage} & \begin{minipage}[t]{0.30\columnwidth}\raggedright
A \textbar{} PEN\strut
\end{minipage} & \begin{minipage}[t]{0.30\columnwidth}\raggedright
ERIMAAN DAN PENDAFTARAN\strut
\end{minipage}\tabularnewline
\begin{minipage}[t]{0.30\columnwidth}\raggedright
PAR/Kerani \textbar{} 1.\strut
\end{minipage} & \begin{minipage}[t]{0.30\columnwidth}\raggedright
\begin{verbatim}
              | Ter
\end{verbatim}

\begin{enumerate}
\def\labelenumi{\arabic{enumi}.}
\setcounter{enumi}{1}
\item
\item
\item
\end{enumerate}\strut
\end{minipage} & \begin{minipage}[t]{0.30\columnwidth}\raggedright
ima dan asingkan surat perkhidmatan biasa dan utusan.

Edarkan surat peribadi kepada kakitangan berkenaan (jika ada).

Rekodkan dan cop penerimaan surat perkhidmatan biasa dan utusan.

Agihkan surat perkhidmatan biasa dan utusan kepada:

i) Sjn Tad jika berkaitan dengan tadbir.

ii) PC/KS jika berkaitan dengan Cawangan- Cawangan atau Sel-Sel.\strut
\end{minipage}\tabularnewline
\begin{minipage}[t]{0.30\columnwidth}\raggedright
Sjn Tad/PC/KS \textbar{} 5.\strut
\end{minipage} & \begin{minipage}[t]{0.30\columnwidth}\raggedright
\begin{verbatim}
              | Ter
\end{verbatim}

\begin{enumerate}
\def\labelenumi{\arabic{enumi}.}
\setcounter{enumi}{5}
\item
\end{enumerate}\strut
\end{minipage} & \begin{minipage}[t]{0.30\columnwidth}\raggedright
ima surat perkhidmatan biasa dan utusan dari PAR/Kerani.

Semak dan majukan untuk perhatian Pengurus Besar/PS 1/PS 2/PS 3.\strut
\end{minipage}\tabularnewline
\begin{minipage}[t]{0.30\columnwidth}\raggedright
Pengurus Besar/PS 1/PS 3 \textbar{}\strut
\end{minipage} & \begin{minipage}[t]{0.30\columnwidth}\raggedright
2/PS \textbar{} 7. \textbar{} s 8.\strut
\end{minipage} & \begin{minipage}[t]{0.30\columnwidth}\raggedright
\begin{verbatim}
    | Semak dan teliti
\end{verbatim}

urat perkhidmatan biasa dan utusan serta tandatangan ringkas.

Majukan kepada Sjn Tad/PC/KS untuk diambil tindakan berpandukan catatan
Pengurus Besar/PS 1/PS 2/PS 3.\strut
\end{minipage}\tabularnewline
\bottomrule
\end{longtable}

\begin{longtable}[]{@{}lll@{}}
\toprule
\begin{minipage}[b]{0.30\columnwidth}\raggedright
Sjn Tad/PC/KS \textbar{} 9.\strut
\end{minipage} & \begin{minipage}[b]{0.30\columnwidth}\raggedright
Ter

10.\strut
\end{minipage} & \begin{minipage}[b]{0.30\columnwidth}\raggedright
ima kembali surat perkhidmatan biasa dan utusan.

Serahkan kepada PT/Kerani.\strut
\end{minipage}\tabularnewline
\midrule
\endhead
\begin{minipage}[t]{0.30\columnwidth}\raggedright
PT/Kerani \textbar{} 11.\strut
\end{minipage} & \begin{minipage}[t]{0.30\columnwidth}\raggedright
\begin{verbatim}
              | Rek
\end{verbatim}
\strut
\end{minipage} & \begin{minipage}[t]{0.30\columnwidth}\raggedright
od dan failkan.\strut
\end{minipage}\tabularnewline
\begin{minipage}[t]{0.30\columnwidth}\raggedright
\strut
\end{minipage} & \begin{minipage}[t]{0.30\columnwidth}\raggedright
\begin{enumerate}
\def\labelenumi{\Alph{enumi}.}
\setcounter{enumi}{1}
\item
\begin{verbatim}
           | PER
\end{verbatim}
\end{enumerate}\strut
\end{minipage} & \begin{minipage}[t]{0.30\columnwidth}\raggedright
GERAKAN FAIL/DOKUMEN\strut
\end{minipage}\tabularnewline
\begin{minipage}[t]{0.30\columnwidth}\raggedright
PC/Sjn Tad/KS \textbar{} 1.\strut
\end{minipage} & \begin{minipage}[t]{0.30\columnwidth}\raggedright
\begin{verbatim}
              | Ter
\end{verbatim}

\begin{enumerate}
\def\labelenumi{\arabic{enumi}.}
\setcounter{enumi}{1}
\item
\end{enumerate}\strut
\end{minipage} & \begin{minipage}[t]{0.30\columnwidth}\raggedright
ima surat/nota permohonan pengeluaran fail/dokumen dari Pengurus
Besar/PS 1/PS 2/PS 3/PC.

Kemukakan surat permohonan kepada PS 1/

PS 2 dan dapatkan kelulusannya jika permohonan dari Jabatan/agensi
luar.\strut
\end{minipage}\tabularnewline
\begin{minipage}[t]{0.30\columnwidth}\raggedright
PS 1/PS 2 \textbar{} 3.\strut
\end{minipage} & \begin{minipage}[t]{0.30\columnwidth}\raggedright
\begin{verbatim}
              | Ber
\end{verbatim}
\strut
\end{minipage} & \begin{minipage}[t]{0.30\columnwidth}\raggedright
i kelulusan dan serahkan surat permohonan di para 2 kepada PC/Sjn Tad/KS
setelah di tandatangan ringkas.\strut
\end{minipage}\tabularnewline
\begin{minipage}[t]{0.30\columnwidth}\raggedright
PC/Sjn Tad/KS \textbar{} 4.\strut
\end{minipage} & \begin{minipage}[t]{0.30\columnwidth}\raggedright
\begin{verbatim}
              | Ter
\end{verbatim}

\begin{enumerate}
\def\labelenumi{\arabic{enumi}.}
\setcounter{enumi}{4}
\item
\end{enumerate}\strut
\end{minipage} & \begin{minipage}[t]{0.30\columnwidth}\raggedright
ima dan failkan surat permohonan di para 2 dan 3.

Arahkan PT/Kerani menyediakan surat iringan/Memo pengesahan pengeluaran
fail/dokumen.\strut
\end{minipage}\tabularnewline
\begin{minipage}[t]{0.30\columnwidth}\raggedright
PT/Kerani \textbar{} 6.\strut
\end{minipage} & \begin{minipage}[t]{0.30\columnwidth}\raggedright
\begin{verbatim}
              | Rek
\end{verbatim}

\begin{enumerate}
\def\labelenumi{\arabic{enumi}.}
\setcounter{enumi}{6}
\item
\item
\end{enumerate}\strut
\end{minipage} & \begin{minipage}[t]{0.30\columnwidth}\raggedright
od dan sahkan pengeluaran fail/dokumen.

Buat tindakan susulan untuk pengembalian fail/dokumen (jika perlu).

Rekod dan sahkan fail/dokumen yang telah dikembalikan.\strut
\end{minipage}\tabularnewline
\bottomrule
\end{longtable}

\begin{longtable}[]{@{}lll@{}}
\toprule
\begin{minipage}[b]{0.30\columnwidth}\raggedright
\strut
\end{minipage} & \begin{minipage}[b]{0.30\columnwidth}\raggedright
C. \textbar{} PEN\strut
\end{minipage} & \begin{minipage}[b]{0.30\columnwidth}\raggedright
UTUPAN FAIL/FAIL MATI\strut
\end{minipage}\tabularnewline
\midrule
\endhead
\begin{minipage}[t]{0.30\columnwidth}\raggedright
PT/Kerani \textbar{} 1.\strut
\end{minipage} & \begin{minipage}[t]{0.30\columnwidth}\raggedright
\begin{verbatim}
              | Sem
\end{verbatim}

\begin{enumerate}
\def\labelenumi{\arabic{enumi}.}
\setcounter{enumi}{1}
\item
\item
\end{enumerate}\strut
\end{minipage} & \begin{minipage}[t]{0.30\columnwidth}\raggedright
ak fail/dokumen yang perlu dibuat penutupan.

Catat `TUTUP PADA' dan tarikh di sebelah luar kulit fail/dokumen.

Simpan fail/dokumen yang telah di buat penutupan ke dalam stor.\strut
\end{minipage}\tabularnewline
\begin{minipage}[t]{0.30\columnwidth}\raggedright
\strut
\end{minipage} & \begin{minipage}[t]{0.30\columnwidth}\raggedright
\begin{enumerate}
\def\labelenumi{\Alph{enumi}.}
\setcounter{enumi}{3}
\item
\begin{verbatim}
           | [KE
\end{verbatim}
\end{enumerate}\strut
\end{minipage} & \begin{minipage}[t]{0.30\columnwidth}\raggedright
SELAMATAN DOKUMEN DI BILIK KEBAL (PENGAMBILAN DOKUMEN/FAIL){]}\{.small
caps\}\strut
\end{minipage}\tabularnewline
\begin{minipage}[t]{0.30\columnwidth}\raggedright
KS Bilik Kebal \textbar{} 1.\strut
\end{minipage} & \begin{minipage}[t]{0.30\columnwidth}\raggedright
\begin{verbatim}
              | Sem
\end{verbatim}

\begin{enumerate}
\def\labelenumi{\arabic{enumi}.}
\setcounter{enumi}{1}
\item
\end{enumerate}\strut
\end{minipage} & \begin{minipage}[t]{0.30\columnwidth}\raggedright
ak borang permohonan dan pengeluaran fail/dokumen dan sahkan

Serah permohonan kepada Kerani Bilik Kebal\strut
\end{minipage}\tabularnewline
\begin{minipage}[t]{0.30\columnwidth}\raggedright
Kerani Bilik \textbar{} 3 Kebal \textbar{}\strut
\end{minipage} & \begin{minipage}[t]{0.30\columnwidth}\raggedright
. \textbar{} T \textbar{} s 4.

\begin{enumerate}
\def\labelenumi{\arabic{enumi}.}
\setcounter{enumi}{4}
\item
\item
\end{enumerate}\strut
\end{minipage} & \begin{minipage}[t]{0.30\columnwidth}\raggedright
erima dari KS buat emakan dalam Sistem Pengurusan Fail

Rekodkan dan pastikan jalur dan rak fail yang dimohon adalah betul

Sekiranya fail tidak ditemui rujuk semula kepada KS untuk semakan rak
dan jalur

Kumpulkan dokumen/fail yang ditemui untuk semakan KS\strut
\end{minipage}\tabularnewline
\begin{minipage}[t]{0.30\columnwidth}\raggedright
KS Bilik Kebal \textbar{} 7.\strut
\end{minipage} & \begin{minipage}[t]{0.30\columnwidth}\raggedright
\begin{verbatim}
              | Ter
\end{verbatim}

\begin{enumerate}
\def\labelenumi{\arabic{enumi}.}
\setcounter{enumi}{7}
\item
\end{enumerate}\strut
\end{minipage} & \begin{minipage}[t]{0.30\columnwidth}\raggedright
ima fail/dokumen yang dipohon untuk semakan kali kedua

Setelah disahkan serah dokumen/fail kepada pemohon (Kerani Sel)\strut
\end{minipage}\tabularnewline
\begin{minipage}[t]{0.30\columnwidth}\raggedright
Kerani Sel \textbar{} 9.\strut
\end{minipage} & \begin{minipage}[t]{0.30\columnwidth}\raggedright
\begin{verbatim}
              | Tan
\end{verbatim}
\strut
\end{minipage} & \begin{minipage}[t]{0.30\columnwidth}\raggedright
datangan borang pinjaman fail setelah dibuat semakan\strut
\end{minipage}\tabularnewline
\bottomrule
\end{longtable}

\begin{longtable}[]{@{}lll@{}}
\toprule
\begin{minipage}[b]{0.30\columnwidth}\raggedright
\strut
\end{minipage} & \begin{minipage}[b]{0.30\columnwidth}\raggedright
10.\strut
\end{minipage} & \begin{minipage}[b]{0.30\columnwidth}\raggedright
Failkan borang permohonan dokumen/fail dari Bilik Kebal\strut
\end{minipage}\tabularnewline
\midrule
\endhead
\begin{minipage}[t]{0.30\columnwidth}\raggedright
\strut
\end{minipage} & \begin{minipage}[t]{0.30\columnwidth}\raggedright
\begin{enumerate}
\def\labelenumi{\Alph{enumi}.}
\setcounter{enumi}{4}
\item
\begin{verbatim}
           | PEM
\end{verbatim}
\end{enumerate}\strut
\end{minipage} & \begin{minipage}[t]{0.30\columnwidth}\raggedright
ULANGAN FAIL/DOKUMEN KE BILIK KEBAL\strut
\end{minipage}\tabularnewline
\begin{minipage}[t]{0.30\columnwidth}\raggedright
Penyelia/KS \textbar{} 1.\strut
\end{minipage} & \begin{minipage}[t]{0.30\columnwidth}\raggedright
\begin{verbatim}
              | Ken
\end{verbatim}

\begin{enumerate}
\def\labelenumi{\arabic{enumi}.}
\setcounter{enumi}{1}
\item
\end{enumerate}\strut
\end{minipage} & \begin{minipage}[t]{0.30\columnwidth}\raggedright
alpasti fail/dokumen yang hendak dihantar ke Bilik Kebal.

Arahkan kerani /PT membuat semakan.\strut
\end{minipage}\tabularnewline
\begin{minipage}[t]{0.30\columnwidth}\raggedright
Kerani/PT \textbar{} 3.\strut
\end{minipage} & \begin{minipage}[t]{0.30\columnwidth}\raggedright
\begin{verbatim}
              | Sem
\end{verbatim}

\begin{enumerate}
\def\labelenumi{\arabic{enumi}.}
\setcounter{enumi}{3}
\item
\end{enumerate}\strut
\end{minipage} & \begin{minipage}[t]{0.30\columnwidth}\raggedright
ak dan rekodkan dalam Buku Rekod penghantaran fail/dokumen.

Hantar kepada Kerani Bilik Kebal.\strut
\end{minipage}\tabularnewline
\begin{minipage}[t]{0.30\columnwidth}\raggedright
Kerani Bilik \textbar{} 5 Kebal \textbar{}\strut
\end{minipage} & \begin{minipage}[t]{0.30\columnwidth}\raggedright
. \textbar{} S \textbar{} 6.

\begin{enumerate}
\def\labelenumi{\arabic{enumi}.}
\setcounter{enumi}{6}
\item
\item
\end{enumerate}\strut
\end{minipage} & \begin{minipage}[t]{0.30\columnwidth}\raggedright
ahkan penerimaan.

Rekodkan:

\begin{enumerate}
\def\labelenumi{\roman{enumi})}
\tightlist
\item
  Nombor rak penyimpanan fail/dokumen.
\end{enumerate}

ii) Keluar/masuk fail (jika ada permintaan).

Pastikan fail/dokumen yang dipulangkan ditempatkan dirak yang telah
ditetapkan

Pastikan kad keluar/masuk dimasukan ke dalam kes fail dan menentukan
tempat di mana kes fail akan disimpan mengikut susunan peruntukan nombor
tentera.\strut
\end{minipage}\tabularnewline
\begin{minipage}[t]{0.30\columnwidth}\raggedright
KS Bilik Kebal \textbar{} 9.\strut
\end{minipage} & \begin{minipage}[t]{0.30\columnwidth}\raggedright
\begin{verbatim}
              | Sem
\end{verbatim}

\begin{enumerate}
\def\labelenumi{\arabic{enumi}.}
\setcounter{enumi}{9}
\item
\end{enumerate}\strut
\end{minipage} & \begin{minipage}[t]{0.30\columnwidth}\raggedright
ak rekod pemulangan dokumen/fail yang telah dipulangkan

Pastikan penukaran jalur dan rak fail direkodkan dalam Sistem Pengurusan
Fail Syarikat\strut
\end{minipage}\tabularnewline
\bottomrule
\end{longtable}

\begin{longtable}[]{@{}lll@{}}
\toprule
\begin{minipage}[b]{0.30\columnwidth}\raggedright
KS Bilik Kebal \textbar{} 11.\strut
\end{minipage} & \begin{minipage}[b]{0.30\columnwidth}\raggedright
Sem

\begin{enumerate}
\def\labelenumi{\arabic{enumi}.}
\setcounter{enumi}{11}
\item
\item
\end{enumerate}

14.\strut
\end{minipage} & \begin{minipage}[b]{0.30\columnwidth}\raggedright
ak dokumen/fail yang tidak dipulangkan melebihi sebulan dan sahkan
dokumen tersebut masih berada di pegangan Kerani Sel

Jika dokumen/fail didapati hilang atau tiada dalam pegangan Kerani Sel.

Laporkan kepada PS 1 Operasi untuk tindakan selanjutnya

Patuh kepada Arahan Kerja seperti di Kembaran A\strut
\end{minipage}\tabularnewline
\midrule
\endhead
\begin{minipage}[t]{0.30\columnwidth}\raggedright
PS 1 Ops \textbar{} 15.\strut
\end{minipage} & \begin{minipage}[t]{0.30\columnwidth}\raggedright
\begin{verbatim}
              | Ter
\end{verbatim}

\begin{enumerate}
\def\labelenumi{\arabic{enumi}.}
\setcounter{enumi}{15}
\item
\item
\end{enumerate}\strut
\end{minipage} & \begin{minipage}[t]{0.30\columnwidth}\raggedright
ima laporan dan ambil Tindakan pembetulan dan pencegahan

Buat siasatan dan adakan Lembaga Siasatan (Jika Perlu)

Hasil Siasatan dimaklumkan kepada Wakil Pengurusan\strut
\end{minipage}\tabularnewline
\begin{minipage}[t]{0.30\columnwidth}\raggedright
\strut
\end{minipage} & \begin{minipage}[t]{0.30\columnwidth}\raggedright
\begin{enumerate}
\def\labelenumi{\Alph{enumi}.}
\setcounter{enumi}{4}
\item
\begin{verbatim}
           | PEL
\end{verbatim}
\end{enumerate}\strut
\end{minipage} & \begin{minipage}[t]{0.30\columnwidth}\raggedright
UPUSAN FAIL\strut
\end{minipage}\tabularnewline
\begin{minipage}[t]{0.30\columnwidth}\raggedright
PT/Kerani \textbar{} 1.\strut
\end{minipage} & \begin{minipage}[t]{0.30\columnwidth}\raggedright
\begin{verbatim}
              | Sem
\end{verbatim}

\begin{enumerate}
\def\labelenumi{\arabic{enumi}.}
\setcounter{enumi}{1}
\item
\end{enumerate}\strut
\end{minipage} & \begin{minipage}[t]{0.30\columnwidth}\raggedright
ak dan senaraikan fail/dokumen yang perlu di buat pelupusan.

Majukan senarai kepada Sjn Tad/PC/KS untuk diteliti.\strut
\end{minipage}\tabularnewline
\begin{minipage}[t]{0.30\columnwidth}\raggedright
Sjn Tad/PC/KS \textbar{} 3.\strut
\end{minipage} & \begin{minipage}[t]{0.30\columnwidth}\raggedright
\begin{verbatim}
              | Tel
\end{verbatim}

\begin{enumerate}
\def\labelenumi{\arabic{enumi}.}
\setcounter{enumi}{3}
\item
\item
\end{enumerate}\strut
\end{minipage} & \begin{minipage}[t]{0.30\columnwidth}\raggedright
iti senarai pelupusan fail/dokumen.

Sediakan surat permohonan pelupusan kepada Pengurus Besar untuk
kelulusan.

Majukan surat permohonan pelupusan kepada PS 3 untuk
ditandatangani.\strut
\end{minipage}\tabularnewline
\bottomrule
\end{longtable}

\begin{longtable}[]{@{}lll@{}}
\toprule
\begin{minipage}[b]{0.30\columnwidth}\raggedright
PS 3 \textbar{} 6.\strut
\end{minipage} & \begin{minipage}[b]{0.30\columnwidth}\raggedright
Sah

7.\strut
\end{minipage} & \begin{minipage}[b]{0.30\columnwidth}\raggedright
kan surat permohonan pelupusan fail/dokumen.

Rekod dan majukan surat permohonan kepada Pengurus Besar.\strut
\end{minipage}\tabularnewline
\midrule
\endhead
\begin{minipage}[t]{0.30\columnwidth}\raggedright
Pengurus Besar\strut
\end{minipage} & \begin{minipage}[t]{0.30\columnwidth}\raggedright
\begin{verbatim}
 | 8.
\end{verbatim}

\begin{enumerate}
\def\labelenumi{\arabic{enumi}.}
\setcounter{enumi}{8}
\item
\item
\item
\end{enumerate}\strut
\end{minipage} & \begin{minipage}[t]{0.30\columnwidth}\raggedright
\begin{verbatim}
 | Terima surat
\end{verbatim}

permohonan pelupusan fail/dokumen dari PS 3.

Semak dan pastikan permohonan adalah bersesuaian dengan peraturan di
perenggan 3.1 hingga 3.5

Sahkan dan catat kelulusan.

Majukan permohonan yang telah diluluskan kepada PS 3.\strut
\end{minipage}\tabularnewline
\begin{minipage}[t]{0.30\columnwidth}\raggedright
PS 3 \textbar{} 12.\strut
\end{minipage} & \begin{minipage}[t]{0.30\columnwidth}\raggedright
\begin{verbatim}
              | Ter
\end{verbatim}

\begin{enumerate}
\def\labelenumi{\arabic{enumi}.}
\setcounter{enumi}{12}
\item
\item
\item
\item
\end{enumerate}\strut
\end{minipage} & \begin{minipage}[t]{0.30\columnwidth}\raggedright
ima permohonan yang telah diluluskan dari Pengurus Besar.

Majukan surat kebenaran pelupusan fail/dokumen ke Jabatan Arkib Negara.

Terima surat kebenaran pelupusan dari Jabatan Arkib Negara.

Laksanakan pelupusan.

Rekodkan fail/dokumen yang telah dibuat pelupusan.\strut
\end{minipage}\tabularnewline
\bottomrule
\end{longtable}

\begin{enumerate}
\def\labelenumi{\arabic{enumi}.}
\setcounter{enumi}{6}
\tightlist
\item
  CARTA ALIRAN KERJA
\end{enumerate}

7.1 Penerimaan dan Pendaftaran

7.2 Pergerakan Fail/Dokumen

7.3 Keselamatan Dokumen di Bilik Kebal (Pengambilan Dokumen/Fail)

7.4 Pemulangan Fail/Dokumen ke Bilik Kebal

7.5 Pelupusan Fail/Dokumen

8.0 REKOD KUALITI

\begin{longtable}[]{@{}llrl@{}}
\toprule
BIL REKO & D KUALITI LOKASI & TEMPOH PENYI & MPANAN\tabularnewline
\midrule
\endhead
1 & Buku Daftar Surat/Utusan & Pejabat Rekod/PC/PJ 5 &
tahun\tabularnewline
2. & Buku Daftar Pergerakan Fail/Dokumen & Pejabat Rekod/PS 3/PC/PJ 5 &
tahun\tabularnewline
7. & Buku Daftar Pelupusan Fail/Dokumen & Pejabat Rekod/PS 3/PC/PJ 7 &
tahun\tabularnewline
8. & Buku Rekod Pinjaman Fail/Dokumen Di Bilik Kebal & Pejabat Bilik
Kebal & 5 tahun\tabularnewline
\bottomrule
\end{longtable}

9.0 Kembaran

Kembaran A -- Arahan Kerja Pengurusan Fail/Dokumen Di Bilik Kebal

{KEMBARAN A}

{ARAHAN KERJA PENGURUSAN FAIL/DOKUMEN DI BILIK KEBAL}

{Rujuk:}

A. PK(W).OA.02

B. FAFSI 1961

C. Arahan Keselamatan BSPP

D. DKM.OA.01

{AM}

1. Bilik Kebal adalah kawasan larangan bagi semua peringkat kakitangan
melainkan membuat tugas-tugas tertentu sahaja. Kakitangan yang bertugas
di Bilik Kebal adalah terdiri dari kakitangan yang telah ditetapkan.

2. Tiap-tiap kes fail/dokumen yang hendak dikeluarkan atau dipulangkan
semula mestilah disemak, direkodkan dan didaftarkan. Kes fail/dokumen
tidak dibenarkan untuk dipinjamkan keluar dari Jabatan Arah Rekod dan
Pencen melainkan dengan kebenaran Pengurus Besar Rekod dan Pencen
sendiri.

3. Semua kes fail/dokumen yang dikeluarkan mestilah dipulangkan ke Bilik
Kebal dalam jangka masa dua minggu dari tarikh dikeluarkan. Sekiranya
tindakan-tindakan yang masih belum selesai diambil, maka Cawangan,
Bahagian atau Sel berkenaan hendaklah memberitahu Ketua Sel Bilik Kebal
semasa memulangkan fail/dokumen yang lain itu.

{CARA MEMOHON}

4. Permohonan untuk mengeluarkan kes fail/dokumen mestilah menggunakan
borang seperti di \emph{Kembaran A} dan pemohon-pemohon dikehendaki
mengisi dan tandatangan ringkas dibahagian borang tersebut dan disahkan
oleh Ketua Sel Bilik Kebal.

{TINDAKAN UNTUK MENGELUARKAN FAIL/DOKUMEN}

5. Apabila menerima permohonan, Ketua Sel Bilik Kebal akan menyemak,
mengesahkan dan mengarahkan Kerani Bilik Kebal untuk menyemak kedudukan
fail/dokumen dalam Sistem Pengurusan Fail bagi menentukan jalur dan rak
fail/dokumen tersebut. Apabila kedudukan fail/dokumen telah diketahui,
Kerani Bilik Kebal akan mencari dan mengeluarkan kes fail/dokumen yang
dikehendaki.

A - 1

Masa yang diberikan untuk mencari fail tersebut ialah lebih kurang 45
minit dan ini mengikut beberapa banyak fail yang dikeluarkan.

6. Kerani dari Cawangan/Bahagian/Sel tidak dibenarkan memasuki sendiri
untuk mencari fail/dokumen yang dikehendaki. Apabila fail/dokumen yang
dimohon telah siap dikumpulkan kerani sel boleh datang untuk mengambil

{TINDAKAN UNTUK MEMULANGKAN}

7. Setelah tindakan diambil, semua kes fail/dokumen yang dipinjam
hendaklah dipulangkan dengan kadar segera melainkan yang masih diambil
tindakan. Kerani Bilik Kebal mestilah mengesahkan penerimaan dan
rekodkan.

8. Kerani Sel Bilik Kebal dikehendaki memastikan fail/dokumen yang
dipulangkan ditempatkan dirak dan jalurkan yang telah ditetapkan.
Pastikan kad keluar/masuk dimasukan semula ke dalam fail tersebut
mengikut susunan nombor tentera.

9. Ketua Sel Bilik Kebal semak rekod pemulangan fail/dokumen yang telah
dipulangkan dan pastikan jalur dan rak yang direkodkan betul.

10. Semak semula fail/dokumen yang tidak dipulangkan melebihi sebulan
berada dalam pegangan Kerani Sel. Sekiranya dokumen tersebut didapati
hilang laporkan kepada PS 1 Ops untuk tindakan selanjutnya.

{TINDAKAN KETUA SEL BILIK KEBAL}

11. Pada tiap-tiap bulan Ketua Bilik Kebal dikehendaki menyemak kes-kes
fail/dokumen yang mana belum dipulangkan oleh Cawangan,Bahagian dan Sel
dan mengingatkan Penyelia Cawangan/Bahagian/Sel untuk memulangkannya.
Sekiranya terdapat ada fail/dokumen yang hilang, maka hendaklah
melaporkan kepada PS 1 Ops untuk diambil tindakan pembetulan dan
pencegahan dan laporan dimajukan ke Wakil Pengurusan.

{LANGKAH KESELAMATAN DAN PERKARA-PERKARA YANG PERLU DIPATUHI BAGI
MENGELAKKAN KEBAKARAN}

12. Punca-punca kebakaran tiada dapat diramalkan tetapi dari aspek-aspek
berikut yang boleh berlaku kebakaran jika tidak diawasi dengan rapi:

A - 2

a. Puntong rokok yang belum dipadam dengan sepenuhnya dibuang di dalam
bakul sampah atau dibuang dimerata tempat pada saat-saat meninggal Bilik
Kebal semasa habis kerja.

b. Suis tidak ditutup (Off) dan plugnya tidak dicabut selepas meninggal
Bilik Kebal semasa habis waktu kerja

12. Untuk mengelak dan mencegah kebakaran, perkara-perkara berikut
perlulah dititikberatkan dan hendaklah dipatuhi oleh semua kakitangan di
dalam Jabatan Arah ini terutama sekali kepada kakitangan yang
mengelolakan Bilik Kebal.

a. Puntong-puntong rokok mestilah dipadamkan apinya terlebih dahulu
sebelum dibuang ke dalam tempat-tempat yang dikhaskan dan dilarang sama
sekali dibuang di merata-rata tempat.

b. Dilarang sama sekali merokok di Bilik Kebal.

c. Semua alat-alat elektrik mestilah dijaga berada di dalam keadaan baik
dan boleh digunakan.

d. Semua alat-alat elekterik yang rosak hendaklah ditanggalkan daripada
penyambungan aliran elektrik dan tidak boleh digunakan sama sekali.

e. Apa jua pendawaian elektrik mestilah mendapat kebenaran dan kelulusan
pihak JKR Kem KEMENTAH

f. Semua kerosakan pendawaian elektrik mestilah dilaporkan dengan segera
ke Pejabat Rekod untuk tindakan sewajarnya.

g. Semua Staf Bilik Kebal adalah bertanggungjawab menentukan semua suis
elektrik ditutup (off) dan plak-plaknya dicabut apabila habis waktu
kerja.

h. Dilarang sama sekali membakar sampah dimerata tempat di Bilik Kebal.

13. Tindakan sekiranya berlaku kebakaran:

(1) Cuba memadamkan api itu dengan alat-alat pemadam yang disediakan dan
tutupkan semua suis alat-alat elektrik.

A - 3

(2) Laporkan perkara tersebut kepada Pegawai Kebakaran/Keselamatan
Syarikat

(3) Jika api merebak dan tidak dapat dikawal lapor terus kepada
pihak-pihak seperti berikut:

(a) Balai Bomba.

(b) Kem KEMENTAH.

(c) Bilik DOR.

(d) Pengurus Besar Rekod dan Pencen atau PS 1 Operasi/Pengurusan.

(e) Kakitangan-kakitangan yang lain hendaklah keluar dari pejabat
berkenaan dan berbaris di Dataran Syarikat sekiranya kebakaran berlaku
di waktu kerja.

A - 4

{LAMPIRAN 1}

{BOR/PK(W).OA.02}

\includegraphics[width=0.94861in,height=0.83194in]{media/image39.jpg}

PERMOHONAN PENGELUARAN FAIL ANGGOTA DARI BILIK KEBAL

BUTIR-BUTIR PERMOHON:

1. No, Ten/IC:\ldots\ldots\ldots\ldots\ldots\ldots\ldots\ldots..
Pkt:\ldots\ldots\ldots..
Nama:\ldots\ldots\ldots\ldots\ldots\ldots\ldots\ldots\ldots\ldots\ldots\ldots\ldots\ldots\ldots\ldots\ldots{}

\begin{enumerate}
\def\labelenumi{\arabic{enumi}.}
\setcounter{enumi}{1}
\tightlist
\item
  Cawangan/Bhg/Sel:\ldots\ldots\ldots\ldots\ldots\ldots\ldots\ldots\ldots\ldots\ldots\ldots\ldots\ldots\ldots\ldots\ldots\ldots\ldots\ldots\ldots\ldots\ldots\ldots\ldots\ldots\ldots\ldots\ldots\ldots..
\end{enumerate}

3. Samb
\url{Tel:/}\ldots.............................................................\ldots.

4. Jumlah Fail yang
dikehendaki:\ldots\ldots\ldots\ldots\ldots\ldots\ldots\ldots\ldots\ldots\ldots\ldots.

\begin{longtable}[]{@{}lll@{}}
\toprule
Bil No T & en/Pkt/Nama Fail/Dokumen Yang Dimohon Catatan
&\tabularnewline
\midrule
\endhead
(a) (b) & (c) &\tabularnewline
\bottomrule
\end{longtable}

.....................................\ldots.
..............................................................................

(Tandatangan Pemohon) (Disahkan oleh KS/Pertugas Bilik Kebal)

Tarikh:
No:......................................................................\ldots{}

Pkt:........................................................................

Nama:................................................................\ldots.

Jawatan:...............................................................

Tarikh:................................................................\ldots{}

\end{document}
